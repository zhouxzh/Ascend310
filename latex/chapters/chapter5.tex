\section{章节总览}\label{ux7ae0ux8282ux603bux89c8}

本章聚焦``定位 → 解释 → 改善''闭环:从性能分析模型、Profiling
工具、瓶颈模式分类、布局与精度策略、内存与并行调度、到自定义算子开发与验证标准,提供工程可落地方法。目标是让读者具备:A)
定量证明问题;B) 选择低风险优化策略;C) 保证功能与性能回归一致性。

\section{性能拆解与衡量框架}\label{ux6027ux80fdux62c6ux89e3ux4e0eux8861ux91cfux6846ux67b6}

总时延公式:\passthrough{\lstinline!T\_total = T\_pre + T\_h2d + T\_infer + T\_d2h + T\_post + T\_idle!}。
吞吐上限受制于 \passthrough{\lstinline!max(T\_component)!};需收集: -
平均/分位数 (p50/p95); - 波动系数 CV=std/mean(\textgreater0.15
需进一步剖析); - 稳定性:长跑 1h 是否存在漂移 (内存泄漏或热降频)。
对比优化前后必须保留固定随机种子和数据集,消除噪声。

\section{Profiling
工具与时间线解读}\label{profiling-ux5de5ux5177ux4e0eux65f6ux95f4ux7ebfux89e3ux8bfb}

关键观测元素: \textbar{} 轨迹 \textbar{} 意义 \textbar{} 异常信号
\textbar{} \textbar{} ---- \textbar{} ---- \textbar{} --------
\textbar{} \textbar{} Stream Timeline \textbar{} 内核调度顺序 \textbar{}
大量空洞 gap \textbar{} \textbar{} MemCopy \textbar{} H2D/D2H 开销
\textbar{} 频繁小块拷贝 \textbar{} \textbar{} Task Kernel \textbar{}
算子执行 \textbar{} 个别算子异常拖长 \textbar{} \textbar{} Sync/Wait
\textbar{} Host 等待 \textbar{} Wait 占比高 \textbar{}

使用策略: 1. 先全量 Profile → 定位热点范围; 2. 二次局部
Profile(过滤特定算子类型); 3. 导出 JSON → 自动解析器归档:算子耗时
TOPK,Copy 占比,Idle 时间。

\section{瓶颈模式与处置策略矩阵}\label{ux74f6ux9888ux6a21ux5f0fux4e0eux5904ux7f6eux7b56ux7565ux77e9ux9635}

\begin{longtable}[]{@{}
  >{\raggedright\arraybackslash}p{(\linewidth - 8\tabcolsep) * \real{0.1176}}
  >{\raggedright\arraybackslash}p{(\linewidth - 8\tabcolsep) * \real{0.2353}}
  >{\raggedright\arraybackslash}p{(\linewidth - 8\tabcolsep) * \real{0.2353}}
  >{\raggedright\arraybackslash}p{(\linewidth - 8\tabcolsep) * \real{0.2941}}
  >{\raggedright\arraybackslash}p{(\linewidth - 8\tabcolsep) * \real{0.1176}}@{}}
\toprule\noalign{}
\begin{minipage}[b]{\linewidth}\raggedright
模式
\end{minipage} & \begin{minipage}[b]{\linewidth}\raggedright
识别特征
\end{minipage} & \begin{minipage}[b]{\linewidth}\raggedright
定量指标
\end{minipage} & \begin{minipage}[b]{\linewidth}\raggedright
处置优先级
\end{minipage} & \begin{minipage}[b]{\linewidth}\raggedright
策略
\end{minipage} \\
\midrule\noalign{}
\endhead
\bottomrule\noalign{}
\endlastfoot
调度空洞 & Timeline gap 多 & Idle \textgreater{} 10\% & 高 & 合并小算子
/ 预加载数据 \\
访存受限 & 算子耗时与内存带宽正相关 & 算子内核利用率低 & 中 & Layout
变换 / Tile 分块 \\
H2D 瓶颈 & MemCopy 比例高 & H2D\textgreater20\% & 高 &
合并/异步/Pin/AIPP 下沉 \\
后处理拖慢 & Post\textgreater25\% & NMS/Decode 长 & 中 & 并行化 / Device
化 \\
量化退化 & INT8 未获收益 & 时延差\textless10\% & 低 &
重新校准/混合精度 \\
单 Stream 阻塞 & 单流串行 & Stream=1 & 中 & 多流/流水线 \\
\end{longtable}

优先处理``结构性收益''\textgreater``微优化'',避免局部手工 hack
影响可维护性。

\section{Layout /
内存访问优化}\label{layout-ux5185ux5b58ux8bbfux95eeux4f18ux5316}

常见格式:NCHW(框架常用)、NHWC(部分算子优化)、NC1HWC0(硬件友好对齐),转换策略:在数据首次落地时转换一次;若前后模型不同布局,以中间标准布局连接,减少重复重排。
对齐:通道/宽高按 16/32 边界对齐可提升访存一致性;小通道 (\textless16)
可考虑 \passthrough{\lstinline!--enable\_small\_channel!}
以加载优化内核。 缓存复用:多模型共享中间 Buffer(需尺寸与 dtype
一致),通过分配表管理生命周期。

\section{精度与性能的层级折衷}\label{ux7cbeux5ea6ux4e0eux6027ux80fdux7684ux5c42ux7ea7ux6298ux8877}

\begin{longtable}[]{@{}llll@{}}
\toprule\noalign{}
精度层级 & 描述 & 性能收益 & 风险 \\
\midrule\noalign{}
\endhead
\bottomrule\noalign{}
\endlastfoot
FP32 & 基准 & - & 内存带宽/算力高 \\
FP16 & 半精度 & 1.2\textasciitilde1.6x & 累积误差 \\
INT8 对称 & 量化整型 & 1.5\textasciitilde2.2x & 量化噪声 \\
混合精度 & 局部高精度 & 中等 & 实现复杂 \\
\end{longtable}

量化流程要点: 1. 收集代表性校准集(覆盖光照/尺度/类别分布); 2.
校准统计(MinMax / KL); 3. 评估 Top1/Top5
差异、关键指标差异(mAP/F1)。 误差定位:Dump 中间张量(FP32 vs INT8)→
层级误差分布 → 定位失真层(常见:激活饱和/尺度不均衡)。

\section{内存管理专题}\label{ux5185ux5b58ux7ba1ux7406ux4e13ux9898}

策略: 1. 长期 Buffer:模型 I/O、常量 Workspace; 2. 短期 Buffer:Batch
临时中间; 3. 建立内存池(按 size class 分类
1KB/4KB/16KB/64KB/大块),分配 → 归还; 4. 避免频繁
\passthrough{\lstinline!aclrtMalloc/Free!}:使用池化接口封装; 5.
监控:每 60s 记录一次池使用率与系统剩余内存,突增后回收未引用对象; 6.
大对象对齐:按 512B/4KB 对齐减少碎片。

\section{并行与流水线}\label{ux5e76ux884cux4e0eux6d41ux6c34ux7ebf}

多 Stream:将独立算子或多模型分离到不同 Stream 并行调度;注意 Host
侧同步点过多会抵消收益。Pipeline:Pre → Infer → Post 分线程队列,目标是
In-Flight 帧数达到平衡(过多增加延迟,过少利用率低)。
自适应调度:定期评估每阶段平均耗时,动态调整线程池大小(PID 控制思想)。

\section{自定义算子开发与评估}\label{ux81eaux5b9aux4e49ux7b97ux5b50ux5f00ux53d1ux4e0eux8bc4ux4f30}

决策条件: \textbar{} 条件 \textbar{} 必须满足至少一项 \textbar{}
\textbar{} ---- \textbar{} ---------------- \textbar{} \textbar{}
复合算子频繁出现 \textbar{} 合并降低访存 \textbar{} \textbar{}
内置实现回退 Host \textbar{} 存在高额拷贝 \textbar{} \textbar{}
内核模式不适配输入规模 \textbar{} 小尺寸性能差 \textbar{}

流程:需求分析 → JSON 定义(\passthrough{\lstinline!op\_type!}, attr,
inputs/outputs) → Kernel C++ 模板 (向量化 / Tile) → 编译注册 → ATC 识别
→ 功能单测(随机张量对比)→ 性能对比(3 次 Warmup + 50 次统计)。
评估表: \textbar{} 版本 \textbar{} 输入规模 \textbar{} 平均耗时(us)
\textbar{} P95(us) \textbar{} 访存次数 \textbar{} 速度提升 \textbar{}
备注 \textbar{} \textbar{} ---- \textbar{} -------- \textbar{}
----------- \textbar{} ------- \textbar{} -------- \textbar{} --------
\textbar{} ---- \textbar{}

\section{优化案例:Add + ReLU
融合}\label{ux4f18ux5316ux6848ux4f8badd-relu-ux878dux5408}

原始:Add → ReLU 两个算子各自读写内存; 融合:单 Kernel 计算
\passthrough{\lstinline!out = relu(a+b)!}:减少一次读写;
收益估算:内存带宽主导场景中延迟≈(T\_add + T\_relu - 重叠),实际提升
10\textasciitilde25\%。 验证:随机输入 100 次 → 检查数值一致(允许 1e-6
FP16 差异)→ Benchmark 对比。

\section{性能报告与回归模板}\label{ux6027ux80fdux62a5ux544aux4e0eux56deux5f52ux6a21ux677f}

\begin{lstlisting}
{
    "commit": "<git-sha>",
    "model": "resnet50_fp16",
    "batch": 1,
    "avg_latency_ms": 5.87,
    "p95_latency_ms": 6.24,
    "throughput_fps": 170.3,
    "h2d_ms_ratio": 0.11,
    "post_ms_ratio": 0.05,
    "memory_peak_mb": 486,
    "temperature_c_range": "54-58",
    "profiling_date": "2025-09-04T10:21:00Z"
}
\end{lstlisting}

自动化:CI 中若 \passthrough{\lstinline!avg\_latency\_ms!} 高于基线 5\%
→ 标红注释。

\section{章节小结}\label{ux7ae0ux8282ux5c0fux7ed3}

性能优化不等于盲调:应以数据驱动 +
分层定位为前提,先解决架构级与内存/拷贝问题,再考虑算子级微调与自定义算子开发。量化收益需伴随精度风险评估,内存与并行策略需要可观测支撑。

\section{实践任务}\label{ux5b9eux8df5ux4efbux52a1}

\begin{enumerate}
\def\labelenumi{\arabic{enumi}.}
\tightlist
\item
  对一个部署模型收集 Profiling JSON,输出前 5 算子耗时与占比表。
\item
  实现 H2D 合并:将 3 个连续小拷贝合并为单次,比较平均时延改善。
\item
  尝试 INT8 量化:输出精度与性能对比(Top1/Latency/FPS)。
\item
  编写一个 Add+ReLU 融合算子伪代码 + 预期性能提升估算。
\item
  生成基线性能报告,并设定 CI 回归阈值策略文本说明。
\end{enumerate}

\section{昇腾310B自定义算子开发全流程}\label{ux6607ux817e310bux81eaux5b9aux4e49ux7b97ux5b50ux5f00ux53d1ux5168ux6d41ux7a0b}

本节面向 Ascend 310B
推理场景,给出``什么时候需要自定义算子、用什么方法开发、如何编译注册、怎样验证与上线''的系统指引。读完后,你应能独立完成一个简单自定义算子的端到端落地。

\subsection{开发概述}\label{ux5f00ux53d1ux6982ux8ff0}

\begin{itemize}
\tightlist
\item
  目标:当模型中存在``内置算子不支持/性能欠佳/需要业务特化融合''的场景,通过自定义算子(Custom
  Op)补齐功能或获得确定性性能收益。
\item
  实现形态:

  \begin{itemize}
  \tightlist
  \item
    AI Core(TBE/TE/TIK,运行于 NPU
    核心,适合数值密集型向量/矩阵计算)。
  \item
    AICPU(C++/CPU 实现,在 Host/AICPU
    执行,适合控制流或少量数据处理,注意 H2D/D2H 开销)。
  \end{itemize}
\item
  产物:算子描述(op info/proto)、算子实现(AI Core: Python
  实现并编译为内核;AICPU: C++ so)、注册与打包(放入 OPP 路径),以及
  ATC 与运行时可识别的元数据。
\item
  适配 310B:选择
  \passthrough{\lstinline!soc\_version=Ascend310B!},优先 FP16
  数据通路;对齐 NC1HWC0 等硬件友好布局;小通道/小尺寸注意 tile 策略。
\end{itemize}

\subsection{开发的理论基础}\label{ux5f00ux53d1ux7684ux7406ux8bbaux57faux7840}

\begin{enumerate}
\def\labelenumi{\arabic{enumi}.}
\tightlist
\item
  硬件/内存模型(简要):

  \begin{itemize}
  \tightlist
  \item
    GM(Global Memory):大容量全局显存,带宽高、时延高;
  \item
    UB(Unified Buffer):片上高速缓冲,容量有限,需 tile 分块搬运;
  \item
    Vector/Scalar 单元:提供 vadd/vmul/vmax
    等向量指令,需保证数据对齐(通常以 16/32 对齐)。
  \item
    DMA:GM 与 UB 之间的数据搬运,批量大块优于频繁小块。
  \end{itemize}
\item
  计算表达与调度:

  \begin{itemize}
  \tightlist
  \item
    TE(Tensor Expression):描述计算公式与算子图(compute);
  \item
    Schedule:描述分块(tilling)、并行、缓存、向量化等执行计划;
  \item
    TIK DSL:更接近硬件指令级的编程接口,适合精细控制。
  \end{itemize}
\item
  算子契约(Operator Contract):

  \begin{itemize}
  \tightlist
  \item
    输入/输出张量的 shape、dtype、format(如
    NCHW/NC1HWC0)、属性(attr);
  \item
    广播/对齐规则、边界行为(溢出/饱和/舍入)、精度策略(FP16/FP32
    混合)。
  \end{itemize}
\item
  形状推断与动态 shape:

  \begin{itemize}
  \tightlist
  \item
    ATC 需要根据 op 描述完成 shape infer;
  \item
    动态尺寸需在实现中处理 tile 策略切换并保证 UB 不溢出。
  \end{itemize}
\end{enumerate}

\subsection{开发流程(AI Core
为例)}\label{ux5f00ux53d1ux6d41ux7a0bai-core-ux4e3aux4f8b}

以下流程以一个``Add+ReLU 融合''示例说明,读者可据此扩展到实际业务算子。

\begin{enumerate}
\def\labelenumi{\arabic{enumi})}
\tightlist
\item
  环境准备
\end{enumerate}

\begin{itemize}
\tightlist
\item
  确保 CANN/Toolkit 已安装,能使用
  \passthrough{\lstinline!atc!}、Profiling 等工具;
\item
  设置环境变量:

  \begin{itemize}
  \tightlist
  \item
    \passthrough{\lstinline!ASCEND\_INSTALL\_PATH!} 指向 Toolkit 根;
  \item
    \passthrough{\lstinline!ASCEND\_OPP\_PATH!} 指向 OPP 包路径(custom
    算子将被放置于此);
  \item
    \passthrough{\lstinline!soc\_version=Ascend310B!}(ATC/编译时指定)。
  \end{itemize}
\end{itemize}

\begin{enumerate}
\def\labelenumi{\arabic{enumi})}
\setcounter{enumi}{1}
\tightlist
\item
  定义算子信息(op info/proto)
\end{enumerate}

\begin{itemize}
\tightlist
\item
  指定:\passthrough{\lstinline!op\_type!}、inputs/outputs
  名称、dtype/format 组合、属性列表、融合类型等;
\item
  作用:

  \begin{itemize}
  \tightlist
  \item
    供 ATC 做图解析、形状推断与算子选择;
  \item
    供运行时校验输入输出与 kernel 适配。
  \end{itemize}
\end{itemize}

\begin{enumerate}
\def\labelenumi{\arabic{enumi})}
\setcounter{enumi}{2}
\tightlist
\item
  编写算子实现(TE/TBE)
\end{enumerate}

\begin{itemize}
\item
  计算表达: ```python \# 伪代码:y = relu(x1 + x2) import te.lang.cce
  as tbe from te import tvm

  def add\_relu\_compute(x1, x2): y = tbe.vadd(x1, x2) z = tbe.vmaxs(y,
  tvm.const(0.0, x1.dtype)) return z ```
\item
  调度策略(示例要点):

  \begin{itemize}
  \tightlist
  \item
    选择合适的 tile 以满足 UB 容量;
  \item
    将连续内存访问向量化,减少非对齐访问;
  \item
    尽量合并搬运,减少 GM\textless-\textgreater UB 往返;
  \item
    小尺寸场景避免过度拆分导致调度开销占比过高。
  \end{itemize}
\end{itemize}

\begin{enumerate}
\def\labelenumi{\arabic{enumi})}
\setcounter{enumi}{3}
\tightlist
\item
  编译与注册
\end{enumerate}

\begin{itemize}
\tightlist
\item
  使用官方提供的 TBE
  编译入口生成内核与元数据(具体命令因版本而异,遵循已安装 Toolkit
  的说明);
\item
  将生成的实现文件/元数据放入
  \passthrough{\lstinline!ASCEND\_OPP\_PATH!} 下的 custom 目录(如
  \passthrough{\lstinline!op\_impl/custom/ai\_core/tbe!}、\passthrough{\lstinline!op\_proto/custom!})。
\end{itemize}

\begin{enumerate}
\def\labelenumi{\arabic{enumi})}
\setcounter{enumi}{4}
\tightlist
\item
  与 ATC 集成
\end{enumerate}

\begin{itemize}
\tightlist
\item
  在模型转换时指定 \passthrough{\lstinline!--soc\_version=Ascend310B!};
\item
  确保 ATC 能从 \passthrough{\lstinline!ASCEND\_OPP\_PATH!} 读取到你的
  op 描述与实现信息;
\item
  若需要限制实现选择,可使用
  \passthrough{\lstinline!--op\_select\_implmode!} 配合算子实现指示。
\end{itemize}

\begin{enumerate}
\def\labelenumi{\arabic{enumi})}
\setcounter{enumi}{5}
\tightlist
\item
  运行时部署与加载
\end{enumerate}

\begin{itemize}
\tightlist
\item
  运行环境中需要包含同样的 OPP 目录(含 custom 实现);
\item
  应用进程启动时配置环境变量,使 Runtime 能定位自定义算子实现;
\item
  按常规 ACL 流程加载 OM 并执行推理。
\end{itemize}

\begin{enumerate}
\def\labelenumi{\arabic{enumi})}
\setcounter{enumi}{6}
\tightlist
\item
  验证与度量
\end{enumerate}

\begin{itemize}
\tightlist
\item
  功能正确性:与参考实现(NumPy/ONNXRuntime)对齐,随机多组张量比较(均值绝对误差、相对误差、边界样本)。
\item
  性能评估:Warmup 3 次 + 采样 50 次,输出
  avg/p95/FPS;对比内置算子或未融合版本;
\item
  资源占用:Profiling 检查 MemCopy 占比、Kernel 占比、Idle;
\item
  兼容性:不同 shape/dtype/format 组合覆盖测试。
\end{itemize}

\begin{enumerate}
\def\labelenumi{\arabic{enumi})}
\setcounter{enumi}{7}
\tightlist
\item
  文档与产物归档
\end{enumerate}

\begin{itemize}
\tightlist
\item
  输出
  \passthrough{\lstinline!op\_contract.yaml!}(IO/Attr/格式/边界规则);
\item
  输出
  \passthrough{\lstinline!benchmark.json!}(avg/p95、对比基线、硬件/版本信息);
\item
  产物目录:\passthrough{\lstinline!op\_pkg/<op\_type>/<version>/\{op\_proto, op\_impl, tests, docs\}!}。
\end{itemize}

\subsection{AICPU 路线(可选)}\label{aicpu-ux8defux7ebfux53efux9009}

\begin{itemize}
\tightlist
\item
  适用:控制流、轻量数据处理或暂不需在 NPU 上运行的功能性算子;
\item
  实现:C/C++ 编写,遵循 AICPU 接口,注册到相应目录生成动态库;
\item
  注意:Host 执行会引入 H2D/D2H;若在性能关键路径,优先 AI Core 版本。
\end{itemize}

\subsection{常见问题与排错}\label{ux5e38ux89c1ux95eeux9898ux4e0eux6392ux9519}

\begin{itemize}
\tightlist
\item
  ATC 提示 Unsupported Op:检查 op info 是否被正确放置且生效;确认
  \passthrough{\lstinline!soc\_version!} 与路径;
\item
  运行时 Fallback:确认实现 dtype/format 与模型一致;必要时扩充
  \passthrough{\lstinline!dtype\_format!} 组合;
\item
  性能未达预期:增大 tile、减少小块 DMA、合并计算、检查是否出现额外
  layout 转换;
\item
  精度差异:检查饱和/舍入策略、对齐与广播规则、数据范围(FP16 溢出)。
\end{itemize}

\subsection{本章小结}\label{ux672cux7ae0ux5c0fux7ed3}

自定义算子是 310B
场景下``功能补齐与性能确定性''的关键手段。核心抓手包括:明确契约(IO/格式/属性)、用
TE/TIK 描述计算并设计合理调度、放在 OPP 中正确注册、生效于 ATC
与运行时、用可度量的基线进行功能/性能回归。建议从``融合与复合算子''起步,优先选择计算密集、访存友好的目标,循序渐进积累模板与脚手架,以降低维护成本。
