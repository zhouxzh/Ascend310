\section{章节总览}\label{ux7ae0ux8282ux603bux89c8}

本章提供``鸟瞰 + 上手 + 约定 +
协作''四个维度:帮助读者在开始代码与实验前,建立清晰地图、完成环境自检、理解术语规范,并加入协作迭代。阅读后应能:A)
明确个人学习路径;B) 快速完成最小可行部署;C) 识别后续章节间的依赖关系。

\section{全书主线结构}\label{ux5168ux4e66ux4e3bux7ebfux7ed3ux6784}

技术主线:硬件与环境 (1) → 软件栈与转换 (2) → 边缘系统视角 (3) →
典型部署实践 (4) → 性能与算子优化 (5) → 高可用工程体系 (6) →
方法论与交付 (7) → 综合案例 (8) → 工具与附录 (9)。 知识图谱建议:

\begin{lstlisting}
硬件/板卡 → CANN 组件 → 模型转换 → 推理编程 → 多模型流水线 → 性能调优 → 系统可靠性 → 交付方法论 → 案例复现
\end{lstlisting}

\section{读者路径矩阵}\label{ux8bfbux8005ux8defux5f84ux77e9ux9635}

\begin{longtable}[]{@{}
  >{\raggedright\arraybackslash}p{(\linewidth - 8\tabcolsep) * \real{0.1111}}
  >{\raggedright\arraybackslash}p{(\linewidth - 8\tabcolsep) * \real{0.2222}}
  >{\raggedright\arraybackslash}p{(\linewidth - 8\tabcolsep) * \real{0.1667}}
  >{\raggedright\arraybackslash}p{(\linewidth - 8\tabcolsep) * \real{0.2222}}
  >{\raggedright\arraybackslash}p{(\linewidth - 8\tabcolsep) * \real{0.2778}}@{}}
\toprule\noalign{}
\begin{minipage}[b]{\linewidth}\raggedright
角色
\end{minipage} & \begin{minipage}[b]{\linewidth}\raggedright
起步路径
\end{minipage} & \begin{minipage}[b]{\linewidth}\raggedright
可跳过
\end{minipage} & \begin{minipage}[b]{\linewidth}\raggedright
深挖章节
\end{minipage} & \begin{minipage}[b]{\linewidth}\raggedright
目标里程碑
\end{minipage} \\
\midrule\noalign{}
\endhead
\bottomrule\noalign{}
\endlastfoot
零基础 & 1 → 2 → 4 & 5 深度优化细节 & 8 案例 & 跑通首个端到端推理 \\
嵌入式 & 1 → 2 → 5 → 6 & 7 方法论部分 & 5/6 性能与可靠性 &
优化资源占比 \\
AI 应用 & 2 → 4 → 7 → 8 & 1 硬件细节 & 4/8 部署差异 & 多任务流水线 \\
技术负责人 & 0 → 3 → 6 → 7 & 具体算子实现 & 7 评测体系 & 制定团队标准 \\
\end{longtable}

\section{硬件准备与兼容性}\label{ux786cux4ef6ux51c6ux5907ux4e0eux517cux5bb9ux6027}

\begin{longtable}[]{@{}llll@{}}
\toprule\noalign{}
组件 & 推荐 & 说明 & 检查点 \\
\midrule\noalign{}
\endhead
\bottomrule\noalign{}
\endlastfoot
开发板 & OrangePi AIpro 310B & 标准平台 & npu-smi 识别型号 \\
存储 & TF 64G+ / SSD & 加速 I/O & iostat 延迟 \textless10ms \\
散热 & 风扇+鳍片 & 长时间稳定 & 温度 \textless{} 85°C \\
摄像头 & USB UVC / MIPI & 即插即用 & v4l2-ctl 列设备 \\
网络 & 千兆以太网 & 低抖动 & ping 丢包率 ≈0 \\
电源 & PD 65W & 稳定供电 & 无随机重启 \\
\end{longtable}

准备完成后记录
\passthrough{\lstinline!hardware\_inventory.md!}:型号、序列号、固件版本、功耗模式。

\section{软件与工具栈细化}\label{ux8f6fux4ef6ux4e0eux5de5ux5177ux6808ux7ec6ux5316}

\begin{longtable}[]{@{}lll@{}}
\toprule\noalign{}
层级 & 工具/组件 & 说明 \\
\midrule\noalign{}
\endhead
\bottomrule\noalign{}
\endlastfoot
OS & Ubuntu 22.04 / openEuler & 官方验证环境 \\
驱动/固件 & 对应 CANN 版本 & 版本矩阵对齐 \\
CANN & Toolkit + Runtime & 提供 atc/acl/profiling \\
Python & 3.10+ & 脚本与评测 \\
依赖 & numpy/onnx/onnxruntime/opencv & 模型与预处理 \\
调试 & npu-smi/Profiler/日志系统 & 性能与稳定性分析 \\
\end{longtable}

建议创建 \passthrough{\lstinline!requirements.txt!} 并使用 venv 或 Conda
隔离。

\section{仓库目录与命名约定}\label{ux4ed3ux5e93ux76eeux5f55ux4e0eux547dux540dux7ea6ux5b9a}

\begin{longtable}[]{@{}lll@{}}
\toprule\noalign{}
目录 & 内容 & 约定 \\
\midrule\noalign{}
\endhead
\bottomrule\noalign{}
\endlastfoot
src/book & 文本章节 & 章节号前缀固定 \\
experiments & 案例 & caseX 模式 \\
models & 原始/导出中间模型 & 按模型名/版本 \\
scripts & 通用脚本 & 跨平台 \passthrough{\lstinline!.sh/.ps1!} \\
tools & 辅助分析脚本 & 单一功能命令化 \\
docs & 生成 PDF / 图 & 不放大模型文件 \\
benchmarks & 性能记录 & 时间戳 + commit \\
\end{longtable}

命名:\passthrough{\lstinline!<model>\_<precision>\_<shape>.om!},例如
\passthrough{\lstinline!yolov5s\_fp16\_1x3x640x640.om!}。

\section{最小可行环境验证
(MVE)}\label{ux6700ux5c0fux53efux884cux73afux5883ux9a8cux8bc1-mve}

执行脚本 \passthrough{\lstinline!scripts/verify\_env.sh!}(建议添加):

\begin{enumerate}
\def\labelenumi{\arabic{enumi}.}
\tightlist
\item
  \passthrough{\lstinline!npu-smi info!}:输出芯片与状态;
\item
  \passthrough{\lstinline!atc --version!}:版本号记录;
\item
  运行随机张量推理(内置简单 OM 或最小网络)验证 ACL API;
\item
  Profiling 采集一次,生成 timeline 文件;
\item
  记录结果写入 \passthrough{\lstinline!env\_report.json!}。
\end{enumerate}

判定:如某步骤失败阻断后续章节学习。

\section{全局术语与约定}\label{ux5168ux5c40ux672fux8bedux4e0eux7ea6ux5b9a}

\begin{longtable}[]{@{}lll@{}}
\toprule\noalign{}
术语 & 约定 & 说明 \\
\midrule\noalign{}
\endhead
\bottomrule\noalign{}
\endlastfoot
FPS & frames/second & 统计处理输出帧数 \\
Latency & ms & 端到端完成时间 \\
Pxx & 分位数 & P95/P99 评估抖动 \\
Pipeline & 阶段组 & 多阶段并行结构 \\
Signature & 模型签名 & I/O 名称与形状/格式 json \\
Baseline & 初始基线 & 第一版性能/精度记录 \\
\end{longtable}

所有时间单位默认 ms;数据大小默认字节(显式写 MB/GiB
时需指出换算基数)。

\section{协作工作流与质量闸门}\label{ux534fux4f5cux5de5ux4f5cux6d41ux4e0eux8d28ux91cfux95f8ux95e8}

工作流:Issue(需求/缺陷)→ 分支
\passthrough{\lstinline!feat|fix/<topic>!} → 提交(含描述)→ PR →
自动测试(Lint+精度/性能轻测)→ Review → Merge。 质量闸门:

\begin{longtable}[]{@{}lll@{}}
\toprule\noalign{}
闸门 & 说明 & 未通过处理 \\
\midrule\noalign{}
\endhead
\bottomrule\noalign{}
\endlastfoot
Lint & 代码/文档格式 & 修复后再提交 \\
Spell & 关键术语拼写 & 更正 \\
Signature 验证 & 模型签名一致 & 拒绝合并 \\
基线回归 & 性能/精度差异超阈值 & 标注需说明 \\
\end{longtable}

PR 模板字段:Motivation / Changes / Test / Risk / Rollback Plan。

\section{学习与实践建议}\label{ux5b66ux4e60ux4e0eux5b9eux8df5ux5efaux8bae}

\begin{enumerate}
\def\labelenumi{\arabic{enumi}.}
\tightlist
\item
  完成前 3 章后立即挑选一个轻量模型跑通部署(建立正反馈)。
\item
  每章输出``总结卡片'':知识点 → 应用场景 → 潜在风险。
\item
  建议建立个人实验日志:参数、结果、疑问与下一步假设。
\item
  失败样本收集:创建 \passthrough{\lstinline!failure\_cases/!}
  目录存储误检/漏检图像用于持续改进。
\end{enumerate}

\section{常见初学误区与规避}\label{ux5e38ux89c1ux521dux5b66ux8befux533aux4e0eux89c4ux907f}

\begin{longtable}[]{@{}lll@{}}
\toprule\noalign{}
误区 & 结果 & 规避 \\
\midrule\noalign{}
\endhead
\bottomrule\noalign{}
\endlastfoot
直接优化无基线 & 无从评估收益 & 先建立 baseline \\
混用不同预处理 & 精度随机波动 & 抽象统一函数 \\
缺少签名文件 & 部署时出错 & 每次转换生成签名 \\
未记录环境版本 & 难以复现 & env\_report.json \\
长日志未切割 & 磁盘占满 & 配置滚动策略 \\
\end{longtable}

\section{章节小结}\label{ux7ae0ux8282ux5c0fux7ed3}

通过环境、目录、术语、协作流程的标准化,后续学习聚焦问题本身,而不是环境与沟通摩擦。建议读者在继续前先完成``最小可行环境验证''并记录结果,以便后续调试时快速排除环境因素。

\section{实践任务}\label{ux5b9eux8df5ux4efbux52a1}

\begin{enumerate}
\def\labelenumi{\arabic{enumi}.}
\tightlist
\item
  撰写 \passthrough{\lstinline!hardware\_inventory.md!} 与
  \passthrough{\lstinline!env\_report.json!}(可手动草拟)。
\item
  建立 \passthrough{\lstinline!requirements.txt!}
  并安装依赖,记录安装耗时。
\item
  创建一个最小随机张量 OM 推理脚本并输出结果摘要。
\item
  制定个人 4 周学习计划(章节→目标→产出)。
\end{enumerate}
