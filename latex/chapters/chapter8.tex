\section{章节总览}\label{ux7ae0ux8282ux603bux89c8}

本章通过九个真实应用场景串联前面章节的知识:模型选择、转换、部署、性能与稳定性验证、迭代优化。所有案例采用统一模板,支持快速复制与对比评估。强调``结构化指标
+ 自动化脚本 + 可视化反馈''。

\section{案例统一模板(标准化规范)}\label{ux6848ux4f8bux7edfux4e00ux6a21ux677fux6807ux51c6ux5316ux89c4ux8303}

\begin{longtable}[]{@{}lll@{}}
\toprule\noalign{}
区块 & 内容要点 & 产出文件 \\
\midrule\noalign{}
\endhead
\bottomrule\noalign{}
\endlastfoot
场景描述 & 背景/输入/目标 & README.md\#scene \\
指标目标 & 延迟/FPS/精度/资源 & requirement.yaml \\
模型选择 & 候选对比 + 取舍 & model\_card*.md \\
数据准备 & 采集/标注/增强 & data\_prep.md \\
转换部署 & 导出→ATC 参数 & atc.sh / export.py \\
运行脚本 & 启动/参数/日志路径 & run.sh / run.ps1 \\
性能结果 & metrics.json (基线/优化) & metrics/*.json \\
质量验证 & 精度/漂移检查 & accuracy.json \\
风险改进 & 已知问题/迭代计划 & roadmap.md \\
\end{longtable}

\section{案例目录结构规范}\label{ux6848ux4f8bux76eeux5f55ux7ed3ux6784ux89c4ux8303}

\begin{lstlisting}
experiments/caseX/
  README.md
  requirement.yaml
  models/         # onnx / om / signatures
  scripts/
    export.py
    atc.sh
    run_infer.py
    benchmark.py
  data/           # 样本(或下载指令)
  metrics/
    baseline.json
    optimized.json
  logs/
  eval/
    accuracy.json
    drift.json
  assets/         # 截图/示意图
\end{lstlisting}

\section{例概览与重点}\label{ux4f8bux6982ux89c8ux4e0eux91cdux70b9}

\begin{longtable}[]{@{}lllll@{}}
\toprule\noalign{}
序 & 名称 & 关键技术点 & 指标核心 & 风险要素 \\
\midrule\noalign{}
\endhead
\bottomrule\noalign{}
\endlastfoot
1 & 人脸打卡机 & 人脸检测+比对+活体 & 识别成功率/伪拒率 & 光照/遮挡 \\
2 & 实时跟踪 & 检测+多目标关联 & 跟踪稳定度(IDF1) & 遮挡/抖动 \\
3 & 智能电子琴 & 音频节拍识别+分类 & 识别延迟/准确率 & 噪声/延迟 \\
4 & 掌纹识别 & ROI 提取+特征匹配 & 误识率/拒识率 & 采集姿态 \\
5 & 数据采集仪 & 传感融合+缓存上传 & 数据丢失率 & 网络波动 \\
6 & 智能小车 & 目标检测+路径策略 & 决策延迟 & 传感器同步 \\
7 & 智能相册 & 分类+聚类+去重 & 聚类纯度 & 相似干扰 \\
8 & 手势识别 & 时序建模(TSM) & 手势准确率/FPS & 动作模糊 \\
9 & 聊天机器人 & NLP 推理+缓存 & 响应时延/意图准确 & 语料漂移 \\
\end{longtable}

下列示例详细展开前三个具代表性的模式。

\section{案例
1:人脸打卡机}\label{ux6848ux4f8b-1ux4ebaux8138ux6253ux5361ux673a}

\subsection{场景}\label{ux573aux666f}

摄像头实时输入,人脸检测→关键点对齐→特征提取→特征库比对→授权决策→事件上报。

\subsection{指标}\label{ux6307ux6807}

\begin{longtable}[]{@{}lll@{}}
\toprule\noalign{}
指标 & 目标 & 说明 \\
\midrule\noalign{}
\endhead
\bottomrule\noalign{}
\endlastfoot
平均识别时延 & \textless{} 120ms & 从帧采集到结果 \\
最大 P95 & \textless{} 150ms & 抖动控制 \\
误识率(FAR) & \textless{} 0.001 & 安全性 \\
拒识率(FRR) & \textless{} 0.02 & 体验 \\
\end{longtable}

\subsection{模型链路}\label{ux6a21ux578bux94feux8def}

\begin{enumerate}
\def\labelenumi{\arabic{enumi}.}
\tightlist
\item
  人脸检测 (RetinaFace);
\item
  5 点关键点仿射对齐;
\item
  ArcFace 特征 512D;
\item
  向量归一化 + 余弦相似度;
\item
  阈值自适应(基于滑动窗口均值校正)。
\end{enumerate}

\subsection{性能优化}\label{ux6027ux80fdux4f18ux5316}

\begin{itemize}
\tightlist
\item
  批量特征比对:向量库转矩阵,使用 SIMD/BLAS;
\item
  缓存:最近识别通过用户特征缓存,减少重复比对;
\item
  光照增强:低光阈值触发 Gamma/直方图均衡。
\end{itemize}

\subsection{metrics 示例}\label{metrics-ux793aux4f8b}

\begin{lstlisting}
{
  "avg_latency_ms": 98.4,
  "p95_latency_ms": 121.3,
  "fps": 10.1,
  "face_detect_ms": 42.1,
  "feature_ms": 18.7,
  "match_ms": 5.2,
  "false_accept_rate": 0.0008,
  "false_reject_rate": 0.017
}
\end{lstlisting}

\section{案例 2:实时跟踪(检测 +
关联)}\label{ux6848ux4f8b-2ux5b9eux65f6ux8ddfux8e2aux68c0ux6d4b-ux5173ux8054}

\subsection{流程}\label{ux6d41ux7a0b}

帧采集 → 目标检测 → 外观特征提取 → 卡尔曼预测 → 匈牙利匹配 → 轨迹输出。

\subsection{难点}\label{ux96beux70b9}

遮挡/丢失:轨迹生命周期管理(状态:Tentative → Confirmed → Lost →
Removed)。

\subsection{优化}\label{ux4f18ux5316}

\begin{enumerate}
\def\labelenumi{\arabic{enumi}.}
\tightlist
\item
  检测降频:每 N 帧做一次全检测,中间帧仅跟踪预测;
\item
  多线程:检测与跟踪解耦;
\item
  ReID 模型轻量化(裁剪通道)。
\end{enumerate}

\subsection{评估指标}\label{ux8bc4ux4f30ux6307ux6807}

IDF1、MOTA、FP/FN、IDSW(身份切换)。

\section{案例
3:智能电子琴(音频)}\label{ux6848ux4f8b-3ux667aux80fdux7535ux5b50ux7434ux97f3ux9891}

\subsection{流程}\label{ux6d41ux7a0b-1}

音频采集 16kHz → 窗口分帧 FFT → 频谱/梅尔特征 → 分类模型(音符/节奏)→
校准节拍输出。

\subsection{优化点}\label{ux4f18ux5316ux70b9}

FFT
批处理使用向量库;低延迟滑动窗口;模型输出置信度平滑(指数滑动平均)。

\subsection{指标}\label{ux6307ux6807-1}

节拍延迟 \textless{} 80ms;识别准确率 \textgreater{} 95\%。

\section{结果记录与差异报告}\label{ux7ed3ux679cux8bb0ux5f55ux4e0eux5deeux5f02ux62a5ux544a}

基线与优化版本差异自动生成:

\begin{longtable}[]{@{}lllll@{}}
\toprule\noalign{}
指标 & baseline & optimized & 差异 & 状态 \\
\midrule\noalign{}
\endhead
\bottomrule\noalign{}
\endlastfoot
avg\_latency\_ms & 112.5 & 98.4 & -12.5\% & ✅ \\
p95\_latency\_ms & 140.3 & 121.3 & -13.5\% & ✅ \\
false\_accept\_rate & 0.0012 & 0.0008 & 改善 & ✅ \\
\end{longtable}

\section{自动化与复现保障}\label{ux81eaux52a8ux5316ux4e0eux590dux73b0ux4fddux969c}

\begin{longtable}[]{@{}ll@{}}
\toprule\noalign{}
机制 & 说明 \\
\midrule\noalign{}
\endhead
\bottomrule\noalign{}
\endlastfoot
Hash 校验 & onnx/om/脚本确保未篡改 \\
repeatable seed & 设定随机种子统一实验 \\
benchmark.py & 统一输出 metrics.json \\
drift 检测 & 周期性对比指标偏差 \\
一键脚本 & run.sh + run.ps1 支持跨平台 \\
\end{longtable}

\section{指标可视化建议}\label{ux6307ux6807ux53efux89c6ux5316ux5efaux8bae}

\begin{itemize}
\tightlist
\item
  时间序列:Latency / FPS / 温度。
\item
  箱线图:不同优化阶段的时延分布。
\item
  堆叠条:阶段占比(检测/特征/比对)。
\item
  散点:光照水平 vs 识别准确度。
\end{itemize}

\section{通用问题经验库}\label{ux901aux7528ux95eeux9898ux7ecfux9a8cux5e93}

\begin{longtable}[]{@{}llll@{}}
\toprule\noalign{}
问题 & 案例 & 根因 & 处理 \\
\midrule\noalign{}
\endhead
\bottomrule\noalign{}
\endlastfoot
相机丢帧 & 1/2 & 帧率不稳 & 缓冲+限速 \\
模型加载慢 & 全部 & 冷启动未预热 & 预加载预热10次 \\
OCR 错字 & 新增 & 图像模糊 & 降噪/锐化 \\
跟踪漂移 & 2 & 过度遮挡 & reinit + 短期外观缓存 \\
\end{longtable}

\section{扩展方向}\label{ux6269ux5c55ux65b9ux5411}

\begin{itemize}
\tightlist
\item
  多模态融合(视觉+语音指令)。
\item
  硬件加速协同(NPU + DSP 解码)。
\item
  大模型边缘裁剪(蒸馏 + 量化 + 分层推理)。
\end{itemize}

\section{贡献工作流}\label{ux8d21ux732eux5de5ux4f5cux6d41}

\begin{enumerate}
\def\labelenumi{\arabic{enumi}.}
\tightlist
\item
  Fork → 分支:\passthrough{\lstinline!case/<name>!};
\item
  新建目录遵循模板;
\item
  提交包含:README、metrics、脚本、model\_card;
\item
  CI 自动校验格式与 hash;
\item
  PR 模板填写:动机/数据/指标/风险。
\end{enumerate}

\section{章节小结}\label{ux7ae0ux8282ux5c0fux7ed3}

案例是知识的验证与反哺:通过统一模板与自动化度量,形成可延展的案例库,帮助新模型与新任务快速落地并保障质量。

\section{实践任务}\label{ux5b9eux8df5ux4efbux52a1}

\begin{enumerate}
\def\labelenumi{\arabic{enumi}.}
\tightlist
\item
  搭建 case1 目录,生成 baseline metrics。
\item
  实现 face detection + feature 比对流程,并输出 FAR/FRR。
\item
  将一次优化(裁剪/量化)前后差异写入 diff 表。
\item
  编写 benchmark.py:支持
  \passthrough{\lstinline!--repeat N --output metrics.json!}。
\item
  增加 drift 检测脚本(比较两次 metrics 差异,阈值报警)。
\end{enumerate}
