昇腾310B在通用算子覆盖广度上已能满足大多数推理任务,但在以下场景,自定义算子(Custom
Op)能显著提升功能完备性与性能确定性:模型含未支持/半支持算子、复合算子频繁导致访存过多、需要业务特化(如阈值/形态学/后处理融合)、或内置实现对特定尺寸/布局性能欠佳。第三章将给出``为什么、怎么做、如何验证与上线''的完整路径。

\section{算子开发概述}\label{ux7b97ux5b50ux5f00ux53d1ux6982ux8ff0}

\begin{itemize}
\tightlist
\item
  目标与收益:

  \begin{itemize}
  \tightlist
  \item
    功能补齐:覆盖模型图中未支持或语义差异较大的算子;
  \item
    性能确定性:融合多算子、减少GM\textless-\textgreater UB搬运与中间落地、利用向量化内核;
  \item
    工程可维护:以``算子契约''形式固化输入/输出/属性与边界行为,便于回归与复用。
  \end{itemize}
\item
  执行形态:

  \begin{itemize}
  \tightlist
  \item
    AI Core(推荐):基于 TBE/TE/TIK 运行于 NPU 核心,适合数值密集型;
  \item
    AICPU(可选):C/C++ 在 AICPU/Host
    侧执行,适合控制流/轻量处理(注意H2D/D2H成本)。
  \end{itemize}
\item
  产物要素:

  \begin{itemize}
  \tightlist
  \item
    算子描述(op info/proto):声明
    op\_type、inputs/outputs、dtype\_format 组合、属性与形状推断;
  \item
    算子实现(Kernel):TE/TIK 计算+调度或 AICPU C++ 实现;
  \item
    注册与打包:产物按规范放入 OPP 目录,ATC/Runtime 可发现与加载。
  \end{itemize}
\end{itemize}

\section{开发的理论基础}\label{ux5f00ux53d1ux7684ux7406ux8bbaux57faux7840}

\begin{enumerate}
\def\labelenumi{\arabic{enumi})}
\tightlist
\item
  硬件与存储层次:
\end{enumerate}

\begin{itemize}
\tightlist
\item
  GM(Global Memory):容量大、带宽高;
\item
  UB(Unified Buffer):片上高速缓存,容量有限;
\item
  DMA:GM↔UB 的数据搬运,偏好大块连续传输;
\item
  向量/标量单元:支持vadd/vmul/vmax等,需数据对齐(常见16/32)。
\end{itemize}

\begin{enumerate}
\def\labelenumi{\arabic{enumi})}
\setcounter{enumi}{1}
\tightlist
\item
  计算表达与调度:
\end{enumerate}

\begin{itemize}
\tightlist
\item
  TE(Tensor Expression)描述计算公式;Schedule 负责
  tile/并行/向量化/缓存;
\item
  TIK 提供更贴近硬件的 DSL,便于精细控制 DMA 与 UB 管理;
\item
  目标:以较少的GM往返在UB内完成尽可能多的计算,提升算子算子效率与吞吐。
\end{itemize}

\begin{enumerate}
\def\labelenumi{\arabic{enumi})}
\setcounter{enumi}{2}
\tightlist
\item
  算子契约(Operator Contract):
\end{enumerate}

\begin{itemize}
\tightlist
\item
  输入/输出张量的
  shape、dtype、layout(NCHW/NC1HWC0等)、属性(如alpha、mode);
\item
  广播与对齐规则、边界行为(溢出/饱和/舍入)、精度策略(FP16/FP32混合);
\item
  动态shape与静态shape:实现需覆盖契约内的形状组合并保证UB不溢出。
\end{itemize}

\begin{enumerate}
\def\labelenumi{\arabic{enumi})}
\setcounter{enumi}{3}
\tightlist
\item
  数值与精度:
\end{enumerate}

\begin{itemize}
\tightlist
\item
  FP16 常用于 310B 推理通路;必要时在关键步骤采用临时 FP32 计算再回写;
\item
  误差控制:选择合适的舍入策略,避免饱和/下溢导致NAN/INF。
\end{itemize}

\section{开发流程(AI Core
路线)}\label{ux5f00ux53d1ux6d41ux7a0bai-core-ux8defux7ebf}

\begin{enumerate}
\def\labelenumi{\arabic{enumi}.}
\tightlist
\item
  环境准备与约束
\end{enumerate}

\begin{itemize}
\tightlist
\item
  安装 CANN/Toolkit 并确认 \passthrough{\lstinline!atc --version!}
  正常;
\item
  设置环境变量:\passthrough{\lstinline!ASCEND\_INSTALL\_PATH!}、\passthrough{\lstinline!ASCEND\_OPP\_PATH!};
\item
  目标芯片:\passthrough{\lstinline!soc\_version=Ascend310B!};优先使用
  FP16 与硬件友好布局(如NC1HWC0)。
\end{itemize}

\begin{enumerate}
\def\labelenumi{\arabic{enumi}.}
\setcounter{enumi}{1}
\tightlist
\item
  定义算子信息(op info/proto)
\end{enumerate}

\begin{itemize}
\tightlist
\item
  声明 \passthrough{\lstinline!op\_type!}、inputs/outputs
  名称与数量、可支持的 \passthrough{\lstinline!dtype\_format!}
  组合、属性与默认值;
\item
  提供形状推断规则(静态或依据属性/输入维度计算)。
\end{itemize}

\begin{enumerate}
\def\labelenumi{\arabic{enumi}.}
\setcounter{enumi}{2}
\tightlist
\item
  编写算子实现(TE/TBE/TIK)
\end{enumerate}

\begin{itemize}
\tightlist
\item
  计算表达(示例:Add+ReLU 融合伪代码):
\end{itemize}

\begin{lstlisting}
# y = relu(x1 + x2)
import te.lang.cce as tbe
from te import tvm

def add_relu_compute(x1, x2):
        y = tbe.vadd(x1, x2)
        z = tbe.vmaxs(y, tvm.const(0.0, x1.dtype))
        return z
\end{lstlisting}

\begin{itemize}
\tightlist
\item
  调度要点:

  \begin{itemize}
  \tightlist
  \item
    Tile 到 UB 容量可承载的块大小;
  \item
    连续向量访问,减少非对齐;
  \item
    合并搬运,避免频繁小块 DMA;
  \item
    小尺寸路径避免调度开销超过计算开销。
  \end{itemize}
\end{itemize}

\begin{enumerate}
\def\labelenumi{\arabic{enumi}.}
\setcounter{enumi}{3}
\tightlist
\item
  编译与注册
\end{enumerate}

\begin{itemize}
\tightlist
\item
  使用 Toolkit 提供的编译入口生成 kernel 与元数据;
\item
  将实现与描述文件放入 \passthrough{\lstinline!ASCEND\_OPP\_PATH!} 下
  custom 目录(如
  \passthrough{\lstinline!op\_impl/custom/ai\_core/tbe!}、\passthrough{\lstinline!op\_proto/custom!})。
\end{itemize}

\begin{enumerate}
\def\labelenumi{\arabic{enumi}.}
\setcounter{enumi}{4}
\tightlist
\item
  与 ATC 集成
\end{enumerate}

\begin{itemize}
\tightlist
\item
  转换模型时指定 \passthrough{\lstinline!--soc\_version=Ascend310B!};
\item
  确保 OPP 路径可被 ATC 读取,必要时调整
  \passthrough{\lstinline!--op\_select\_implmode!};
\item
  转换日志中应能看到自定义算子被匹配与编译。
\end{itemize}

\begin{enumerate}
\def\labelenumi{\arabic{enumi}.}
\setcounter{enumi}{5}
\tightlist
\item
  运行时部署
\end{enumerate}

\begin{itemize}
\tightlist
\item
  目标环境包含同版本 OPP(含 custom 产物);
\item
  设置环境变量使 Runtime 能定位到自定义实现;
\item
  按常规 ACL 流程加载 OM 并执行推理。
\end{itemize}

\begin{enumerate}
\def\labelenumi{\arabic{enumi}.}
\setcounter{enumi}{6}
\tightlist
\item
  验证与度量
\end{enumerate}

\begin{itemize}
\tightlist
\item
  功能:与 NumPy/ONNX
  参考实现对齐,随机多组张量比较(平均绝对/相对误差、边界样本);
\item
  性能:Warmup≥3 次,采样≥50 次,统计 avg/p95/FPS;
\item
  资源:Profiling 检查 MemCopy 占比、Kernel 占比、Idle;
\item
  兼容:覆盖不同 shape/dtype/layout 组合。
\end{itemize}

\begin{enumerate}
\def\labelenumi{\arabic{enumi}.}
\setcounter{enumi}{7}
\tightlist
\item
  打包与版本化
\end{enumerate}

\begin{itemize}
\tightlist
\item
  输出 \passthrough{\lstinline!op\_contract.yaml!}(契约)与
  \passthrough{\lstinline!benchmark.json!}(性能);
\item
  目录建议:
\end{itemize}

\begin{lstlisting}
op_pkg/<op_type>/<version>/
    ├─ op_proto/custom/
    ├─ op_impl/custom/ai_core/tbe/
    ├─ tests/
    └─ docs/
\end{lstlisting}

\section{常见问题与排查}\label{ux5e38ux89c1ux95eeux9898ux4e0eux6392ux67e5}

\begin{itemize}
\tightlist
\item
  ATC 提示 Unsupported Op:检查 op 描述是否生效、路径与
  \passthrough{\lstinline!soc\_version!} 是否匹配;
\item
  运行时回退(fallback):确认 \passthrough{\lstinline!dtype\_format!}
  覆盖到当前张量组合;
\item
  性能无提升:检查是否出现额外 layout 转换、tile 过小造成 DMA 频繁;
\item
  精度异常:核对归一化/广播规则、溢出与舍入策略,必要时局部切 FP32;
\item
  动态 shape OOM:缩小 tile 或分桶处理,保证 UB 与工作区不溢出。
\end{itemize}

\section{章节小结}\label{ux7ae0ux8282ux5c0fux7ed3}

自定义算子是 310B
场景下实现``功能补齐与性能确定性''的关键手段。遵循``明确契约 → 正确调度
→ 可观测验证 →
规范打包''的路径,选择计算/访存比例合适、出现频繁的目标起步,先易后难、以基线与回归保障质量与收益的可持续。

\section{实践任务}\label{ux5b9eux8df5ux4efbux52a1}

\begin{enumerate}
\def\labelenumi{\arabic{enumi}.}
\tightlist
\item
  选择你项目中的一个复合算子(例如归一化+阈值),写出算子契约草案(IO/attr/dtype\_format/边界)。
\item
  基于 TE 写出该算子的计算表达伪代码,并说明预期的 tile 与向量化策略。
\item
  在开发环境完成编译注册,将产物放入 OPP custom 目录并用一个最小模型验证
  ATC 识别。
\item
  设计功能与性能验证脚本:随机张量对齐、Warmup/采样策略、输出 avg/p95
  与资源占比。
\item
  生成 \passthrough{\lstinline!op\_contract.yaml!} 与
  \passthrough{\lstinline!benchmark.json!},并归档到
  \passthrough{\lstinline!op\_pkg/<op\_type>/<version>/!}。
\end{enumerate}
