\section{章节总览}\label{ux7ae0ux8282ux603bux89c8}

本章建立从需求澄清→指标体系→评测集→迭代节奏→资产沉淀→交付与回归的一套闭环方法论,让技术决策基于指标与风险敞口,而非经验臆测。核心理念:可量化、可比较、可复用、可追溯。

\section{需求澄清 Canvas}\label{ux9700ux6c42ux6f84ux6e05-canvas}

\begin{longtable}[]{@{}
  >{\raggedright\arraybackslash}p{(\linewidth - 6\tabcolsep) * \real{0.2000}}
  >{\raggedright\arraybackslash}p{(\linewidth - 6\tabcolsep) * \real{0.2000}}
  >{\raggedright\arraybackslash}p{(\linewidth - 6\tabcolsep) * \real{0.4000}}
  >{\raggedright\arraybackslash}p{(\linewidth - 6\tabcolsep) * \real{0.2000}}@{}}
\toprule\noalign{}
\begin{minipage}[b]{\linewidth}\raggedright
维度
\end{minipage} & \begin{minipage}[b]{\linewidth}\raggedright
要素
\end{minipage} & \begin{minipage}[b]{\linewidth}\raggedright
问题提示
\end{minipage} & \begin{minipage}[b]{\linewidth}\raggedright
示例
\end{minipage} \\
\midrule\noalign{}
\endhead
\bottomrule\noalign{}
\endlastfoot
场景 & 输入源/运行环境 & 摄像头?批处理? & 室内 1080p30 低光 \\
目标 & 功能/业务价值 & 用户希望看到什么结果? & 实时检测 + 分析 \\
指标 & Latency/FPS/精度 & 哪些分位数重要? & \textless80ms / ≥25FPS /
mAP≥0.6 \\
约束 & 能耗/内存/带宽 & 上限是多少? & 功耗≤15W 内存≤3GB \\
风险 & 数据/硬件/算法 & 失败模式有哪些? & 低光/遮挡/抖动 \\
合规 & 隐私/许可 & 是否需要脱敏? & 仅上传事件元数据 \\
成本 & 硬件/云 & ROI 衡量? & 10 台板卡预算 \\
\end{longtable}

输出:\passthrough{\lstinline!requirement.yaml!}(版本化),后续所有评审基于此文档。

\section{指标分层与优先级}\label{ux6307ux6807ux5206ux5c42ux4e0eux4f18ux5148ux7ea7}

\begin{longtable}[]{@{}lllll@{}}
\toprule\noalign{}
层级 & 类别 & 指标 & 说明 & 失败后果 \\
\midrule\noalign{}
\endhead
\bottomrule\noalign{}
\endlastfoot
SLO A & 体验 & p95 延迟 & 端到端 & 体验差/丢帧 \\
SLO A & 性能 & FPS & 稳态吞吐 & 处理拥堵 \\
SLO B & 质量 & mAP/F1/Top1 & 任务正确性 & 无法满足业务 \\
SLO B & 稳定 & Crash/小时 & 可靠性 & 运维成本高 \\
SLO C & 资源 & 内存峰值/功耗 & 成本约束 & 设备异常/降频 \\
SLO C & 带宽 & 上行 kbps & 成本/合规 & 费用/拥塞 \\
\end{longtable}

优先级:先保障 A(体验+功能可用),再稳定 B(质量/稳定),最后优化
C(资源效率)。

\section{Baseline
策略与控制变量法}\label{baseline-ux7b56ux7565ux4e0eux63a7ux5236ux53d8ux91cfux6cd5}

Baseline 目标:建立 ``最小改动可运行'' 标尺。原则:

\begin{enumerate}
\def\labelenumi{\arabic{enumi}.}
\tightlist
\item
  不提前做微优化;
\item
  记录所有关键参数:模型版本、输入尺寸、预处理策略、硬件温度范围;
\item
  一次仅改变单个变量(batch、精度、线程数)。
  基线存档:\passthrough{\lstinline!baseline/<date>-<commit>/metrics.json!};对比脚本生成差异报告。
\end{enumerate}

\section{评测集设计原则}\label{ux8bc4ux6d4bux96c6ux8bbeux8ba1ux539fux5219}

\begin{longtable}[]{@{}ll@{}}
\toprule\noalign{}
原则 & 内容 \\
\midrule\noalign{}
\endhead
\bottomrule\noalign{}
\endlastfoot
代表性 & 涵盖主流场景/光照/角度 \\
覆盖边界 & 极端尺寸、模糊、遮挡 \\
可再现 & 文件命名规范 + 固定清单 \\
可扩展 & 新增样本不破坏旧索引 \\
标注一致 & 标注工具/规范/审校流程 \\
\end{longtable}

目录示例:

\begin{lstlisting}
dataset_eval/
  images/
    day/*.jpg
    night/*.jpg
    occlusion/*.jpg
  annotations/
    instances_train.json
    instances_val.json
  meta/
    README.md
    version.txt
\end{lstlisting}

提供 Hash 列表,防止样本被替换而影响回归可信度。

\section{迭代计划与看板}\label{ux8fedux4ee3ux8ba1ux5212ux4e0eux770bux677f}

四阶段:

\begin{longtable}[]{@{}llll@{}}
\toprule\noalign{}
Sprint & 目标 & 核心产出 & 风险控制 \\
\midrule\noalign{}
\endhead
\bottomrule\noalign{}
\endlastfoot
0 & 环境/基线 & baseline metrics & 依赖清单齐全 \\
1 & 精度与功能稳固 & 精度报告 & 数据问题快速反馈 \\
2 & 性能与稳定 & 性能对比表/监控上线 & Watchdog 验证 \\
3 & 工程交付包装 & Release Notes/脚本 & 灰度计划制定 \\
\end{longtable}

看板列:Backlog → Doing → Review → Bench → Done;性能/精度任务需进入
Bench 列执行对比脚本通过后才可 Done。

\section{资产沉淀文档体系}\label{ux8d44ux4ea7ux6c89ux6dc0ux6587ux6863ux4f53ux7cfb}

\begin{longtable}[]{@{}lll@{}}
\toprule\noalign{}
文档 & 内容 & 更新频率 \\
\midrule\noalign{}
\endhead
\bottomrule\noalign{}
\endlastfoot
README & 快速启动 & 版本变化时 \\
ARCHITECTURE & 架构图/模块说明 & 结构调整 \\
MODEL\_CARD & 模型来源/许可/精度/限制 & 模型更新 \\
EVAL\_REPORT & 数据与评测方法/指标 & 每次发布 \\
PERF\_REPORT & 基线/优化对比 & 优化后 \\
CHANGELOG & 可见版本差异 & 每次版本 \\
RISK\_LOG & 已知风险列表 & 动态 \\
\end{longtable}

MODEL\_CARD 需包含:数据来源、训练超参摘要、输入契约、已知局限、许可(如
Apache-2.0)、安全与偏见说明(若涉及识别敏感属性声明避免用途)。

\section{交付目录与不可变产物}\label{ux4ea4ux4ed8ux76eeux5f55ux4e0eux4e0dux53efux53d8ux4ea7ux7269}

\begin{lstlisting}
release/
  v1.0/
    manifest.json     # 产物 hash / 版本矩阵
    models/
      detect.om
      classify.om
      signature.json
    scripts/
      run.ps1
      run.sh
      watchdog.sh
    configs/
      default.yaml
    docs/
      model_card_detect.md
      model_card_classify.md
      QUICKSTART.md
    reports/
      perf.json
      accuracy.json
\end{lstlisting}

manifest.json
字段:\passthrough{\lstinline!\{version, commit, build\_time, model\_hashes, dependencies\}!}。

\section{上线前综合
Checklist}\label{ux4e0aux7ebfux524dux7efcux5408-checklist}

\begin{longtable}[]{@{}lll@{}}
\toprule\noalign{}
类别 & 检查项 & 通过标准 \\
\midrule\noalign{}
\endhead
\bottomrule\noalign{}
\endlastfoot
功能 & 核心用例 100\% & 自动化用例通过 \\
性能 & p95 \textless{} 目标 +5\% & 连续 30min 稳定 \\
精度 & mAP/Top1 回归差 \textless{} 阈值 & 与基线对比 \\
资源 & 内存峰值 \textless{} 75\% & 1h 稳态无泄漏 \\
稳定 & Crash=0, 重启=0 & 守护日志清洁 \\
安全 & 日志无敏感泄露 & 关键字段脱敏 \\
配置 & 签名校验一致 & Hash 匹配 \\
回滚 & 验证上一版本可用 & 切换 \textless{} 30s \\
\end{longtable}

\section{验收、回归与漂移监测}\label{ux9a8cux6536ux56deux5f52ux4e0eux6f02ux79fbux76d1ux6d4b}

交付后 7 天加密监控:记录时延、精度漂移(采样对比模型输出变化)。
漂移检测:相同输入集合(Shadow Set)每天抽样跑一次 → 统计 logits KL
散度/TopK 变化率,高于阈值(如 KL \textgreater{}
0.05)触发报警(潜在数据分布变化或模型文件损坏)。
回归集版本化:\passthrough{\lstinline!eval\_set\_vX!};若需替换样本 →
新增版本,不覆盖旧数据。

\section{风险管理与决策日志}\label{ux98ceux9669ux7ba1ux7406ux4e0eux51b3ux7b56ux65e5ux5fd7}

风险登记表:\passthrough{\lstinline!risk\_log.md!}
每条包含:ID、描述、影响、概率、缓解、当前状态。决策日志(Decision
Record,
ADR):记录架构/模型/精度策略选择及备选方案放弃理由,以便新成员快速建立上下文。

\section{章节小结}\label{ux7ae0ux8282ux5c0fux7ed3}

方法论的核心不是流程文档堆砌,而是``指标驱动 + 资产沉淀 +
可回滚''三支柱。通过契约化需求、标准化
Baseline、规范化评测与回归体系,使团队协作更高效、风险暴露更透明、交付结果更可信。

\section{实践任务}\label{ux5b9eux8df5ux4efbux52a1}

\begin{enumerate}
\def\labelenumi{\arabic{enumi}.}
\tightlist
\item
  输出 \passthrough{\lstinline!requirement.yaml!}(含指标与约束)。
\item
  构建 30 张代表图像的 mini 评测集并附 Hash 列表。
\item
  生成 baseline \passthrough{\lstinline!metrics.json!} 与后一次优化对比
  diff 报告。
\item
  制作一个 MODEL\_CARD 模板并填写一个模型示例。
\item
  编写上线 Checklist 并模拟一项未通过情形与处置方案。
\end{enumerate}
