\section{章节总览}\label{ux7ae0ux8282ux603bux89c8}

本章以``统一流程 → 四类典型任务(分类/检测/OCR/NLP)→ 多模型 Pipeline →
工程化目录与脚本 → 性能基线采集 →
问题诊断''逻辑展开,强调``可复现、可量化、可演进''的部署范式。所有示例策略均可推广到后续复杂场景(多输入、多分辨率、流式/批式混合)。

\section{统一部署工作流与契约化}\label{ux7edfux4e00ux90e8ux7f72ux5de5ux4f5cux6d41ux4e0eux5951ux7ea6ux5316}

标准六步:模型选择 → 框架导出 ONNX → ATC 转换(参数冻结)→
推理引擎封装(I/O 契约)→ 运行形态编排 → 验证(精度 + 性能)。
核心产物: \textbar{} 文件 \textbar{} 作用 \textbar{} \textbar{} ----
\textbar{} ---- \textbar{} \textbar{} export.py \textbar{} 导出 \& 简化
ONNX \textbar{} \textbar{} atc.sh \textbar{} 标准化转换命令 \textbar{}
\textbar{} config.yaml \textbar{} 输入/归一化/颜色/阈值 \textbar{}
\textbar{} signature.json \textbar{} 模型输入输出字段与 dtype \textbar{}
\textbar{} metrics.json \textbar{} 性能统计(avg/p95/memory) \textbar{}

输入预处理必须模块化,业务层仅提供原始图像对象;可在 AIPP
中下沉部分(色彩空间、均值/方差),减少 Host 侧拷贝和转换。

\section{图像分类:ResNet /
MobileNet}\label{ux56feux50cfux5206ux7c7bresnet-mobilenet}

\subsection{模型导出}\label{ux6a21ux578bux5bfcux51fa}

PyTorch →
ONNX:\passthrough{\lstinline!torch.onnx.export(model, dummy, opset\_version=13, dynamic\_axes=None)!};确保去掉训练专属层(Dropout,
BN 置 eval)。 \#\#\# 预处理一致性 1. Resize: 保持短边 256 → CenterCrop
224。 2. Normalize: mean/std 与训练保持一致。 3. Layout:
NCHW;若原始图像为 HWC(RGB) → 转 BGR/或保持一致并在 config 标记。 \#\#\#
转换要点
\passthrough{\lstinline!--precision\_mode=allow\_fp32\_to\_fp16!};若需
INT8:先做离线标定导出校准表,再加量化参数。 \#\#\# 推理后处理 Softmax →
ArgTopK → LabelMap。为避免数值不稳定:FP16 logits 可先转 FP32 再
softmax。 \#\#\# 性能采集 Warmup 5 次,采集 100 次:记录 avg, p50, p95,
max;统计预处理耗时占比:\passthrough{\lstinline!pre\_ms / total\_ms!},超过
25\% 提示 AIPP 下沉或批处理优化。

\section{目标检测:YOLO /
FasterRCNN}\label{ux76eeux6807ux68c0ux6d4byolo-fasterrcnn}

\subsection{输入尺寸与
Letterbox}\label{ux8f93ux5165ux5c3aux5bf8ux4e0e-letterbox}

Letterbox 使图像等比例缩放 +
填充,保持方形输入。部署需重现训练阶段相同逻辑,否则框坐标偏移。保存
\passthrough{\lstinline!scale!} 与 \passthrough{\lstinline!pad!}
用于反算原始坐标。 \#\#\# 多输出解析 YOLOv5s OM
输出通常包含一个或多个特征拼接张量:\passthrough{\lstinline!(num\_boxes, attributes)!};后处理:过滤
conf \textgreater{} 阈值 → 按类合并 → NMS。 \#\#\# NMS 实现决策
\textbar{} 方案 \textbar{} 优点 \textbar{} 缺点 \textbar{} \textbar{}
---- \textbar{} ---- \textbar{} ---- \textbar{} \textbar{} CPU Python
\textbar{} 简单 \textbar{} 高开销,多框场景慢 \textbar{} \textbar{} CPU
C++ SIMD \textbar{} 中等复杂 \textbar{} 仍需 D2H 拷贝 \textbar{}
\textbar{} Device Kernel \textbar{} 减少拷贝 \textbar{} 实现复杂
\textbar{} 先评估 D2H + CPU NMS 占比,\textgreater15\% 再考虑下沉。
\#\#\# 动态尺度支持 转换阶段可生成多尺度 OM 或使用动态
shape;推荐:统计输入分辨率 → 选择 3 桶(640/704/768)提升命中率。

\section{OCR:文本检测 + 识别
Pipeline}\label{ocrux6587ux672cux68c0ux6d4b-ux8bc6ux522b-pipeline}

\subsection{结构}\label{ux7ed3ux6784}

检测模型(DB) → 文本框多边形 → 透视裁剪 → 识别模型(CRNN / SVTR)。
\#\#\# 难点与策略 \textbar{} 环节 \textbar{} 风险 \textbar{} 对策
\textbar{} \textbar{} ---- \textbar{} ---- \textbar{} ---- \textbar{}
\textbar{} 多边形裁剪 \textbar{} 仿射失真 \textbar{} 统一仿射矩阵 +
padding \textbar{} \textbar{} 长短文本差异 \textbar{} 序列长度不均
\textbar{} 动态 Batch 分组(长度分桶) \textbar{} \textbar{} 识别延迟
\textbar{} 串行处理 \textbar{} 检测与上一批识别并行 \textbar{}
\textbar{} 字典映射 \textbar{} 乱码/对齐 \textbar{} 固定 vocab + 版本号
\textbar{} \#\#\# CTC 解码 贪心:移除重复与 blank;大规模需 Beam
Search(权衡性能)。

\section{NLP:BERT 推理优化}\label{nlpbert-ux63a8ux7406ux4f18ux5316}

\subsection{序列长度策略}\label{ux5e8fux5217ux957fux5ea6ux7b56ux7565}

\begin{enumerate}
\def\labelenumi{\arabic{enumi}.}
\tightlist
\item
  静态最大长度(简单,浪费算力)。
\item
  Bucketing:按输入长短分类(32/64/128/256),多 OM。
\item
  动态 shape:需评估内存分配抖动;提前预热各常见长度。 \#\#\# FP16
  注意点 LayerNorm/Softmax 数值范围敏感;若发现精度下降:保持部分算子
  FP32(通过混合精度策略或模型修改)。 \#\#\# 性能指标
  tokens/s、avg\_latency\_ms(batch=1 与
  batch\textgreater1)、内存占用;观察自注意力占比,必要时进行剪枝(去除冗余
  head)或蒸馏。
\end{enumerate}

\section{多模型 Pipeline
串联}\label{ux591aux6a21ux578b-pipeline-ux4e32ux8054}

案例:检测 → 裁剪 → 分类。 \textbar{} Stage \textbar{} 输入/输出
\textbar{} 并行策略 \textbar{} 指标采集 \textbar{} \textbar{} -----
\textbar{} -------- \textbar{} -------- \textbar{} -------- \textbar{}
\textbar{} Detector \textbar{} 原始帧 → 框 \textbar{} 批处理+单模型
\textbar{} 时延/框数 \textbar{} \textbar{} Cropper \textbar{} 帧+框 →
Patch 列表 \textbar{} 多线程 CPU \textbar{} 单 Patch 平均耗时 \textbar{}
\textbar{} Classifier \textbar{} Patch → TopK 类别 \textbar{} 合批
(N≤32) \textbar{} FPS/准确率 \textbar{} \#\#\# 优化要点 1. Buffer
池:重用图像与 Patch 内存,避免频繁 malloc。 2. 批量裁剪:收集一定数量
Patch 再统一预处理。 3. 超时控制:某帧超过阈值后续结果丢弃,保持实时性。
4. 滑窗统计:最近 60s FPS、平均队列深度。

\section{工程目录与脚本标准}\label{ux5de5ux7a0bux76eeux5f55ux4e0eux811aux672cux6807ux51c6}

\begin{lstlisting}
deploy/
  classify/
    export.py
    atc.sh
    config.yaml
  detect/
    export.py
    atc.sh
  ocr/
    export_det.py
    export_rec.py
    atc_det.sh
    atc_rec.sh
runtime/
  core/acl_session.cpp
  preprocess/
  postprocess/
  pipelines/
tests/
  data/
  benchmark/
docs/
  model_cards/
\end{lstlisting}

版本归档要求: \textbar{} 产物 \textbar{} 检查点 \textbar{} \textbar{}
---- \textbar{} ------- \textbar{} \textbar{} *.om \textbar{} 与 atc.log
hash 对应 \textbar{} \textbar{} signature.json \textbar{}
与运行时动态查询一致 \textbar{} \textbar{} metrics.json \textbar{}
包含时间戳/commit\_sha \textbar{} \textbar{} model\_card.md \textbar{}
模型来源/License/精度 \textbar{}

\section{性能基线方法与统计置信}\label{ux6027ux80fdux57faux7ebfux65b9ux6cd5ux4e0eux7edfux8ba1ux7f6eux4fe1}

推荐: 1. Warmup 5\textasciitilde10 次; 2. 收集 ≥200 次稳定样本; 3.
计算 avg, p50, p95, p99; 4.
计算置信区间:\passthrough{\lstinline!mean ± 1.96 * (std/sqrt(n))!}; 5.
记录环境:芯片序列号/温度区间/电源模式/版本矩阵。 差异判定:新版本 avg
降低 \textgreater5\% 或 p95 上升 \textgreater8\% 触发报警分析。

\section{常见问题诊断深度版}\label{ux5e38ux89c1ux95eeux9898ux8bcaux65adux6df1ux5ea6ux7248}

\begin{longtable}[]{@{}llll@{}}
\toprule\noalign{}
问题 & 表现 & 诊断步骤 & 修复 \\
\midrule\noalign{}
\endhead
\bottomrule\noalign{}
\endlastfoot
输出全 0 & logits 恒定 & Dump 中间 tensor & 校验预处理/权重损坏 \\
检测框偏移 & 坐标不准 & 可视化缩放/Pad 参数 & 修正 letterbox 逆变换 \\
OCR 乱码 & 字符错位 & 对比 index→char 映射 & 统一 vocab \& 排序 \\
BERT 性能差 & tokens/s 低 & 分析长度分布 & 分桶/裁剪长度 \\
Pipeline 堵塞 & 帧延迟增长 & 监控队列深度 & 降帧/扩线程池 \\
内存持续上涨 & long run OOM & 内存快照/工具 & 释放缓存/池化 \\
\end{longtable}

\section{章节小结}\label{ux7ae0ux8282ux5c0fux7ed3}

本章提供四类典型任务部署详解,并抽象了跨任务可复用的脚手架与性能度量方法。重点在于``输入契约统一''、``阶段解耦''、``可观察性内建''。掌握后可进入性能与算子优化专题。

\section{实践任务}\label{ux5b9eux8df5ux4efbux52a1}

\begin{enumerate}
\def\labelenumi{\arabic{enumi}.}
\tightlist
\item
  部署 ResNet50:输出 Top5 及概率、提交 metrics.json。
\item
  部署 YOLOv5s:5 张测试图片生成可视化结果(描述框坐标与类别统计)。
\item
  构建 OCR 双模型流水线:统计单帧平均文本块数 + 平均识别耗时。
\item
  BERT:对 3 组长度(32/64/128) 测 tokens/s 与时延差异,生成对比表。
\item
  Pipeline 检测→分类:实现批裁剪 + Buffer
  池,比较优化前后平均时延下降百分比。
\end{enumerate}
