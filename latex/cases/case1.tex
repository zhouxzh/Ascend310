\section{\#
案例1:智能人脸识别打卡机}\label{ux6848ux4f8b1ux667aux80fdux4ebaux8138ux8bc6ux522bux6253ux5361ux673a}

\section{项目简介}\label{ux9879ux76eeux7b80ux4ecb}

本项目旨在利用昇腾310B的强大AI算力,构建一个功能完整、响应迅速的智能人脸识别打卡系统。系统通过USB摄像头实时捕捉视频流,检测画面中的人脸,并与预先注册的员工/学生人脸数据库进行比对,完成身份验证和自动记录考勤。

该项目不仅是一个功能性的应用,更是一个端到端的AI实践案例,涵盖了从硬件选型、软件环境搭建、数据准备、模型训练与优化,到最终在边缘设备上部署的全过程。

\section{内容大纲}\label{ux5185ux5bb9ux5927ux7eb2}

\subsection{硬件准备}\label{ux786cux4ef6ux51c6ux5907}

\begin{itemize}
\tightlist
\item
  \textbf{核心计算单元}: 昇腾310B开发者套件
\item
  \textbf{图像采集}: USB摄像头 (推荐罗技C920或同等规格)
\item
  \textbf{显示设备 (可选)}: HDMI显示器,用于实时预览或UI展示
\item
  \textbf{外设}: 键盘、鼠标
\item
  \textbf{电源}: 为昇腾310B开发板提供稳定供电
\item
  \textbf{连接线}: HDMI线, USB-C数据线等
\end{itemize}

\emph{附:硬件连接示意图} \textgreater{}
(此处可插入一张图片,清晰展示所有硬件的连接方式)

\subsection{软件环境}\label{ux8f6fux4ef6ux73afux5883}

\begin{itemize}
\tightlist
\item
  \textbf{操作系统}: Ubuntu 20.04 或 openEuler
\item
  \textbf{CANN版本}: 7.0或更高
\item
  \textbf{Python版本}: 3.8.x
\item
  \textbf{主要依赖库}:

  \begin{itemize}
  \tightlist
  \item
    \passthrough{\lstinline!opencv-python!}: 用于图像和视频处理
  \item
    \passthrough{\lstinline!numpy!}: 用于数值计算
  \item
    \passthrough{\lstinline!scikit-learn!}: 用于评估模型性能
  \item
    \passthrough{\lstinline!onnxruntime!}: 用于运行ONNX模型
  \item
    \passthrough{\lstinline!PyQt5!} (可选): 用于构建图形用户界面
  \end{itemize}
\end{itemize}

\emph{附:一键安装环境脚本 (\passthrough{\lstinline!install\_env.sh!})}

\begin{lstlisting}[language=bash]
# 示例脚本
sudo apt update
sudo apt install -y python3-pip python3-opencv
pip3 install numpy scikit-learn onnxruntime
# ... 其他依赖安装命令
\end{lstlisting}

\subsection{数据集准备}\label{ux6570ux636eux96c6ux51c6ux5907}

\begin{itemize}
\item
  \textbf{数据集来源}:

  \begin{enumerate}
  \def\labelenumi{\arabic{enumi}.}
  \tightlist
  \item
    \textbf{公开数据集}: 如LFW (Labeled Faces in the
    Wild)、CASIA-WebFace等。
  \item
    \textbf{自建数据集}:
    推荐!使用摄像头为每位用户(员工/学生)拍摄多张、多角度、不同光照和表情的人脸照片。
  \end{enumerate}
\item
  \textbf{数据组织结构}:

\begin{lstlisting}
datasets/
├── zhang_san/
│   ├── 001.jpg
│   ├── 002.jpg
│   └── ...
├── li_si/
│   ├── 001.jpg
│   ├── 002.jpg
│   └── ...
└── ...
\end{lstlisting}
\item
  \textbf{预处理脚本 (\passthrough{\lstinline!preprocess.py!})}:

  \begin{itemize}
  \tightlist
  \item
    人脸检测与对齐
  \item
    图像增强(旋转、裁剪、调整亮度等)
  \item
    划分训练集和验证集
  \end{itemize}
\end{itemize}

\subsection{模型训练}\label{ux6a21ux578bux8badux7ec3}

\begin{itemize}
\tightlist
\item
  \textbf{模型选择}:

  \begin{itemize}
  \tightlist
  \item
    \textbf{人脸检测}: MTCNN 或 RetinaFace
  \item
    \textbf{人脸识别}: ArcFace, CosFace, 或 MobileFaceNet
    (推荐,因其轻量高效)
  \end{itemize}
\item
  \textbf{训练流程}:

  \begin{enumerate}
  \def\labelenumi{\arabic{enumi}.}
  \tightlist
  \item
    使用预处理好的数据集进行模型训练。
  \item
    调整超参数(学习率、批大小等)以获得最佳性能。
  \item
    在验证集上评估模型准确率、召回率等指标。
  \end{enumerate}
\item
  \textbf{模型导出}:
  将训练好的PyTorch或TensorFlow模型转换为昇腾亲和的ONNX格式。
\end{itemize}

\subsection{模型部署}\label{ux6a21ux578bux90e8ux7f72}

\begin{itemize}
\item
  \textbf{模型转换}: 使用ATC (Ascend Tensor Compiler)
  工具将ONNX模型转换为昇腾310B支持的\passthrough{\lstinline!.om!}离线模型。

\begin{lstlisting}[language=bash]
atc --model=./face_recognition.onnx --framework=5 --output=./face_recognition --input_format=NCHW --input_shape="data:1,3,112,112" --soc_version=Ascend310B1
\end{lstlisting}
\item
  \textbf{部署代码 (\passthrough{\lstinline!main.py!})}:

  \begin{enumerate}
  \def\labelenumi{\arabic{enumi}.}
  \tightlist
  \item
    初始化CANN和ACL (Ascend Computing Language) 资源。
  \item
    加载\passthrough{\lstinline!.om!}离线模型。
  \item
    循环读取摄像头帧。
  \item
    对每一帧进行人脸检测和识别推理。
  \item
    将识别结果与数据库比对,输出姓名。
  \item
    在画面上绘制矩形框和姓名,并记录打卡时间。
  \end{enumerate}
\end{itemize}

\subsection{3D打印结构件}\label{dux6253ux5370ux7ed3ux6784ux4ef6}

为了方便地将摄像头固定在合适的位置,我们设计了专用的摄像头支架和设备保护外壳。
- \textbf{文件列表}: - \passthrough{\lstinline!camera\_holder.stl!}:
摄像头支架模型文件 - \passthrough{\lstinline!device\_case.stl!}:
设备外壳模型文件 - \textbf{打印建议}: - \textbf{材料}: PLA 或 PETG -
\textbf{层高}: 0.2mm - \textbf{填充率}: 20\%

\subsection{用户手册}\label{ux7528ux6237ux624bux518c}

\begin{enumerate}
\def\labelenumi{\arabic{enumi}.}
\tightlist
\item
  \textbf{硬件组装}:
  参照\passthrough{\lstinline!2.1!}节的连接图连接好所有硬件。
\item
  \textbf{环境配置}:
  运行\passthrough{\lstinline!install\_env.sh!}脚本安装所有软件依赖。
\item
  \textbf{人脸注册}:
  运行\passthrough{\lstinline!register\_face.py!}脚本,根据提示输入姓名,并拍摄多张人脸照片进行注册。
\item
  \textbf{启动系统}:
  运行\passthrough{\lstinline!main.py!}启动人脸识别打卡程序。
\item
  \textbf{查看记录}:
  打卡记录将保存在\passthrough{\lstinline!attendance.csv!}文件中。
\end{enumerate}

\section{源代码}\label{ux6e90ux4ee3ux7801}

\begin{quote}
(此处未来可替换为GitHub仓库链接或详细的文件树)
\end{quote}

\section{效果演示}\label{ux6548ux679cux6f14ux793a}

\begin{quote}
(此处可插入系统运行时的截图或GIF动图,例如摄像头实时识别人脸的画面)
\end{quote}
