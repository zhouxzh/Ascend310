\chapter{案例6:智能小车}\label{ux6848ux4f8b6ux667aux80fdux5c0fux8f66}

\section{1. 项目简介}\label{ux9879ux76eeux7b80ux4ecb}

本项目基于昇腾310B平台,构建一个具备自主导航、障碍物避让、自动循迹等功能的智能小车。该小车集成了计算机视觉、路径规划、运动控制等多项AI技术,能够在复杂环境中自主行驶,为无人驾驶、服务机器人、智能巡检等应用领域提供技术验证平台。

相比传统的遥控小车,智能小车具备环境感知、决策规划和自主执行的能力,代表了移动机器人技术的发展方向。本项目将详细介绍从硬件搭建、软件开发到AI算法部署的完整实现过程。

\section{2. 内容大纲}\label{ux5185ux5bb9ux5927ux7eb2}

\subsection{2.1. 硬件准备}\label{ux786cux4ef6ux51c6ux5907}

\begin{itemize}
\tightlist
\item
  \textbf{核心计算单元}: 昇腾310B开发者套件
\item
  \textbf{底盘系统}:

  \begin{itemize}
  \tightlist
  \item
    \textbf{驱动方式}: 四轮差速驱动
  \item
    \textbf{电机}: 直流减速电机 × 4 (12V, 100RPM)
  \item
    \textbf{编码器}: 增量式光电编码器 × 4
  \item
    \textbf{轮胎}: 全向轮或麦克纳姆轮
  \end{itemize}
\item
  \textbf{传感器系统}:

  \begin{itemize}
  \tightlist
  \item
    \textbf{视觉传感器}: USB摄像头 (1080p 30fps)
  \item
    \textbf{激光雷达}: RPLiDAR A1 或 A2 (可选)
  \item
    \textbf{超声波传感器}: HC-SR04 × 6 (前后左右)
  \item
    \textbf{IMU}: MPU6050 (姿态检测)
  \item
    \textbf{GPS模块}: NEO-8M (户外定位)
  \end{itemize}
\item
  \textbf{控制系统}:

  \begin{itemize}
  \tightlist
  \item
    \textbf{主控板}: 树莓派4B 或 专用控制板
  \item
    \textbf{电机驱动}: L298N驱动模块 × 2
  \item
    \textbf{舵机}: SG90 (摄像头云台)
  \end{itemize}
\item
  \textbf{电源系统}:

  \begin{itemize}
  \tightlist
  \item
    \textbf{主电池}: 12V 锂电池组 (5000mAh)
  \item
    \textbf{辅助电源}: 5V/3.3V稳压模块
  \item
    \textbf{电源管理}: 充电保护电路
  \end{itemize}
\end{itemize}

\emph{智能小车系统架构}

\begin{lstlisting}
     摄像头 + 云台
          │
    ┌─────▼─────┐
    │  昇腾310B │ ← AI决策中心
    └─────┬─────┘
          │
    ┌─────▼─────┐
    │ 主控制器  │ ← 运动控制
    └─────┬─────┘
          │
  ┌───────┼───────┐
  │       │       │
 传感器   电机    电源系统
 阵列    驱动
\end{lstlisting}

\subsection{2.2. 软件环境}\label{ux8f6fux4ef6ux73afux5883}

\begin{itemize}
\tightlist
\item
  \textbf{操作系统}: Ubuntu 20.04 LTS
\item
  \textbf{CANN版本}: 7.0.RC1
\item
  \textbf{Python版本}: 3.8.10
\item
  \textbf{机器人框架}:

  \begin{itemize}
  \tightlist
  \item
    \passthrough{\lstinline!ROS2 Foxy!}: 机器人操作系统
  \item
    \passthrough{\lstinline!rospy!}: Python ROS接口
  \item
    \passthrough{\lstinline!tf2!}: 坐标变换库
  \end{itemize}
\item
  \textbf{计算机视觉}:

  \begin{itemize}
  \tightlist
  \item
    \passthrough{\lstinline!opencv-python!}: 图像处理
  \item
    \passthrough{\lstinline!ultralytics!}: YOLOv8目标检测
  \item
    \passthrough{\lstinline!apriltag!}: AprilTag标签识别
  \end{itemize}
\item
  \textbf{路径规划}:

  \begin{itemize}
  \tightlist
  \item
    \passthrough{\lstinline!scipy!}: 科学计算
  \item
    \passthrough{\lstinline!numpy!}: 数值计算
  \item
    \passthrough{\lstinline!matplotlib!}: 路径可视化
  \end{itemize}
\item
  \textbf{硬件控制}:

  \begin{itemize}
  \tightlist
  \item
    \passthrough{\lstinline!RPi.GPIO!}: GPIO控制
  \item
    \passthrough{\lstinline!pyserial!}: 串口通信
  \item
    \passthrough{\lstinline!smbus!}: I2C通信
  \end{itemize}
\end{itemize}

\emph{环境配置脚本 (\passthrough{\lstinline!setup\_smartcar.sh!})}

\begin{lstlisting}[language=bash]
#!/bin/bash
# ROS2安装
sudo apt update
sudo apt install curl gnupg2 lsb-release
curl -s https://raw.githubusercontent.com/ros/rosdistro/master/ros.asc | sudo apt-key add -
sudo sh -c 'echo "deb [arch=$(dpkg --print-architecture)] http://packages.ros.org/ros2/ubuntu $(lsb_release -cs) main" > /etc/apt/sources.list.d/ros2-latest.list'
sudo apt update
sudo apt install ros-foxy-desktop

# Python依赖安装
pip3 install opencv-python ultralytics scipy numpy matplotlib
pip3 install RPi.GPIO pyserial smbus rospkg

echo "智能小车环境配置完成!"
\end{lstlisting}

\subsection{2.3.
计算机视觉系统}\label{ux8ba1ux7b97ux673aux89c6ux89c9ux7cfbux7edf}

\begin{itemize}
\item
  \textbf{目标检测与识别}: ```python \# YOLO目标检测类 class
  ObjectDetector: def \textbf{init}(self, model\_path): self.model =
  YOLO(model\_path) self.target\_classes = {[}`person', `car',
  `traffic\_light', `stop\_sign'{]}

\begin{lstlisting}
  def detect_objects(self, image):
      results = self.model(image)
      detections = []

      for result in results:
          for box in result.boxes:
              if box.cls in self.target_classes:
                  detections.append({
                      'class': self.target_classes[box.cls],
                      'confidence': box.conf,
                      'bbox': box.xyxy,
                      'distance': self.estimate_distance(box)
                  })

      return detections
\end{lstlisting}

  ```
\item
  \textbf{车道线检测}:

  \begin{itemize}
  \tightlist
  \item
    \textbf{图像预处理}: 灰度化、高斯滤波、边缘检测
  \item
    \textbf{ROI提取}: 感兴趣区域设定
  \item
    \textbf{Hough变换}: 直线检测
  \item
    \textbf{拟合优化}: 车道线拟合和平滑
  \end{itemize}
\item
  \textbf{深度估计}:

  \begin{itemize}
  \tightlist
  \item
    \textbf{单目深度估计}: 基于物体大小的距离估算
  \item
    \textbf{双目视觉} (可选): 立体视觉深度计算
  \item
    \textbf{传感器融合}: 结合超声波和视觉信息
  \end{itemize}
\end{itemize}

\subsection{2.4.
路径规划与导航}\label{ux8defux5f84ux89c4ux5212ux4e0eux5bfcux822a}

\begin{itemize}
\item
  \textbf{全局路径规划}: ```python \# A*路径规划算法 class AStarPlanner:
  def \textbf{init}(self, grid\_map): self.grid\_map = grid\_map
  self.open\_set = {[}{]} self.closed\_set = set()

\begin{lstlisting}
  def plan_path(self, start, goal):
      # A*算法实现
      current = start
      path = []

      while current != goal:
          # 搜索最优路径
          current = self.find_best_node()
          path.append(current)

          if self.is_goal_reached(current, goal):
              break

      return self.smooth_path(path)

  def smooth_path(self, path):
      # 路径平滑处理
      return smoothed_path
\end{lstlisting}

  ```
\item
  \textbf{局部路径规划}:

  \begin{itemize}
  \tightlist
  \item
    \textbf{DWA算法}: 动态窗口法避障
  \item
    \textbf{人工势场法}: 基于势场的路径规划
  \item
    \textbf{RRT算法}: 快速随机树路径搜索
  \end{itemize}
\item
  \textbf{SLAM建图} (可选):

  \begin{itemize}
  \tightlist
  \item
    \textbf{激光SLAM}: 基于激光雷达的地图构建
  \item
    \textbf{视觉SLAM}: 基于摄像头的视觉SLAM
  \item
    \textbf{多传感器融合}: 激光雷达 + 视觉 + IMU
  \end{itemize}
\end{itemize}

\subsection{2.5.
运动控制系统}\label{ux8fd0ux52a8ux63a7ux5236ux7cfbux7edf}

\begin{itemize}
\item
  \textbf{底层运动控制}: ```python \# 差速驱动控制器 class
  DifferentialDriveController: def \textbf{init}(self, wheel\_base,
  wheel\_radius): self.wheel\_base = wheel\_base self.wheel\_radius =
  wheel\_radius self.max\_speed = 1.0 \# m/s

\begin{lstlisting}
  def velocity_to_wheel_speeds(self, linear_vel, angular_vel):
      # 将线速度和角速度转换为左右轮速度
      left_speed = linear_vel - angular_vel * self.wheel_base / 2
      right_speed = linear_vel + angular_vel * self.wheel_base / 2

      # 速度限制
      left_speed = self.limit_speed(left_speed)
      right_speed = self.limit_speed(right_speed)

      return left_speed, right_speed

  def execute_motion(self, left_speed, right_speed):
      # 发送速度指令到电机驱动器
      self.set_motor_speeds(left_speed, right_speed)
\end{lstlisting}

  ```
\item
  \textbf{PID控制器}:

  \begin{itemize}
  \tightlist
  \item
    \textbf{位置控制}: PID位置控制器
  \item
    \textbf{速度控制}: PID速度控制器
  \item
    \textbf{姿态控制}: PID姿态稳定控制
  \end{itemize}
\item
  \textbf{轨迹跟踪}:

  \begin{itemize}
  \tightlist
  \item
    \textbf{Pure Pursuit}: 纯跟踪算法
  \item
    \textbf{Stanley控制器}: Stanley横向控制
  \item
    \textbf{MPC控制}: 模型预测控制
  \end{itemize}
\end{itemize}

\subsection{2.6. AI决策系统}\label{aiux51b3ux7b56ux7cfbux7edf}

\begin{itemize}
\item
  \textbf{行为决策树}: ```python \# 智能行为决策系统 class
  BehaviorPlanner: def \textbf{init}(self): self.behaviors = \{
  `follow\_lane': self.follow\_lane\_behavior, `avoid\_obstacle':
  self.avoid\_obstacle\_behavior, `stop\_for\_traffic':
  self.stop\_for\_traffic\_behavior, `explore':
  self.exploration\_behavior \}

\begin{lstlisting}
  def make_decision(self, sensor_data, current_state):
      # 根据传感器数据和当前状态做出决策
      if self.detect_obstacle(sensor_data):
          return 'avoid_obstacle'
      elif self.detect_traffic_sign(sensor_data):
          return 'stop_for_traffic'
      elif self.lane_detected(sensor_data):
          return 'follow_lane'
      else:
          return 'explore'
\end{lstlisting}

  ```
\item
  \textbf{强化学习} (高级功能):

  \begin{itemize}
  \tightlist
  \item
    \textbf{Q-Learning}: 基于值函数的学习
  \item
    \textbf{Deep Q-Network}: 深度Q网络
  \item
    \textbf{Policy Gradient}: 策略梯度方法
  \end{itemize}
\end{itemize}

\subsection{2.7.
模型部署与优化}\label{ux6a21ux578bux90e8ux7f72ux4e0eux4f18ux5316}

\begin{itemize}
\item
  \textbf{模型转换流程}: ```bash \# YOLOv8模型转换 \# 1. PyTorch转ONNX
  yolo export model=yolov8n.pt format=onnx

  \# 2. ONNX转昇腾模型 atc --model=yolov8n.onnx --framework=5
  --output=yolov8n\_car\\
  --input\_format=NCHW --input\_shape=``images:1,3,640,640''\\
  --soc\_version=Ascend310B1 --out\_nodes=``output0:0'' ```
\item
  \textbf{实时推理优化}:

  \begin{itemize}
  \tightlist
  \item
    \textbf{模型量化}: INT8量化减少计算量
  \item
    \textbf{图优化}: 算子融合和内存优化
  \item
    \textbf{并行处理}: 多线程并行推理
  \end{itemize}
\end{itemize}

\subsection{2.8. 安全系统}\label{ux5b89ux5168ux7cfbux7edf}

\begin{itemize}
\tightlist
\item
  \textbf{紧急制动系统}:

  \begin{itemize}
  \tightlist
  \item
    \textbf{超声波触发}: 近距离障碍物检测
  \item
    \textbf{视觉确认}: 摄像头二次确认
  \item
    \textbf{硬件保护}: 硬件级别的紧急停止
  \end{itemize}
\item
  \textbf{故障检测}:

  \begin{itemize}
  \tightlist
  \item
    \textbf{传感器健康监测}: 实时检测传感器状态
  \item
    \textbf{通信故障检测}: 网络和串口通信监控
  \item
    \textbf{电源监控}: 电池电量和电压监测
  \end{itemize}
\end{itemize}

\subsection{2.9. 3D打印结构件}\label{dux6253ux5370ux7ed3ux6784ux4ef6}

\begin{itemize}
\tightlist
\item
  \textbf{底盘框架 (\passthrough{\lstinline!chassis\_frame.stl!})}:

  \begin{itemize}
  \tightlist
  \item
    轻量化设计
  \item
    模块化安装孔
  \item
    线缆走线槽
  \end{itemize}
\item
  \textbf{传感器安装座 (\passthrough{\lstinline!sensor\_mounts.stl!})}:

  \begin{itemize}
  \tightlist
  \item
    摄像头云台支架
  \item
    超声波传感器固定座
  \item
    激光雷达安装底座
  \end{itemize}
\item
  \textbf{保护外壳 (\passthrough{\lstinline!protective\_covers.stl!})}:

  \begin{itemize}
  \tightlist
  \item
    控制板保护罩
  \item
    电池仓外壳
  \item
    防撞缓冲器
  \end{itemize}
\end{itemize}

\emph{3D打印规格}: - \textbf{材料}: PLA (原型) 或 ABS (实用) -
\textbf{层高}: 0.2mm - \textbf{填充率}: 25\% (结构件) / 15\% (外壳)

\subsection{2.10. 用户手册}\label{ux7528ux6237ux624bux518c}

\subsubsection{2.10.1 硬件组装}\label{ux786cux4ef6ux7ec4ux88c5}

\begin{enumerate}
\def\labelenumi{\arabic{enumi}.}
\tightlist
\item
  \textbf{底盘组装}: 安装电机、轮子和编码器
\item
  \textbf{传感器安装}: 固定摄像头、超声波等传感器
\item
  \textbf{电路连接}: 按照接线图连接所有电子模块
\item
  \textbf{系统测试}: 验证各个子系统功能
\end{enumerate}

\subsubsection{2.10.2 软件配置}\label{ux8f6fux4ef6ux914dux7f6e}

\begin{enumerate}
\def\labelenumi{\arabic{enumi}.}
\tightlist
\item
  \textbf{环境安装}: 运行环境配置脚本
\item
  \textbf{ROS配置}: 配置ROS节点和话题
\item
  \textbf{参数标定}: 标定传感器和运动参数
\item
  \textbf{地图构建}: 构建环境地图 (如需要)
\end{enumerate}

\subsubsection{2.10.3 操作指南}\label{ux64cdux4f5cux6307ux5357}

\begin{enumerate}
\def\labelenumi{\arabic{enumi}.}
\tightlist
\item
  \textbf{手动模式}: 遥控器控制小车运动
\item
  \textbf{半自动模式}: 人工指定目标点自动导航
\item
  \textbf{全自动模式}: 完全自主探索和导航
\item
  \textbf{调试模式}: 实时查看传感器数据和算法状态
\end{enumerate}

\subsubsection{2.10.4 维护保养}\label{ux7ef4ux62a4ux4fddux517b}

\begin{enumerate}
\def\labelenumi{\arabic{enumi}.}
\tightlist
\item
  \textbf{定期检查}: 检查轮子、电机和传感器
\item
  \textbf{软件更新}: 更新算法和模型
\item
  \textbf{电池保养}: 正确充放电保护电池
\item
  \textbf{故障排除}: 常见问题解决方案
\end{enumerate}

\section{3. 源代码结构}\label{ux6e90ux4ee3ux7801ux7ed3ux6784}

\begin{lstlisting}
smart_car/
├── src/
│   ├── perception/          # 感知模块
│   │   ├── camera/         # 摄像头处理
│   │   ├── lidar/          # 激光雷达
│   │   └── ultrasonic/     # 超声波传感器
│   ├── planning/           # 规划模块
│   │   ├── global_planner/ # 全局路径规划
│   │   ├── local_planner/  # 局部路径规划
│   │   └── behavior/       # 行为规划
│   ├── control/            # 控制模块
│   │   ├── motion_control/ # 运动控制
│   │   └── safety/         # 安全控制
│   └── utils/              # 工具模块
├── models/
│   ├── detection/          # 目标检测模型
│   ├── segmentation/       # 图像分割模型
│   └── rl_models/          # 强化学习模型
├── configs/
│   ├── sensors.yaml        # 传感器配置
│   ├── motion.yaml         # 运动参数配置
│   └── behavior.yaml       # 行为配置
├── launch/
│   └── smartcar.launch.py  # ROS启动文件
└── hardware/
    ├── 3d_models/          # 3D打印文件
    ├── pcb_design/         # 电路设计
    └── assembly/           # 组装指南
\end{lstlisting}

\section{4. 效果演示}\label{ux6548ux679cux6f14ux793a}

\begin{itemize}
\tightlist
\item
  \textbf{自动循迹}: 沿着地面标线自动行驶
\item
  \textbf{障碍物避让}: 检测并绕过静态和动态障碍物
\item
  \textbf{目标跟踪}: 跟随指定目标人员或物体
\item
  \textbf{自主导航}: 在已知地图中自主导航到目标点
\item
  \textbf{智能巡检}: 按照预设路线进行巡逻检查
\end{itemize}
