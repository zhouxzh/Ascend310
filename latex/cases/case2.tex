\chapter{案例2:边缘端实时目标跟踪}\label{ux6848ux4f8b2ux8fb9ux7f18ux7aefux5b9eux65f6ux76eeux6807ux8ddfux8e2a}

\begin{center}\rule{0.5\linewidth}{0.5pt}\end{center}

\section{项目简介}\label{ux9879ux76eeux7b80ux4ecb}

本项目展示如何在昇腾310B上部署一个高效的实时目标跟踪系统。系统能够在视频流中自动检测并持续跟踪指定目标(如人员、车辆、特定物体等),即使目标发生遮挡、形变或快速移动,也能保持稳定的跟踪效果。

该项目结合了深度学习目标检测和传统跟踪算法的优势,在保证跟踪精度的同时,实现了边缘设备上的实时推理,为安防监控、智能交通、机器人导航等应用场景提供了技术基础。

\section{内容大纲}\label{ux5185ux5bb9ux5927ux7eb2}

\subsection{硬件准备}\label{ux786cux4ef6ux51c6ux5907}

\begin{itemize}
\tightlist
\item
  \textbf{核心计算单元}: 昇腾310B开发者套件
\item
  \textbf{图像采集}: 高清USB摄像头或IP摄像头
\item
  \textbf{显示设备}: HDMI显示器,用于实时查看跟踪效果
\item
  \textbf{存储设备}: 高速SD卡或USB存储设备,用于保存跟踪视频
\item
  \textbf{网络设备 (可选)}: 以太网连接,用于远程监控
\item
  \textbf{电源}: 稳定的电源供应
\end{itemize}

\emph{硬件连接示意图}

\begin{lstlisting}
昇腾310B ── USB摄像头
    │
    ├── HDMI显示器
    ├── 网络连接
    └── 电源适配器
\end{lstlisting}

\subsection{软件环境}\label{ux8f6fux4ef6ux73afux5883}

\begin{itemize}
\tightlist
\item
  \textbf{操作系统}: Ubuntu 20.04 LTS
\item
  \textbf{CANN版本}: 7.0.RC1 或更高
\item
  \textbf{Python版本}: 3.8.10
\item
  \textbf{核心依赖库}:

  \begin{itemize}
  \tightlist
  \item
    \passthrough{\lstinline!opencv-python!}: 图像处理和视频读取
  \item
    \passthrough{\lstinline!numpy!}: 数值计算
  \item
    \passthrough{\lstinline!torch!}: 深度学习框架
  \item
    \passthrough{\lstinline!torchvision!}: 计算机视觉工具
  \item
    \passthrough{\lstinline!pillow!}: 图像处理
  \item
    \passthrough{\lstinline!matplotlib!}: 数据可视化
  \end{itemize}
\end{itemize}

\emph{环境安装脚本 (\passthrough{\lstinline!setup\_env.sh!})}

\begin{lstlisting}[language=bash]
#!/bin/bash
# 更新系统
sudo apt update && sudo apt upgrade -y

# 安装Python依赖
pip3 install opencv-python numpy torch torchvision pillow matplotlib

# 安装CANN开发套件
# (此处应包含具体的CANN安装步骤)

echo "环境安装完成!"
\end{lstlisting}

\subsection{数据集准备}\label{ux6570ux636eux96c6ux51c6ux5907}

\begin{itemize}
\tightlist
\item
  \textbf{目标检测数据集}:

  \begin{itemize}
  \tightlist
  \item
    \textbf{COCO数据集}: 用于预训练目标检测模型
  \item
    \textbf{自定义数据集}: 根据具体应用场景收集的目标图像
  \end{itemize}
\item
  \textbf{跟踪数据集}:

  \begin{itemize}
  \tightlist
  \item
    \textbf{MOT Challenge}: 多目标跟踪基准数据集
  \item
    \textbf{LaSOT}: 大规模单目标跟踪数据集
  \item
    \textbf{自建跟踪序列}: 针对特定场景的视频序列
  \end{itemize}
\item
  \textbf{数据预处理 (\passthrough{\lstinline!preprocess\_data.py!})}:

  \begin{itemize}
  \tightlist
  \item
    视频帧提取
  \item
    目标标注格式转换
  \item
    数据增强(翻转、缩放、亮度调整)
  \item
    训练/验证集划分
  \end{itemize}
\end{itemize}

\subsection{模型训练与选择}\label{ux6a21ux578bux8badux7ec3ux4e0eux9009ux62e9}

\begin{itemize}
\tightlist
\item
  \textbf{目标检测模型}:

  \begin{itemize}
  \tightlist
  \item
    \textbf{YOLOv8}: 最新的YOLO系列,速度和精度平衡
  \item
    \textbf{SSD MobileNet}: 轻量级检测模型,适合边缘设备
  \item
    \textbf{RetinaNet}: 单阶段检测器,处理小目标效果好
  \end{itemize}
\item
  \textbf{跟踪算法选择}:

  \begin{itemize}
  \tightlist
  \item
    \textbf{DeepSORT}: 结合外观特征的多目标跟踪
  \item
    \textbf{ByteTrack}: 基于简单关联的高性能跟踪算法
  \item
    \textbf{FairMOT}: 端到端的检测与跟踪联合训练
  \end{itemize}
\item
  \textbf{训练流程}:

  \begin{enumerate}
  \def\labelenumi{\arabic{enumi}.}
  \tightlist
  \item
    使用预训练检测模型作为backbone
  \item
    在自定义数据集上fine-tune
  \item
    集成跟踪算法
  \item
    端到端优化跟踪性能
  \end{enumerate}
\end{itemize}

\subsection{模型部署}\label{ux6a21ux578bux90e8ux7f72}

\begin{itemize}
\item
  \textbf{模型转换流程}: ```bash \# 1. PyTorch模型转ONNX python3
  convert\_to\_onnx.py --model yolov8s.pt --output yolov8s.onnx

  \# 2. ONNX转昇腾.om格式 atc --model=yolov8s.onnx --framework=5
  --output=yolov8s\\
  --input\_format=NCHW --input\_shape=``images:1,3,640,640''\\
  --soc\_version=Ascend310B1 ```
\item
  \textbf{实时跟踪主程序 (\passthrough{\lstinline!tracking\_app.py!})}:
  ```python \# 核心流程示例

  \begin{enumerate}
  \def\labelenumi{\arabic{enumi}.}
  \tightlist
  \item
    初始化昇腾推理引擎
  \item
    加载检测和跟踪模型
  \item
    打开视频流
  \item
    循环处理每一帧:

    \begin{itemize}
    \tightlist
    \item
      目标检测
    \item
      跟踪算法更新
    \item
      绘制跟踪轨迹
    \item
      显示结果
    \end{itemize}
  \item
    保存跟踪结果 ```
  \end{enumerate}
\end{itemize}

\subsection{3D打印结构件}\label{dux6253ux5370ux7ed3ux6784ux4ef6}

\begin{itemize}
\tightlist
\item
  \textbf{摄像头云台 (\passthrough{\lstinline!camera\_gimbal.stl!})}:

  \begin{itemize}
  \tightlist
  \item
    支持水平360°、垂直±45°旋转
  \item
    可配合步进电机实现自动跟踪
  \end{itemize}
\item
  \textbf{设备保护外壳
  (\passthrough{\lstinline!protective\_case.stl!})}:

  \begin{itemize}
  \tightlist
  \item
    防尘防水设计
  \item
    散热孔布局优化
  \end{itemize}
\item
  \textbf{安装支架 (\passthrough{\lstinline!mounting\_bracket.stl!})}:

  \begin{itemize}
  \tightlist
  \item
    适配标准三脚架接口
  \item
    多角度调节机构
  \end{itemize}
\end{itemize}

\emph{3D打印参数建议}: - \textbf{材料}: PETG (强度高,耐温性好) -
\textbf{层高}: 0.15mm (保证精度) - \textbf{填充率}: 30\%
(强度与重量平衡)

\subsection{用户手册}\label{ux7528ux6237ux624bux518c}

\subsubsection{快速开始}\label{ux5febux901fux5f00ux59cb}

\begin{enumerate}
\def\labelenumi{\arabic{enumi}.}
\tightlist
\item
  \textbf{硬件连接}: 按照连接图组装硬件
\item
  \textbf{环境配置}: 运行 \passthrough{\lstinline!setup\_env.sh!}
  安装依赖
\item
  \textbf{模型下载}: 下载预训练模型文件
\item
  \textbf{启动跟踪}:
  \passthrough{\lstinline!python3 tracking\_app.py --source 0!}
\end{enumerate}

\subsubsection{高级配置}\label{ux9ad8ux7ea7ux914dux7f6e}

\begin{itemize}
\tightlist
\item
  \textbf{跟踪参数调优}: 修改 \passthrough{\lstinline!config.yaml!}
  中的跟踪阈值
\item
  \textbf{多目标跟踪}: 启用 \passthrough{\lstinline!--multi-target!}
  参数
\item
  \textbf{结果保存}: 使用 \passthrough{\lstinline!--save-video!}
  保存跟踪视频
\end{itemize}

\subsubsection{性能优化}\label{ux6027ux80fdux4f18ux5316}

\begin{itemize}
\tightlist
\item
  \textbf{推理加速}: 启用混合精度推理
\item
  \textbf{内存优化}: 调整批处理大小
\item
  \textbf{延迟优化}: 减少后处理步骤
\end{itemize}

\section{源代码结构}\label{ux6e90ux4ee3ux7801ux7ed3ux6784}

\begin{lstlisting}
tracking_project/
├── models/
│   ├── detection/      # 检测模型
│   ├── tracking/       # 跟踪算法
│   └── utils/         # 工具函数
├── data/
│   ├── datasets/      # 训练数据
│   └── pretrained/    # 预训练模型
├── configs/           # 配置文件
├── scripts/           # 训练和转换脚本
└── demo/             # 演示程序
\end{lstlisting}

\section{效果演示}\label{ux6548ux679cux6f14ux793a}

\begin{itemize}
\tightlist
\item
  \textbf{单目标跟踪}: 视频中展示对人员的持续跟踪
\item
  \textbf{多目标跟踪}: 同时跟踪多个车辆或行人
\item
  \textbf{遮挡处理}: 目标被遮挡后重新出现的跟踪恢复
\item
  \textbf{实时性能}: FPS指标和延迟统计
\end{itemize}
