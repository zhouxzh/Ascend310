% !TEX program = xelatex
% 使用 XeLaTeX 编译
\documentclass[fontsize=12pt, paper=a4, oneside, DIV=calc]{scrbook}

% 基本设置
\usepackage[UTF8]{ctex} % 中文支持
\usepackage{fontspec} % 字体设置
\usepackage{xeCJK} % 中日韩文字支持

% 设置默认字体为无衬线体(思源黑体)
% 中文/英文字体配对:正文用思源宋体(Noto Serif CJK SC)+ Noto Serif,标题用思源黑体(Noto Sans CJK SC)+ Noto Sans,等宽用 Noto Sans Mono CJK SC

\setCJKmainfont{Noto Serif CJK SC}[Scale=MatchLowercase, AutoFakeSlant = 0.15]        % 正文字体(衬线 / 宋体样式)
\setCJKsansfont{Noto Sans CJK SC}[Scale=MatchLowercase, AutoFakeSlant = 0.15] % 无衬线(适合章节标题、图例)
\setCJKmonofont{Noto Sans Mono CJK SC}[Scale=MatchLowercase] % 等宽字体(代码、表格)

% 页面布局和外观设置
\usepackage[automark,headsepline]{scrlayer-scrpage} % 页眉页脚
\clearpairofpagestyles
\ohead{\headmark} % 页眉显示章节名
\ofoot[\pagemark]{\pagemark} % 页脚显示页码

% 数学相关包
\usepackage{amsmath, amssymb, amsthm}

% 图形和颜色
\usepackage{graphicx}
\usepackage{xcolor}
\definecolor{mainblue}{RGB}{0, 90, 156}

\usepackage{booktabs}
\usepackage{longtable}
\usepackage{array}
\usepackage{calc} % 提供增强的计算功能
% 为 Pandoc 片段提供安全的宏定义,避免 Undefined control sequence
\usepackage{graphicx}
\usepackage{listings}
\usepackage{textcomp}

\providecommand{\passthrough}[1]{#1}
% 保证只有在未定义时才提供,避免覆盖模板已有定义
\providecommand{\passthrough}[1]{#1}
\providecommand{\tightlist}{\setlength{\itemsep}{0pt}\setlength{\parskip}{0pt}}
% 自动限制图片不超出行宽并保持纵横比
\makeatletter
\def\maxwidth{\ifdim\Gin@nat@width>\linewidth \linewidth \else \Gin@nat@width\fi}
\makeatother
\setkeys{Gin}{width=\maxwidth,keepaspectratio}

% 如果希望 pandoc 生成的 \pandocbounded 包裹为 figure 环境:
\providecommand{\pandocbounded}[1]{\begin{center}#1\end{center}} % 或者 \providecommand{\pandocbounded}[1]{#1}

\usepackage{enumitem}
\setlist{nosep} % 去掉列表间距
% 超链接设置
\usepackage{hyperref}
\hypersetup{
    colorlinks=true,
    linkcolor=mainblue,
    citecolor=mainblue,
    urlcolor=mainblue,
    pdftitle={昇腾310B实战——从入门到精通边缘计算与人工智能},
    pdfauthor={周贤中}
}

% 定理环境设置
\newtheoremstyle{break}
  {\topsep}{\topsep}%
  {\itshape}{}%
  {\bfseries}{}%
  {\newline}{}%
\theoremstyle{break}
\newtheorem{theorem}{定理}[chapter]
\newtheorem{lemma}[theorem]{引理}
\newtheorem{proposition}[theorem]{命题}
\newtheorem{corollary}[theorem]{推论}
\newtheorem{definition}[theorem]{定义}
\newtheorem{example}[theorem]{例子}
\newtheorem{remark}[theorem]{注记}

% 文档信息
\title{昇腾310B实战\\从入门到精通边缘计算与人工智能}
\author{周贤中}
\date{\today}

% 文档内容开始
\begin{document}

% 封面页
\maketitle

% 前言页
\frontmatter
\chapter*{前言}
\input{chapters/chapter0.tex}


% 目录
\tableofcontents

% 主内容开始
\mainmatter


% 第一章
\chapter{昇腾310B边缘计算基础}
\hypertarget{ux6607ux817e310bux5f00ux53d1ux677fux4ecbux7ecd}{%
\section{昇腾310B开发板介绍}\label{ux6607ux817e310bux5f00ux53d1ux677fux4ecbux7ecd}}

OrangePi
AIpro(8T)开发板是香橙派联合华为精心打造的高性能AI开发板,采用昇腾AI技术路线,搭载的昇腾310B为4核64位处理器+AI处理器,集成图形处理器,支持8TOPS
INT8的AI算力,拥有8GB/16GB LPDDR4X内存,可以外接32GB/64GB/128GB/256GB
eMMC模块,支持双4K高清输出。OrangePi
AIpro(8T)引用了相当丰富的接口,包括两个HDMI输出、GPIO接口、Type-C电源接口、支持SATA/NVMe
SSD 2280的M.2插槽、TF插槽、千兆网口、两个USB3.0、一个USB Type-C
3.0、一个Micro
USB(串口打印调试功能)、两个MIPI摄像头、一个MIPI屏等,预留电池接口,可广泛适用于AI边缘计算、深度视觉学习及视频流AI分析、视频图像分析、自然语言处理、智能小车、机械臂、人工智能、无人机、云计算、AR/VR、智能安防、智能家居等领域,覆盖
AIoT各个行业。 OrangePi
AIpro(8T)支持Ubuntu、openEuler操作系统,满足大多数AI算法原型验证、推理应用开发的需求。
\includegraphics{img0/aipro.png}

\hypertarget{ux5f00ux53d1ux677fux8be6ux7ec6ux89c6ux56fe}{%
\subsection{开发板详细视图}\label{ux5f00ux53d1ux677fux8be6ux7ec6ux89c6ux56fe}}

\includegraphics{img0/4.png} \includegraphics{img0/5.png}
\includegraphics{img0/1.png} \includegraphics{img0/2.png}
\includegraphics{img0/3.png}

\hypertarget{ux5f00ux53d1ux677fux786cux4ef6ux89c4ux683c}{%
\subsection{开发板硬件规格}\label{ux5f00ux53d1ux677fux786cux4ef6ux89c4ux683c}}

\begin{center}\rule{0.5\linewidth}{0.5pt}\end{center}

\hypertarget{ux6240ux9700ux914dux4ef6}{%
\subsection{所需配件}\label{ux6240ux9700ux914dux4ef6}}

\begin{enumerate}
\def\labelenumi{\arabic{enumi}.}
\item
  TF卡
  容量最小为32GB,速率为Class10级以上的闪迪品牌的TF卡,如下图所示。建议使用64G及以上的TF卡,以避免在开发过程中出现磁盘空间不足的问题。
  \includegraphics{img1/tf.jpg}
\item
  TF卡读卡器
  用于读写TF卡,刷写系统,建议选择速率为USB3.0以上的,减少系统刷写的等待时间。
  \includegraphics{img1/reader.jpg}
\item
  HDMI线或HDMI转mini-HDMI线
  主要取决于显示器的接口类型该开发板的视频输出接口为标准HDMI接口。
  \includegraphics{img1/hdmi.jpg} \includegraphics{img1/minihdmi.jpg}
\item
  电源 该开发板的电源输入为PD
  20V,需要搭配支持PD协议20V挡位的65W电源适配器。
  \includegraphics{img1/power.png}
\item
  USB接口的鼠标以及键盘 在无远程访问的条件下对开发板进行本地调试。
  \includegraphics{img1/mouse.png}
\item
  金属配套外壳 用于保护开发板硬件。 \includegraphics{img1/cover.png}
\item
  12V散热风扇以及散热鳍块
  开发板的风扇接口为2pin,输出电压为12v,支持PWM调速。由于该开发板的CPU发热较大,强烈建议安装主动扇热设备。
  \includegraphics{img1/fan.png}
\item
  Type-C转USB 3.0转接线(可选) OrangePi
  AIPro开发板具有一个Type-C接口,协议为USB3.0(不支持USB
  2.0),可外接支持USB3.0以上协议的外置设备。
  \includegraphics{img1/otg.png}
\item
  M.2接口 2280规格的PCIe Nvme SSD(可选)
  开发板的背部设计有M.2接口,可外接一个M.2的SSD作为开发板的系统盘或者存储。
  \includegraphics{img1/nvme.png}
\item
  M.2接口 2280规格的Sata Ngff SSD(可选)
  同样,开发板的M.2接口不仅支持PCIe协议,也支持Sata协议,因此也可以使用Sata协议的SSD。
  \includegraphics{img1/ngff.png}
\item
  香橙派的eMMC模块(可选)
  eMMC(嵌入式多媒体卡)是一种集成了闪存和控制器的低成本存储解决方案,主要用于智能手机、平板电脑和低端笔记本电脑等消费电子产品。其读写速度适中(100-400MB/s),比传统机械硬盘快但不及固态硬盘(SSD),具有体积小、功耗低和易于集成的特点。开发板支持使用eMMC模块作为存储,但需要额外购置eMMC模块。
  \includegraphics{img1/emmc1.png} \includegraphics{img1/emmc2.png}
\item
  USB摄像头模块(可选) 可用于图像识别、视频通话等多方面用途。
  \includegraphics{img1/camera.png}
\item
  网线(可选)
  开发板自带有wifi模块可用于连接wifi,若需要更稳定的网络连接,建议使用网线连接。
  \includegraphics{img1/cable.png}
\item
  树莓派IMX219型号摄像头(MIPI-CSI)(可选)
  开发板带有两个MIPI-CSI接口,可以兼容树莓派的MIPI摄像头,无需占用USB接口。
  \includegraphics{img1/csi.png}
\item
  树莓派5寸MIPI LCD显示屏(可选)
  开发板带有一个MIPI-DSI显示输出接口,可以直接驱动MIPI的显示屏,而无需外接显示器。
  \includegraphics{img1/dsi.png}
\item
  Micro USB数据线(可选) 开发板自带了CH343P芯片,将UART转发为Micro
  USB接口,若需要使用串口对开发板进行调试,则需要使用Micro USB数据线。
  \includegraphics{img1/microusb.png}
\end{enumerate}

\hypertarget{ux4e0bux8f7dux5f00ux53d1ux677fux7684ux7cfbux7edfux955cux50cf}{%
\subsection{下载开发板的系统镜像}\label{ux4e0bux8f7dux5f00ux53d1ux677fux7684ux7cfbux7edfux955cux50cf}}

作为华为生态中重要的一员,开发板不仅支持Ubuntu系统,也支持openEuler系统,但由于开发板自身并无存储,我们在使用开发板的过程中需要使用电脑对TF卡进行系统的刷写,建议使用安装有Windows11
或 Ubuntu22.04以上版本的PC。

首先,打开香橙派官网的\href{http://www.orangepi.cn/html/hardWare/computerAndMicrocontrollers/service-and-support/Orange-Pi-AIpro.html}{技术支持界面}。\includegraphics{img1/技术支持.png}

向下滑动网页,找到官方镜像部分,分为Ubuntu和openEuler两个部分,两个系统都是官方为我们编译完成的,且预装了部分昇腾NPU的应用环境以及软件,非常方便新手用户上手使用。
\includegraphics{img1/官方镜像.png}

\hypertarget{ubuntu}{%
\subsubsection{Ubuntu}\label{ubuntu}}

\begin{enumerate}
\def\labelenumi{\arabic{enumi}.}
\tightlist
\item
  点击下载\includegraphics{img1/download_ubuntu.png}
\item
  复制提取码并跳转\includegraphics{img1/copyandjump.png}
\item
  打开百度网盘的链接后有一个命名为Ubuntu的文件夹,点开该文件夹\includegraphics{img1/folder.png}
\item
  文件夹中,后缀为.xz的文件是镜像压缩包文件,.sha文件是压缩包的md5校验码文件,用于校验镜像包文件是否完整。
\item
  文件夹中的镜像有两种,一种文件名带有Desktop的,是带有GUI图形化界面的,另一种文件名带有minimal的,是不具有图形化界面的,只有命令行界面。建议新学习的用户使用带有desktop的镜像。
  \includegraphics{img1/chooseubuntu.png}
\item
  下载后先校验压缩包是否完整,后解压压缩包
\end{enumerate}

\hypertarget{openeuler}{%
\subsubsection{openEuler}\label{openeuler}}

\begin{enumerate}
\def\labelenumi{\arabic{enumi}.}
\tightlist
\item
  点击下载\includegraphics{img1/download_openeuler.png}
\item
  复制提取码并跳转\includegraphics{img1/cpjp.png}
\item
  打开百度网盘的链接后有一个命名为OpenEuler的文件夹,点开该文件夹\includegraphics{img1/folderr.png}
\item
  文件夹中,后缀为.xz的文件是镜像压缩包文件,.sha文件是压缩包的md5校验码文件,用于校验镜像包文件是否完整。
\item
  文件夹中的镜像只有一种,即具有GUI图形化界面的openEuler系统。\includegraphics{img1/chooseeuler.png}
\item
  下载后先校验压缩包是否完整,后解压压缩包
\end{enumerate}

\hypertarget{ux4f7fux7528md5ux6821ux9a8cux4e0bux8f7dux7684ux6587ux4ef6}{%
\subsubsection{使用md5校验下载的文件}\label{ux4f7fux7528md5ux6821ux9a8cux4e0bux8f7dux7684ux6587ux4ef6}}

在Windows系统下,可以使用\texttt{certutil\ -hashfile\ \textless{}filename\textgreater{}\ md5};在Ubuntu系统下,可以使用\texttt{md5sum\ \textless{}filename\textgreater{}};在MacOS系统下,可以使用\texttt{md5\ \textless{}filename\textgreater{}}进行计算,此处以Windows系统为例:在文件夹按住Shift键并单击鼠标右键,选择``在终端(Powershell/命令提示符)中打开''\includegraphics{img1/shell.png},然后在打开的窗口中输入\texttt{certutil\ -hashfile\ opiaipro\_ubuntu22.04\_desktop\_aarch64\_20241128.img.xz\ md5}\includegraphics{img1/md5.png},将得到的md5值与\texttt{opiaipro\_ubuntu22.04\_desktop\_aarch64\_20241128.img.xz.sha}文件进行对比,若一致可进行下一步操作,否则需要重新下载。

\hypertarget{ux5237ux5199ux7cfbux7edfux5230tfux5361}{%
\subsection{刷写系统到TF卡}\label{ux5237ux5199ux7cfbux7edfux5230tfux5361}}

\hypertarget{ux4e0bux8f7dux5e76ux5b89ux88c5ux5fc5ux8981ux7684ux5de5ux5177}{%
\subsubsection{下载并安装必要的工具}\label{ux4e0bux8f7dux5e76ux5b89ux88c5ux5fc5ux8981ux7684ux5de5ux5177}}

\begin{quote}
下载链接:\href{http://www.orangepi.cn/html/hardWare/computerAndMicrocontrollers/service-and-support/Orange-Pi-AIpro.html}{官网}
\href{https://pan.baidu.com/s/1Jho73pw91r5GJD2KijY45Q?pwd=3xuz\#list/path=\%2F}{百度网盘}
1. SD Card Formatter
这个是TF卡的快速格式化工具,在每次需要刷写系统之前,都必须先对TF卡进行格式化操作,若不格式化在后续的刷写系统过程中有较大概率出错。
2. balenaEther 这个是系统镜像的刷写工具,用于刷写img镜像文件进入TF卡。
\end{quote}

\hypertarget{ux683cux5f0fux5316tfux5361}{%
\subsubsection{格式化TF卡}\label{ux683cux5f0fux5316tfux5361}}

\begin{enumerate}
\def\labelenumi{\arabic{enumi}.}
\tightlist
\item
  将TF卡插入读卡器中,并将读卡器插入电脑
\item
  打开SD Card Formatter软件\includegraphics{img1/SDFmt.png}
\item
  点击右下角Format按键,格式化TF卡\includegraphics{img1/fmt.png}
  \textgreater{}
  警告内容是关于格式化操作会清除TF卡上原有的所有数据,此处选是\includegraphics{img1/warning.png}
\item
  等待软件格式化完成,并点击确定\includegraphics{img1/fmtfin.png}
\end{enumerate}

\hypertarget{ux5237ux5199ux7cfbux7edfux5230tfux5361ux4ee5ubuntuux4e3aux4f8b}{%
\subsubsection{刷写系统到TF卡(以Ubuntu为例)}\label{ux5237ux5199ux7cfbux7edfux5230tfux5361ux4ee5ubuntuux4e3aux4f8b}}

\begin{quote}
此处以Ubuntu为例 1.
打开balenaEther,选择``从文件烧录''\includegraphics{img1/ether1.png} 2.
选择好要烧录的镜像文件(\textbf{.img}格式),再选择目标磁盘为TF卡对应的位置,如图中名称为``SDXC
Card''的位置,选中并选择``选定1''。\includegraphics{img1/chooseether.png}
3. 点击``现在烧录!'',耐心等待烧录完成。\includegraphics{img1/dd.png}
4. 烧录完成后进入校验过程,也请耐心等待。\includegraphics{img1/val.png}
5.
烧录完成后即可关闭程序,并安全弹出TF卡\includegraphics{img1/finish.png}
\end{quote}

\hypertarget{ux5237ux5199ux7cfbux7edfux5230emmc}{%
\subsubsection{刷写系统到eMMC}\label{ux5237ux5199ux7cfbux7edfux5230emmc}}

由于板上并不自带有eMMC模块,若要想使用需要额外购买香橙派的eMMC模块,此处暂时不列入参考,若需使用,请查阅香橙派的用户手册。

\hypertarget{ux5237ux5199ux7cfbux7edfux5230ssd}{%
\subsubsection{刷写系统到SSD}\label{ux5237ux5199ux7cfbux7edfux5230ssd}}

开发板带有M.2接口,可以使用SSD作为启动设备。但SSD需要自行准备,且根据香橙派

\hypertarget{ux8c03ux6574ux8bbeux5907ux542fux52a8ux65b9ux5f0fux7684ux62e8ux7801ux5f00ux5173}{%
\subsubsection{调整设备启动方式的拨码开关}\label{ux8c03ux6574ux8bbeux5907ux542fux52a8ux65b9ux5f0fux7684ux62e8ux7801ux5f00ux5173}}

开发板支持多种启动方式,包括TF卡、eMMC以及M.2
SSD,当这些存储设备都同时存在时,需要让开发板选定一个存储设备作为启动来源。
\includegraphics{img1/bootswitch.png}
两个开关都有左、右两种状态,因此共有4种状态,但是目前开发板仅使用3种模式,对应的参数表如下:
\textbar{} Boot1开关 \textbar{} Boot2开关 \textbar{} 启动设备 \textbar{}
\textbar{} :------: \textbar{} :------: \textbar{} :------: \textbar{}
\textbar{} 左 \textbar{} 左 \textbar{} 未使用 \textbar{} \textbar{} 右
\textbar{} 右 \textbar{} TF卡 \textbar{} \textbar{} 左 \textbar{} 右
\textbar{} eMMC \textbar{} \textbar{} 右 \textbar{} 左 \textbar{} M.2
SSD (Nvme或Ngff)\textbar{}
切换拨码开关后,必须要将开发板完全断电再重新上电才能使新的启动配置生效,使用RESET按键重启则不会使新的启动配置生效。

\hypertarget{ux542fux52a8ux5f00ux53d1ux677fubuntu}{%
\subsection{启动开发板(Ubuntu)}\label{ux542fux52a8ux5f00ux53d1ux677fubuntu}}

\begin{itemize}
\tightlist
\item
  图形化界面
\end{itemize}

\begin{enumerate}
\def\labelenumi{\arabic{enumi}.}
\tightlist
\item
  将系统刷写完成的TF卡从读卡器中取出,插入开发板的TF卡插槽中,并确保两个启动开关的位置均在右边,接入HDMI数据线到靠近USB3.0接口的HDMI0接口,然后将Type-C电源线插入开发板最边缘的TYPE-C供电口,等待风扇的声音变小以及屏幕出现系统登录界面。
  \includegraphics{img1/beforelogin.png}
\item
  进入登录界面后,将键盘接入开发板的USB接口中,默认的登录用户名是\texttt{HwHiAiUser},输入该账户的密码\texttt{Mind@123},登录进入系统。
  \includegraphics{img1/logingui.png} 默认账户表格: \textbar{} 用户名
  \textbar{} 密码 \textbar{} \textbar{} :---: \textbar{} :--: \textbar{}
  \textbar{} root \textbar{} Mind@123 \textbar{} \textbar{} HwHiAiUser
  \textbar{} Mind@123 \textbar{}
\end{enumerate}

\begin{itemize}
\tightlist
\item
  串口界面
\end{itemize}

\begin{enumerate}
\def\labelenumi{\arabic{enumi}.}
\tightlist
\item
  使用USB2TTL模块,与开发板的GPIO口进行连线\includegraphics{img1/gpio_ttl.png},开发板的TX(GPIO8)接入USB2TTL模块的RX接口,开发板的RX(GPIO10)则接入模块的TX接口,并连接好GND接地,在Windows电脑下可以使用PUTTY连接串口。
\item
  使用开发板自带的Micro
  USB接口进行串口调试,该方法更为方便,只需要一根Micro
  USB数据线,接入电脑后打开设备管理器查询对应的串口,然后使用PUTTY进行链接即可。
  以Micro USB接口为例:
\item
  使用Micro USB数据线连接开发板和电脑
\item
  打开电脑的设备管理器,选择端口,寻找开发板对应的串口端口号\includegraphics{img1/ttl.png}
\item
  打开串口调试软件(PUTTY)\includegraphics{img1/putty.png},将Connection
  Type选择为\texttt{Serial},然后在Serial
  Line处将端口号修改为设备管理器中查到的端口号,如作者此处端口号为\texttt{COM3},此外,还需要将Speed从9600修改为115200,最后点击Open打开串口。
\item
  等待出现\texttt{Ubuntu\ 22.04.3\ LTS\ orangepiaipro\ ttyAM0}字样,输入登录的用户名HwHiAiUser并回车,然后输入密码Mind@123并回车,注意在输入密码的时候屏幕并不会显示任何东西,登陆后的界面如图所示。
  \includegraphics{img1/serial.png} \includegraphics{img1/login.png}
\end{enumerate}

\hypertarget{ubuntu-xfceux684cux9762ux4f7fux7528ux8bf4ux660e}{%
\subsection{Ubuntu
Xfce桌面使用说明}\label{ubuntu-xfceux684cux9762ux4f7fux7528ux8bf4ux660e}}

目前系统仅支持Ubuntu 22.04 - Jammy系统,内核版本为Linux 5.10 \#\#\#\#
当前版本适配情况
请详见香橙派官方的用户手册,有部分功能仅支持使用官方程序进行测试,无法直接从系统中调用,在使用过程中需注意这些限制。

\hypertarget{hdmiux53e3ux4f7fux7528}{%
\subsubsection{HDMI口使用}\label{hdmiux53e3ux4f7fux7528}}

开发板有两个HDMI2.0 接口,目前只有HDMI0 支持显示Linux
系统的桌面,当Linux 系统的桌面系统关闭时,HDMI0 和HDMI1 还可以用于NVR
二次开 发场景输出图片。

\hypertarget{ux97f3ux9891ux4f7fux7528}{%
\subsubsection{音频使用}\label{ux97f3ux9891ux4f7fux7528}}

Linux 内核没有适配耳机和HDMI 等的ALSA 音频驱动,此部分驱动还在开
发中,目前只能通过音频样例代码来测试耳机、HDMI 的音频播放和板载MIC
的录音功能。或者自行购买Linux系统免驱的USB外置声卡,经测试可以正常使用。


% 第二章
\chapter{CANN 软件栈核心与模型转换全流程}
\begin{center}\rule{0.5\linewidth}{0.5pt}\end{center}

\hypertarget{gpioux53e3ux7684ux5f15ux811aux987aux5e8f}{%
\subsection{GPIO口的引脚顺序}\label{gpioux53e3ux7684ux5f15ux811aux987aux5e8f}}

如图,单号引脚和双号引脚分别在一排。 \includegraphics{img2/gpio.png}
\includegraphics{img2/gpio2.png}

注意事项: 1. 40 pin 接口中总共有26 个GPIO 口,但8 号和10
号引脚默认是用于调试串 口功能的,并且这两个引脚和Micro USB
调试串口是连接在一起的,所以这两个 引脚请不要设置为GPIO 等功能。 2.
所有的GPIO 口的电压都是3.3v。 3. 40 pin 接口中27 号和28 号引脚只有I2C
的功能,没有GPIO 等其他复用功 能,另外这两个引脚的电压默认都为1.8v。


% 第三章
\chapter{昇腾310B算子开发基础}
昇腾310B在通用算子覆盖广度上已能满足大多数推理任务,但在以下场景,自定义算子(Custom
Op)能显著提升功能完备性与性能确定性:模型含未支持/半支持算子、复合算子频繁导致访存过多、需要业务特化(如阈值/形态学/后处理融合)、或内置实现对特定尺寸/布局性能欠佳。第三章将给出``为什么、怎么做、如何验证与上线''的完整路径。

\section{算子开发概述}\label{ux7b97ux5b50ux5f00ux53d1ux6982ux8ff0}

\begin{itemize}
\tightlist
\item
  目标与收益:

  \begin{itemize}
  \tightlist
  \item
    功能补齐:覆盖模型图中未支持或语义差异较大的算子;
  \item
    性能确定性:融合多算子、减少GM\textless-\textgreater UB搬运与中间落地、利用向量化内核;
  \item
    工程可维护:以``算子契约''形式固化输入/输出/属性与边界行为,便于回归与复用。
  \end{itemize}
\item
  执行形态:

  \begin{itemize}
  \tightlist
  \item
    AI Core(推荐):基于 TBE/TE/TIK 运行于 NPU 核心,适合数值密集型;
  \item
    AICPU(可选):C/C++ 在 AICPU/Host
    侧执行,适合控制流/轻量处理(注意H2D/D2H成本)。
  \end{itemize}
\item
  产物要素:

  \begin{itemize}
  \tightlist
  \item
    算子描述(op info/proto):声明
    op\_type、inputs/outputs、dtype\_format 组合、属性与形状推断;
  \item
    算子实现(Kernel):TE/TIK 计算+调度或 AICPU C++ 实现;
  \item
    注册与打包:产物按规范放入 OPP 目录,ATC/Runtime 可发现与加载。
  \end{itemize}
\end{itemize}

\section{开发的理论基础}\label{ux5f00ux53d1ux7684ux7406ux8bbaux57faux7840}

\begin{enumerate}
\def\labelenumi{\arabic{enumi})}
\tightlist
\item
  硬件与存储层次:
\end{enumerate}

\begin{itemize}
\tightlist
\item
  GM(Global Memory):容量大、带宽高;
\item
  UB(Unified Buffer):片上高速缓存,容量有限;
\item
  DMA:GM↔UB 的数据搬运,偏好大块连续传输;
\item
  向量/标量单元:支持vadd/vmul/vmax等,需数据对齐(常见16/32)。
\end{itemize}

\begin{enumerate}
\def\labelenumi{\arabic{enumi})}
\setcounter{enumi}{1}
\tightlist
\item
  计算表达与调度:
\end{enumerate}

\begin{itemize}
\tightlist
\item
  TE(Tensor Expression)描述计算公式;Schedule 负责
  tile/并行/向量化/缓存;
\item
  TIK 提供更贴近硬件的 DSL,便于精细控制 DMA 与 UB 管理;
\item
  目标:以较少的GM往返在UB内完成尽可能多的计算,提升算子算子效率与吞吐。
\end{itemize}

\begin{enumerate}
\def\labelenumi{\arabic{enumi})}
\setcounter{enumi}{2}
\tightlist
\item
  算子契约(Operator Contract):
\end{enumerate}

\begin{itemize}
\tightlist
\item
  输入/输出张量的
  shape、dtype、layout(NCHW/NC1HWC0等)、属性(如alpha、mode);
\item
  广播与对齐规则、边界行为(溢出/饱和/舍入)、精度策略(FP16/FP32混合);
\item
  动态shape与静态shape:实现需覆盖契约内的形状组合并保证UB不溢出。
\end{itemize}

\begin{enumerate}
\def\labelenumi{\arabic{enumi})}
\setcounter{enumi}{3}
\tightlist
\item
  数值与精度:
\end{enumerate}

\begin{itemize}
\tightlist
\item
  FP16 常用于 310B 推理通路;必要时在关键步骤采用临时 FP32 计算再回写;
\item
  误差控制:选择合适的舍入策略,避免饱和/下溢导致NAN/INF。
\end{itemize}

\section{开发流程(AI Core
路线)}\label{ux5f00ux53d1ux6d41ux7a0bai-core-ux8defux7ebf}

\begin{enumerate}
\def\labelenumi{\arabic{enumi}.}
\tightlist
\item
  环境准备与约束
\end{enumerate}

\begin{itemize}
\tightlist
\item
  安装 CANN/Toolkit 并确认 \passthrough{\lstinline!atc --version!}
  正常;
\item
  设置环境变量:\passthrough{\lstinline!ASCEND\_INSTALL\_PATH!}、\passthrough{\lstinline!ASCEND\_OPP\_PATH!};
\item
  目标芯片:\passthrough{\lstinline!soc\_version=Ascend310B!};优先使用
  FP16 与硬件友好布局(如NC1HWC0)。
\end{itemize}

\begin{enumerate}
\def\labelenumi{\arabic{enumi}.}
\setcounter{enumi}{1}
\tightlist
\item
  定义算子信息(op info/proto)
\end{enumerate}

\begin{itemize}
\tightlist
\item
  声明 \passthrough{\lstinline!op\_type!}、inputs/outputs
  名称与数量、可支持的 \passthrough{\lstinline!dtype\_format!}
  组合、属性与默认值;
\item
  提供形状推断规则(静态或依据属性/输入维度计算)。
\end{itemize}

\begin{enumerate}
\def\labelenumi{\arabic{enumi}.}
\setcounter{enumi}{2}
\tightlist
\item
  编写算子实现(TE/TBE/TIK)
\end{enumerate}

\begin{itemize}
\tightlist
\item
  计算表达(示例:Add+ReLU 融合伪代码):
\end{itemize}

\begin{lstlisting}
# y = relu(x1 + x2)
import te.lang.cce as tbe
from te import tvm

def add_relu_compute(x1, x2):
        y = tbe.vadd(x1, x2)
        z = tbe.vmaxs(y, tvm.const(0.0, x1.dtype))
        return z
\end{lstlisting}

\begin{itemize}
\tightlist
\item
  调度要点:

  \begin{itemize}
  \tightlist
  \item
    Tile 到 UB 容量可承载的块大小;
  \item
    连续向量访问,减少非对齐;
  \item
    合并搬运,避免频繁小块 DMA;
  \item
    小尺寸路径避免调度开销超过计算开销。
  \end{itemize}
\end{itemize}

\begin{enumerate}
\def\labelenumi{\arabic{enumi}.}
\setcounter{enumi}{3}
\tightlist
\item
  编译与注册
\end{enumerate}

\begin{itemize}
\tightlist
\item
  使用 Toolkit 提供的编译入口生成 kernel 与元数据;
\item
  将实现与描述文件放入 \passthrough{\lstinline!ASCEND\_OPP\_PATH!} 下
  custom 目录(如
  \passthrough{\lstinline!op\_impl/custom/ai\_core/tbe!}、\passthrough{\lstinline!op\_proto/custom!})。
\end{itemize}

\begin{enumerate}
\def\labelenumi{\arabic{enumi}.}
\setcounter{enumi}{4}
\tightlist
\item
  与 ATC 集成
\end{enumerate}

\begin{itemize}
\tightlist
\item
  转换模型时指定 \passthrough{\lstinline!--soc\_version=Ascend310B!};
\item
  确保 OPP 路径可被 ATC 读取,必要时调整
  \passthrough{\lstinline!--op\_select\_implmode!};
\item
  转换日志中应能看到自定义算子被匹配与编译。
\end{itemize}

\begin{enumerate}
\def\labelenumi{\arabic{enumi}.}
\setcounter{enumi}{5}
\tightlist
\item
  运行时部署
\end{enumerate}

\begin{itemize}
\tightlist
\item
  目标环境包含同版本 OPP(含 custom 产物);
\item
  设置环境变量使 Runtime 能定位到自定义实现;
\item
  按常规 ACL 流程加载 OM 并执行推理。
\end{itemize}

\begin{enumerate}
\def\labelenumi{\arabic{enumi}.}
\setcounter{enumi}{6}
\tightlist
\item
  验证与度量
\end{enumerate}

\begin{itemize}
\tightlist
\item
  功能:与 NumPy/ONNX
  参考实现对齐,随机多组张量比较(平均绝对/相对误差、边界样本);
\item
  性能:Warmup≥3 次,采样≥50 次,统计 avg/p95/FPS;
\item
  资源:Profiling 检查 MemCopy 占比、Kernel 占比、Idle;
\item
  兼容:覆盖不同 shape/dtype/layout 组合。
\end{itemize}

\begin{enumerate}
\def\labelenumi{\arabic{enumi}.}
\setcounter{enumi}{7}
\tightlist
\item
  打包与版本化
\end{enumerate}

\begin{itemize}
\tightlist
\item
  输出 \passthrough{\lstinline!op\_contract.yaml!}(契约)与
  \passthrough{\lstinline!benchmark.json!}(性能);
\item
  目录建议:
\end{itemize}

\begin{lstlisting}
op_pkg/<op_type>/<version>/
    ├─ op_proto/custom/
    ├─ op_impl/custom/ai_core/tbe/
    ├─ tests/
    └─ docs/
\end{lstlisting}

\section{常见问题与排查}\label{ux5e38ux89c1ux95eeux9898ux4e0eux6392ux67e5}

\begin{itemize}
\tightlist
\item
  ATC 提示 Unsupported Op:检查 op 描述是否生效、路径与
  \passthrough{\lstinline!soc\_version!} 是否匹配;
\item
  运行时回退(fallback):确认 \passthrough{\lstinline!dtype\_format!}
  覆盖到当前张量组合;
\item
  性能无提升:检查是否出现额外 layout 转换、tile 过小造成 DMA 频繁;
\item
  精度异常:核对归一化/广播规则、溢出与舍入策略,必要时局部切 FP32;
\item
  动态 shape OOM:缩小 tile 或分桶处理,保证 UB 与工作区不溢出。
\end{itemize}

\section{章节小结}\label{ux7ae0ux8282ux5c0fux7ed3}

自定义算子是 310B
场景下实现``功能补齐与性能确定性''的关键手段。遵循``明确契约 → 正确调度
→ 可观测验证 →
规范打包''的路径,选择计算/访存比例合适、出现频繁的目标起步,先易后难、以基线与回归保障质量与收益的可持续。

\section{实践任务}\label{ux5b9eux8df5ux4efbux52a1}

\begin{enumerate}
\def\labelenumi{\arabic{enumi}.}
\tightlist
\item
  选择你项目中的一个复合算子(例如归一化+阈值),写出算子契约草案(IO/attr/dtype\_format/边界)。
\item
  基于 TE 写出该算子的计算表达伪代码,并说明预期的 tile 与向量化策略。
\item
  在开发环境完成编译注册,将产物放入 OPP custom 目录并用一个最小模型验证
  ATC 识别。
\item
  设计功能与性能验证脚本:随机张量对齐、Warmup/采样策略、输出 avg/p95
  与资源占比。
\item
  生成 \passthrough{\lstinline!op\_contract.yaml!} 与
  \passthrough{\lstinline!benchmark.json!},并归档到
  \passthrough{\lstinline!op\_pkg/<op\_type>/<version>/!}。
\end{enumerate}


% 第四章
\chapter{典型模型部署实践}
\section{章节总览}\label{ux7ae0ux8282ux603bux89c8}

本章以``统一流程 → 四类典型任务(分类/检测/OCR/NLP)→ 多模型 Pipeline →
工程化目录与脚本 → 性能基线采集 →
问题诊断''逻辑展开,强调``可复现、可量化、可演进''的部署范式。所有示例策略均可推广到后续复杂场景(多输入、多分辨率、流式/批式混合)。

\section{统一部署工作流与契约化}\label{ux7edfux4e00ux90e8ux7f72ux5de5ux4f5cux6d41ux4e0eux5951ux7ea6ux5316}

标准六步:模型选择 → 框架导出 ONNX → ATC 转换(参数冻结)→
推理引擎封装(I/O 契约)→ 运行形态编排 → 验证(精度 + 性能)。
核心产物: \textbar{} 文件 \textbar{} 作用 \textbar{} \textbar{} ----
\textbar{} ---- \textbar{} \textbar{} export.py \textbar{} 导出 \& 简化
ONNX \textbar{} \textbar{} atc.sh \textbar{} 标准化转换命令 \textbar{}
\textbar{} config.yaml \textbar{} 输入/归一化/颜色/阈值 \textbar{}
\textbar{} signature.json \textbar{} 模型输入输出字段与 dtype \textbar{}
\textbar{} metrics.json \textbar{} 性能统计(avg/p95/memory) \textbar{}

输入预处理必须模块化,业务层仅提供原始图像对象;可在 AIPP
中下沉部分(色彩空间、均值/方差),减少 Host 侧拷贝和转换。

\section{图像分类:ResNet /
MobileNet}\label{ux56feux50cfux5206ux7c7bresnet-mobilenet}

\subsection{模型导出}\label{ux6a21ux578bux5bfcux51fa}

PyTorch →
ONNX:\passthrough{\lstinline!torch.onnx.export(model, dummy, opset\_version=13, dynamic\_axes=None)!};确保去掉训练专属层(Dropout,
BN 置 eval)。 \#\#\# 预处理一致性 1. Resize: 保持短边 256 → CenterCrop
224。 2. Normalize: mean/std 与训练保持一致。 3. Layout:
NCHW;若原始图像为 HWC(RGB) → 转 BGR/或保持一致并在 config 标记。 \#\#\#
转换要点
\passthrough{\lstinline!--precision\_mode=allow\_fp32\_to\_fp16!};若需
INT8:先做离线标定导出校准表,再加量化参数。 \#\#\# 推理后处理 Softmax →
ArgTopK → LabelMap。为避免数值不稳定:FP16 logits 可先转 FP32 再
softmax。 \#\#\# 性能采集 Warmup 5 次,采集 100 次:记录 avg, p50, p95,
max;统计预处理耗时占比:\passthrough{\lstinline!pre\_ms / total\_ms!},超过
25\% 提示 AIPP 下沉或批处理优化。

\section{目标检测:YOLO /
FasterRCNN}\label{ux76eeux6807ux68c0ux6d4byolo-fasterrcnn}

\subsection{输入尺寸与
Letterbox}\label{ux8f93ux5165ux5c3aux5bf8ux4e0e-letterbox}

Letterbox 使图像等比例缩放 +
填充,保持方形输入。部署需重现训练阶段相同逻辑,否则框坐标偏移。保存
\passthrough{\lstinline!scale!} 与 \passthrough{\lstinline!pad!}
用于反算原始坐标。 \#\#\# 多输出解析 YOLOv5s OM
输出通常包含一个或多个特征拼接张量:\passthrough{\lstinline!(num\_boxes, attributes)!};后处理:过滤
conf \textgreater{} 阈值 → 按类合并 → NMS。 \#\#\# NMS 实现决策
\textbar{} 方案 \textbar{} 优点 \textbar{} 缺点 \textbar{} \textbar{}
---- \textbar{} ---- \textbar{} ---- \textbar{} \textbar{} CPU Python
\textbar{} 简单 \textbar{} 高开销,多框场景慢 \textbar{} \textbar{} CPU
C++ SIMD \textbar{} 中等复杂 \textbar{} 仍需 D2H 拷贝 \textbar{}
\textbar{} Device Kernel \textbar{} 减少拷贝 \textbar{} 实现复杂
\textbar{} 先评估 D2H + CPU NMS 占比,\textgreater15\% 再考虑下沉。
\#\#\# 动态尺度支持 转换阶段可生成多尺度 OM 或使用动态
shape;推荐:统计输入分辨率 → 选择 3 桶(640/704/768)提升命中率。

\section{OCR:文本检测 + 识别
Pipeline}\label{ocrux6587ux672cux68c0ux6d4b-ux8bc6ux522b-pipeline}

\subsection{结构}\label{ux7ed3ux6784}

检测模型(DB) → 文本框多边形 → 透视裁剪 → 识别模型(CRNN / SVTR)。
\#\#\# 难点与策略 \textbar{} 环节 \textbar{} 风险 \textbar{} 对策
\textbar{} \textbar{} ---- \textbar{} ---- \textbar{} ---- \textbar{}
\textbar{} 多边形裁剪 \textbar{} 仿射失真 \textbar{} 统一仿射矩阵 +
padding \textbar{} \textbar{} 长短文本差异 \textbar{} 序列长度不均
\textbar{} 动态 Batch 分组(长度分桶) \textbar{} \textbar{} 识别延迟
\textbar{} 串行处理 \textbar{} 检测与上一批识别并行 \textbar{}
\textbar{} 字典映射 \textbar{} 乱码/对齐 \textbar{} 固定 vocab + 版本号
\textbar{} \#\#\# CTC 解码 贪心:移除重复与 blank;大规模需 Beam
Search(权衡性能)。

\section{NLP:BERT 推理优化}\label{nlpbert-ux63a8ux7406ux4f18ux5316}

\subsection{序列长度策略}\label{ux5e8fux5217ux957fux5ea6ux7b56ux7565}

\begin{enumerate}
\def\labelenumi{\arabic{enumi}.}
\tightlist
\item
  静态最大长度(简单,浪费算力)。
\item
  Bucketing:按输入长短分类(32/64/128/256),多 OM。
\item
  动态 shape:需评估内存分配抖动;提前预热各常见长度。 \#\#\# FP16
  注意点 LayerNorm/Softmax 数值范围敏感;若发现精度下降:保持部分算子
  FP32(通过混合精度策略或模型修改)。 \#\#\# 性能指标
  tokens/s、avg\_latency\_ms(batch=1 与
  batch\textgreater1)、内存占用;观察自注意力占比,必要时进行剪枝(去除冗余
  head)或蒸馏。
\end{enumerate}

\section{多模型 Pipeline
串联}\label{ux591aux6a21ux578b-pipeline-ux4e32ux8054}

案例:检测 → 裁剪 → 分类。 \textbar{} Stage \textbar{} 输入/输出
\textbar{} 并行策略 \textbar{} 指标采集 \textbar{} \textbar{} -----
\textbar{} -------- \textbar{} -------- \textbar{} -------- \textbar{}
\textbar{} Detector \textbar{} 原始帧 → 框 \textbar{} 批处理+单模型
\textbar{} 时延/框数 \textbar{} \textbar{} Cropper \textbar{} 帧+框 →
Patch 列表 \textbar{} 多线程 CPU \textbar{} 单 Patch 平均耗时 \textbar{}
\textbar{} Classifier \textbar{} Patch → TopK 类别 \textbar{} 合批
(N≤32) \textbar{} FPS/准确率 \textbar{} \#\#\# 优化要点 1. Buffer
池:重用图像与 Patch 内存,避免频繁 malloc。 2. 批量裁剪:收集一定数量
Patch 再统一预处理。 3. 超时控制:某帧超过阈值后续结果丢弃,保持实时性。
4. 滑窗统计:最近 60s FPS、平均队列深度。

\section{工程目录与脚本标准}\label{ux5de5ux7a0bux76eeux5f55ux4e0eux811aux672cux6807ux51c6}

\begin{lstlisting}
deploy/
  classify/
    export.py
    atc.sh
    config.yaml
  detect/
    export.py
    atc.sh
  ocr/
    export_det.py
    export_rec.py
    atc_det.sh
    atc_rec.sh
runtime/
  core/acl_session.cpp
  preprocess/
  postprocess/
  pipelines/
tests/
  data/
  benchmark/
docs/
  model_cards/
\end{lstlisting}

版本归档要求: \textbar{} 产物 \textbar{} 检查点 \textbar{} \textbar{}
---- \textbar{} ------- \textbar{} \textbar{} *.om \textbar{} 与 atc.log
hash 对应 \textbar{} \textbar{} signature.json \textbar{}
与运行时动态查询一致 \textbar{} \textbar{} metrics.json \textbar{}
包含时间戳/commit\_sha \textbar{} \textbar{} model\_card.md \textbar{}
模型来源/License/精度 \textbar{}

\section{性能基线方法与统计置信}\label{ux6027ux80fdux57faux7ebfux65b9ux6cd5ux4e0eux7edfux8ba1ux7f6eux4fe1}

推荐: 1. Warmup 5\textasciitilde10 次; 2. 收集 ≥200 次稳定样本; 3.
计算 avg, p50, p95, p99; 4.
计算置信区间:\passthrough{\lstinline!mean ± 1.96 * (std/sqrt(n))!}; 5.
记录环境:芯片序列号/温度区间/电源模式/版本矩阵。 差异判定:新版本 avg
降低 \textgreater5\% 或 p95 上升 \textgreater8\% 触发报警分析。

\section{常见问题诊断深度版}\label{ux5e38ux89c1ux95eeux9898ux8bcaux65adux6df1ux5ea6ux7248}

\begin{longtable}[]{@{}llll@{}}
\toprule\noalign{}
问题 & 表现 & 诊断步骤 & 修复 \\
\midrule\noalign{}
\endhead
\bottomrule\noalign{}
\endlastfoot
输出全 0 & logits 恒定 & Dump 中间 tensor & 校验预处理/权重损坏 \\
检测框偏移 & 坐标不准 & 可视化缩放/Pad 参数 & 修正 letterbox 逆变换 \\
OCR 乱码 & 字符错位 & 对比 index→char 映射 & 统一 vocab \& 排序 \\
BERT 性能差 & tokens/s 低 & 分析长度分布 & 分桶/裁剪长度 \\
Pipeline 堵塞 & 帧延迟增长 & 监控队列深度 & 降帧/扩线程池 \\
内存持续上涨 & long run OOM & 内存快照/工具 & 释放缓存/池化 \\
\end{longtable}

\section{章节小结}\label{ux7ae0ux8282ux5c0fux7ed3}

本章提供四类典型任务部署详解,并抽象了跨任务可复用的脚手架与性能度量方法。重点在于``输入契约统一''、``阶段解耦''、``可观察性内建''。掌握后可进入性能与算子优化专题。

\section{实践任务}\label{ux5b9eux8df5ux4efbux52a1}

\begin{enumerate}
\def\labelenumi{\arabic{enumi}.}
\tightlist
\item
  部署 ResNet50:输出 Top5 及概率、提交 metrics.json。
\item
  部署 YOLOv5s:5 张测试图片生成可视化结果(描述框坐标与类别统计)。
\item
  构建 OCR 双模型流水线:统计单帧平均文本块数 + 平均识别耗时。
\item
  BERT:对 3 组长度(32/64/128) 测 tokens/s 与时延差异,生成对比表。
\item
  Pipeline 检测→分类:实现批裁剪 + Buffer
  池,比较优化前后平均时延下降百分比。
\end{enumerate}


% 第五章
\chapter{性能与算子优化初阶}
\section{章节总览}\label{ux7ae0ux8282ux603bux89c8}

本章聚焦``定位 → 解释 → 改善''闭环:从性能分析模型、Profiling
工具、瓶颈模式分类、布局与精度策略、内存与并行调度、到自定义算子开发与验证标准,提供工程可落地方法。目标是让读者具备:A)
定量证明问题;B) 选择低风险优化策略;C) 保证功能与性能回归一致性。

\section{性能拆解与衡量框架}\label{ux6027ux80fdux62c6ux89e3ux4e0eux8861ux91cfux6846ux67b6}

总时延公式:\passthrough{\lstinline!T\_total = T\_pre + T\_h2d + T\_infer + T\_d2h + T\_post + T\_idle!}。
吞吐上限受制于 \passthrough{\lstinline!max(T\_component)!};需收集: -
平均/分位数 (p50/p95); - 波动系数 CV=std/mean(\textgreater0.15
需进一步剖析); - 稳定性:长跑 1h 是否存在漂移 (内存泄漏或热降频)。
对比优化前后必须保留固定随机种子和数据集,消除噪声。

\section{Profiling
工具与时间线解读}\label{profiling-ux5de5ux5177ux4e0eux65f6ux95f4ux7ebfux89e3ux8bfb}

关键观测元素: \textbar{} 轨迹 \textbar{} 意义 \textbar{} 异常信号
\textbar{} \textbar{} ---- \textbar{} ---- \textbar{} --------
\textbar{} \textbar{} Stream Timeline \textbar{} 内核调度顺序 \textbar{}
大量空洞 gap \textbar{} \textbar{} MemCopy \textbar{} H2D/D2H 开销
\textbar{} 频繁小块拷贝 \textbar{} \textbar{} Task Kernel \textbar{}
算子执行 \textbar{} 个别算子异常拖长 \textbar{} \textbar{} Sync/Wait
\textbar{} Host 等待 \textbar{} Wait 占比高 \textbar{}

使用策略: 1. 先全量 Profile → 定位热点范围; 2. 二次局部
Profile(过滤特定算子类型); 3. 导出 JSON → 自动解析器归档:算子耗时
TOPK,Copy 占比,Idle 时间。

\section{瓶颈模式与处置策略矩阵}\label{ux74f6ux9888ux6a21ux5f0fux4e0eux5904ux7f6eux7b56ux7565ux77e9ux9635}

\begin{longtable}[]{@{}
  >{\raggedright\arraybackslash}p{(\linewidth - 8\tabcolsep) * \real{0.1176}}
  >{\raggedright\arraybackslash}p{(\linewidth - 8\tabcolsep) * \real{0.2353}}
  >{\raggedright\arraybackslash}p{(\linewidth - 8\tabcolsep) * \real{0.2353}}
  >{\raggedright\arraybackslash}p{(\linewidth - 8\tabcolsep) * \real{0.2941}}
  >{\raggedright\arraybackslash}p{(\linewidth - 8\tabcolsep) * \real{0.1176}}@{}}
\toprule\noalign{}
\begin{minipage}[b]{\linewidth}\raggedright
模式
\end{minipage} & \begin{minipage}[b]{\linewidth}\raggedright
识别特征
\end{minipage} & \begin{minipage}[b]{\linewidth}\raggedright
定量指标
\end{minipage} & \begin{minipage}[b]{\linewidth}\raggedright
处置优先级
\end{minipage} & \begin{minipage}[b]{\linewidth}\raggedright
策略
\end{minipage} \\
\midrule\noalign{}
\endhead
\bottomrule\noalign{}
\endlastfoot
调度空洞 & Timeline gap 多 & Idle \textgreater{} 10\% & 高 & 合并小算子
/ 预加载数据 \\
访存受限 & 算子耗时与内存带宽正相关 & 算子内核利用率低 & 中 & Layout
变换 / Tile 分块 \\
H2D 瓶颈 & MemCopy 比例高 & H2D\textgreater20\% & 高 &
合并/异步/Pin/AIPP 下沉 \\
后处理拖慢 & Post\textgreater25\% & NMS/Decode 长 & 中 & 并行化 / Device
化 \\
量化退化 & INT8 未获收益 & 时延差\textless10\% & 低 &
重新校准/混合精度 \\
单 Stream 阻塞 & 单流串行 & Stream=1 & 中 & 多流/流水线 \\
\end{longtable}

优先处理``结构性收益''\textgreater``微优化'',避免局部手工 hack
影响可维护性。

\section{Layout /
内存访问优化}\label{layout-ux5185ux5b58ux8bbfux95eeux4f18ux5316}

常见格式:NCHW(框架常用)、NHWC(部分算子优化)、NC1HWC0(硬件友好对齐),转换策略:在数据首次落地时转换一次;若前后模型不同布局,以中间标准布局连接,减少重复重排。
对齐:通道/宽高按 16/32 边界对齐可提升访存一致性;小通道 (\textless16)
可考虑 \passthrough{\lstinline!--enable\_small\_channel!}
以加载优化内核。 缓存复用:多模型共享中间 Buffer(需尺寸与 dtype
一致),通过分配表管理生命周期。

\section{精度与性能的层级折衷}\label{ux7cbeux5ea6ux4e0eux6027ux80fdux7684ux5c42ux7ea7ux6298ux8877}

\begin{longtable}[]{@{}llll@{}}
\toprule\noalign{}
精度层级 & 描述 & 性能收益 & 风险 \\
\midrule\noalign{}
\endhead
\bottomrule\noalign{}
\endlastfoot
FP32 & 基准 & - & 内存带宽/算力高 \\
FP16 & 半精度 & 1.2\textasciitilde1.6x & 累积误差 \\
INT8 对称 & 量化整型 & 1.5\textasciitilde2.2x & 量化噪声 \\
混合精度 & 局部高精度 & 中等 & 实现复杂 \\
\end{longtable}

量化流程要点: 1. 收集代表性校准集(覆盖光照/尺度/类别分布); 2.
校准统计(MinMax / KL); 3. 评估 Top1/Top5
差异、关键指标差异(mAP/F1)。 误差定位:Dump 中间张量(FP32 vs INT8)→
层级误差分布 → 定位失真层(常见:激活饱和/尺度不均衡)。

\section{内存管理专题}\label{ux5185ux5b58ux7ba1ux7406ux4e13ux9898}

策略: 1. 长期 Buffer:模型 I/O、常量 Workspace; 2. 短期 Buffer:Batch
临时中间; 3. 建立内存池(按 size class 分类
1KB/4KB/16KB/64KB/大块),分配 → 归还; 4. 避免频繁
\passthrough{\lstinline!aclrtMalloc/Free!}:使用池化接口封装; 5.
监控:每 60s 记录一次池使用率与系统剩余内存,突增后回收未引用对象; 6.
大对象对齐:按 512B/4KB 对齐减少碎片。

\section{并行与流水线}\label{ux5e76ux884cux4e0eux6d41ux6c34ux7ebf}

多 Stream:将独立算子或多模型分离到不同 Stream 并行调度;注意 Host
侧同步点过多会抵消收益。Pipeline:Pre → Infer → Post 分线程队列,目标是
In-Flight 帧数达到平衡(过多增加延迟,过少利用率低)。
自适应调度:定期评估每阶段平均耗时,动态调整线程池大小(PID 控制思想)。

\section{自定义算子开发与评估}\label{ux81eaux5b9aux4e49ux7b97ux5b50ux5f00ux53d1ux4e0eux8bc4ux4f30}

决策条件: \textbar{} 条件 \textbar{} 必须满足至少一项 \textbar{}
\textbar{} ---- \textbar{} ---------------- \textbar{} \textbar{}
复合算子频繁出现 \textbar{} 合并降低访存 \textbar{} \textbar{}
内置实现回退 Host \textbar{} 存在高额拷贝 \textbar{} \textbar{}
内核模式不适配输入规模 \textbar{} 小尺寸性能差 \textbar{}

流程:需求分析 → JSON 定义(\passthrough{\lstinline!op\_type!}, attr,
inputs/outputs) → Kernel C++ 模板 (向量化 / Tile) → 编译注册 → ATC 识别
→ 功能单测(随机张量对比)→ 性能对比(3 次 Warmup + 50 次统计)。
评估表: \textbar{} 版本 \textbar{} 输入规模 \textbar{} 平均耗时(us)
\textbar{} P95(us) \textbar{} 访存次数 \textbar{} 速度提升 \textbar{}
备注 \textbar{} \textbar{} ---- \textbar{} -------- \textbar{}
----------- \textbar{} ------- \textbar{} -------- \textbar{} --------
\textbar{} ---- \textbar{}

\section{优化案例:Add + ReLU
融合}\label{ux4f18ux5316ux6848ux4f8badd-relu-ux878dux5408}

原始:Add → ReLU 两个算子各自读写内存; 融合:单 Kernel 计算
\passthrough{\lstinline!out = relu(a+b)!}:减少一次读写;
收益估算:内存带宽主导场景中延迟≈(T\_add + T\_relu - 重叠),实际提升
10\textasciitilde25\%。 验证:随机输入 100 次 → 检查数值一致(允许 1e-6
FP16 差异)→ Benchmark 对比。

\section{性能报告与回归模板}\label{ux6027ux80fdux62a5ux544aux4e0eux56deux5f52ux6a21ux677f}

\begin{lstlisting}
{
    "commit": "<git-sha>",
    "model": "resnet50_fp16",
    "batch": 1,
    "avg_latency_ms": 5.87,
    "p95_latency_ms": 6.24,
    "throughput_fps": 170.3,
    "h2d_ms_ratio": 0.11,
    "post_ms_ratio": 0.05,
    "memory_peak_mb": 486,
    "temperature_c_range": "54-58",
    "profiling_date": "2025-09-04T10:21:00Z"
}
\end{lstlisting}

自动化:CI 中若 \passthrough{\lstinline!avg\_latency\_ms!} 高于基线 5\%
→ 标红注释。

\section{章节小结}\label{ux7ae0ux8282ux5c0fux7ed3}

性能优化不等于盲调:应以数据驱动 +
分层定位为前提,先解决架构级与内存/拷贝问题,再考虑算子级微调与自定义算子开发。量化收益需伴随精度风险评估,内存与并行策略需要可观测支撑。

\section{实践任务}\label{ux5b9eux8df5ux4efbux52a1}

\begin{enumerate}
\def\labelenumi{\arabic{enumi}.}
\tightlist
\item
  对一个部署模型收集 Profiling JSON,输出前 5 算子耗时与占比表。
\item
  实现 H2D 合并:将 3 个连续小拷贝合并为单次,比较平均时延改善。
\item
  尝试 INT8 量化:输出精度与性能对比(Top1/Latency/FPS)。
\item
  编写一个 Add+ReLU 融合算子伪代码 + 预期性能提升估算。
\item
  生成基线性能报告,并设定 CI 回归阈值策略文本说明。
\end{enumerate}

\section{昇腾310B自定义算子开发全流程}\label{ux6607ux817e310bux81eaux5b9aux4e49ux7b97ux5b50ux5f00ux53d1ux5168ux6d41ux7a0b}

本节面向 Ascend 310B
推理场景,给出``什么时候需要自定义算子、用什么方法开发、如何编译注册、怎样验证与上线''的系统指引。读完后,你应能独立完成一个简单自定义算子的端到端落地。

\subsection{开发概述}\label{ux5f00ux53d1ux6982ux8ff0}

\begin{itemize}
\tightlist
\item
  目标:当模型中存在``内置算子不支持/性能欠佳/需要业务特化融合''的场景,通过自定义算子(Custom
  Op)补齐功能或获得确定性性能收益。
\item
  实现形态:

  \begin{itemize}
  \tightlist
  \item
    AI Core(TBE/TE/TIK,运行于 NPU
    核心,适合数值密集型向量/矩阵计算)。
  \item
    AICPU(C++/CPU 实现,在 Host/AICPU
    执行,适合控制流或少量数据处理,注意 H2D/D2H 开销)。
  \end{itemize}
\item
  产物:算子描述(op info/proto)、算子实现(AI Core: Python
  实现并编译为内核;AICPU: C++ so)、注册与打包(放入 OPP 路径),以及
  ATC 与运行时可识别的元数据。
\item
  适配 310B:选择
  \passthrough{\lstinline!soc\_version=Ascend310B!},优先 FP16
  数据通路;对齐 NC1HWC0 等硬件友好布局;小通道/小尺寸注意 tile 策略。
\end{itemize}

\subsection{开发的理论基础}\label{ux5f00ux53d1ux7684ux7406ux8bbaux57faux7840}

\begin{enumerate}
\def\labelenumi{\arabic{enumi}.}
\tightlist
\item
  硬件/内存模型(简要):

  \begin{itemize}
  \tightlist
  \item
    GM(Global Memory):大容量全局显存,带宽高、时延高;
  \item
    UB(Unified Buffer):片上高速缓冲,容量有限,需 tile 分块搬运;
  \item
    Vector/Scalar 单元:提供 vadd/vmul/vmax
    等向量指令,需保证数据对齐(通常以 16/32 对齐)。
  \item
    DMA:GM 与 UB 之间的数据搬运,批量大块优于频繁小块。
  \end{itemize}
\item
  计算表达与调度:

  \begin{itemize}
  \tightlist
  \item
    TE(Tensor Expression):描述计算公式与算子图(compute);
  \item
    Schedule:描述分块(tilling)、并行、缓存、向量化等执行计划;
  \item
    TIK DSL:更接近硬件指令级的编程接口,适合精细控制。
  \end{itemize}
\item
  算子契约(Operator Contract):

  \begin{itemize}
  \tightlist
  \item
    输入/输出张量的 shape、dtype、format(如
    NCHW/NC1HWC0)、属性(attr);
  \item
    广播/对齐规则、边界行为(溢出/饱和/舍入)、精度策略(FP16/FP32
    混合)。
  \end{itemize}
\item
  形状推断与动态 shape:

  \begin{itemize}
  \tightlist
  \item
    ATC 需要根据 op 描述完成 shape infer;
  \item
    动态尺寸需在实现中处理 tile 策略切换并保证 UB 不溢出。
  \end{itemize}
\end{enumerate}

\subsection{开发流程(AI Core
为例)}\label{ux5f00ux53d1ux6d41ux7a0bai-core-ux4e3aux4f8b}

以下流程以一个``Add+ReLU 融合''示例说明,读者可据此扩展到实际业务算子。

\begin{enumerate}
\def\labelenumi{\arabic{enumi})}
\tightlist
\item
  环境准备
\end{enumerate}

\begin{itemize}
\tightlist
\item
  确保 CANN/Toolkit 已安装,能使用
  \passthrough{\lstinline!atc!}、Profiling 等工具;
\item
  设置环境变量:

  \begin{itemize}
  \tightlist
  \item
    \passthrough{\lstinline!ASCEND\_INSTALL\_PATH!} 指向 Toolkit 根;
  \item
    \passthrough{\lstinline!ASCEND\_OPP\_PATH!} 指向 OPP 包路径(custom
    算子将被放置于此);
  \item
    \passthrough{\lstinline!soc\_version=Ascend310B!}(ATC/编译时指定)。
  \end{itemize}
\end{itemize}

\begin{enumerate}
\def\labelenumi{\arabic{enumi})}
\setcounter{enumi}{1}
\tightlist
\item
  定义算子信息(op info/proto)
\end{enumerate}

\begin{itemize}
\tightlist
\item
  指定:\passthrough{\lstinline!op\_type!}、inputs/outputs
  名称、dtype/format 组合、属性列表、融合类型等;
\item
  作用:

  \begin{itemize}
  \tightlist
  \item
    供 ATC 做图解析、形状推断与算子选择;
  \item
    供运行时校验输入输出与 kernel 适配。
  \end{itemize}
\end{itemize}

\begin{enumerate}
\def\labelenumi{\arabic{enumi})}
\setcounter{enumi}{2}
\tightlist
\item
  编写算子实现(TE/TBE)
\end{enumerate}

\begin{itemize}
\item
  计算表达: ```python \# 伪代码:y = relu(x1 + x2) import te.lang.cce
  as tbe from te import tvm

  def add\_relu\_compute(x1, x2): y = tbe.vadd(x1, x2) z = tbe.vmaxs(y,
  tvm.const(0.0, x1.dtype)) return z ```
\item
  调度策略(示例要点):

  \begin{itemize}
  \tightlist
  \item
    选择合适的 tile 以满足 UB 容量;
  \item
    将连续内存访问向量化,减少非对齐访问;
  \item
    尽量合并搬运,减少 GM\textless-\textgreater UB 往返;
  \item
    小尺寸场景避免过度拆分导致调度开销占比过高。
  \end{itemize}
\end{itemize}

\begin{enumerate}
\def\labelenumi{\arabic{enumi})}
\setcounter{enumi}{3}
\tightlist
\item
  编译与注册
\end{enumerate}

\begin{itemize}
\tightlist
\item
  使用官方提供的 TBE
  编译入口生成内核与元数据(具体命令因版本而异,遵循已安装 Toolkit
  的说明);
\item
  将生成的实现文件/元数据放入
  \passthrough{\lstinline!ASCEND\_OPP\_PATH!} 下的 custom 目录(如
  \passthrough{\lstinline!op\_impl/custom/ai\_core/tbe!}、\passthrough{\lstinline!op\_proto/custom!})。
\end{itemize}

\begin{enumerate}
\def\labelenumi{\arabic{enumi})}
\setcounter{enumi}{4}
\tightlist
\item
  与 ATC 集成
\end{enumerate}

\begin{itemize}
\tightlist
\item
  在模型转换时指定 \passthrough{\lstinline!--soc\_version=Ascend310B!};
\item
  确保 ATC 能从 \passthrough{\lstinline!ASCEND\_OPP\_PATH!} 读取到你的
  op 描述与实现信息;
\item
  若需要限制实现选择,可使用
  \passthrough{\lstinline!--op\_select\_implmode!} 配合算子实现指示。
\end{itemize}

\begin{enumerate}
\def\labelenumi{\arabic{enumi})}
\setcounter{enumi}{5}
\tightlist
\item
  运行时部署与加载
\end{enumerate}

\begin{itemize}
\tightlist
\item
  运行环境中需要包含同样的 OPP 目录(含 custom 实现);
\item
  应用进程启动时配置环境变量,使 Runtime 能定位自定义算子实现;
\item
  按常规 ACL 流程加载 OM 并执行推理。
\end{itemize}

\begin{enumerate}
\def\labelenumi{\arabic{enumi})}
\setcounter{enumi}{6}
\tightlist
\item
  验证与度量
\end{enumerate}

\begin{itemize}
\tightlist
\item
  功能正确性:与参考实现(NumPy/ONNXRuntime)对齐,随机多组张量比较(均值绝对误差、相对误差、边界样本)。
\item
  性能评估:Warmup 3 次 + 采样 50 次,输出
  avg/p95/FPS;对比内置算子或未融合版本;
\item
  资源占用:Profiling 检查 MemCopy 占比、Kernel 占比、Idle;
\item
  兼容性:不同 shape/dtype/format 组合覆盖测试。
\end{itemize}

\begin{enumerate}
\def\labelenumi{\arabic{enumi})}
\setcounter{enumi}{7}
\tightlist
\item
  文档与产物归档
\end{enumerate}

\begin{itemize}
\tightlist
\item
  输出
  \passthrough{\lstinline!op\_contract.yaml!}(IO/Attr/格式/边界规则);
\item
  输出
  \passthrough{\lstinline!benchmark.json!}(avg/p95、对比基线、硬件/版本信息);
\item
  产物目录:\passthrough{\lstinline!op\_pkg/<op\_type>/<version>/\{op\_proto, op\_impl, tests, docs\}!}。
\end{itemize}

\subsection{AICPU 路线(可选)}\label{aicpu-ux8defux7ebfux53efux9009}

\begin{itemize}
\tightlist
\item
  适用:控制流、轻量数据处理或暂不需在 NPU 上运行的功能性算子;
\item
  实现:C/C++ 编写,遵循 AICPU 接口,注册到相应目录生成动态库;
\item
  注意:Host 执行会引入 H2D/D2H;若在性能关键路径,优先 AI Core 版本。
\end{itemize}

\subsection{常见问题与排错}\label{ux5e38ux89c1ux95eeux9898ux4e0eux6392ux9519}

\begin{itemize}
\tightlist
\item
  ATC 提示 Unsupported Op:检查 op info 是否被正确放置且生效;确认
  \passthrough{\lstinline!soc\_version!} 与路径;
\item
  运行时 Fallback:确认实现 dtype/format 与模型一致;必要时扩充
  \passthrough{\lstinline!dtype\_format!} 组合;
\item
  性能未达预期:增大 tile、减少小块 DMA、合并计算、检查是否出现额外
  layout 转换;
\item
  精度差异:检查饱和/舍入策略、对齐与广播规则、数据范围(FP16 溢出)。
\end{itemize}

\subsection{本章小结}\label{ux672cux7ae0ux5c0fux7ed3}

自定义算子是 310B
场景下``功能补齐与性能确定性''的关键手段。核心抓手包括:明确契约(IO/格式/属性)、用
TE/TIK 描述计算并设计合理调度、放在 OPP 中正确注册、生效于 ATC
与运行时、用可度量的基线进行功能/性能回归。建议从``融合与复合算子''起步,优先选择计算密集、访存友好的目标,循序渐进积累模板与脚手架,以降低维护成本。


% 第六章
\chapter{系统工程与高可用部署}
\input{chapters/chapter6.tex}

% 第七章
\chapter{项目实战方法论与交付模板}
\section{章节总览}\label{ux7ae0ux8282ux603bux89c8}

本章建立从需求澄清→指标体系→评测集→迭代节奏→资产沉淀→交付与回归的一套闭环方法论,让技术决策基于指标与风险敞口,而非经验臆测。核心理念:可量化、可比较、可复用、可追溯。

\section{需求澄清 Canvas}\label{ux9700ux6c42ux6f84ux6e05-canvas}

\begin{longtable}[]{@{}
  >{\raggedright\arraybackslash}p{(\linewidth - 6\tabcolsep) * \real{0.2000}}
  >{\raggedright\arraybackslash}p{(\linewidth - 6\tabcolsep) * \real{0.2000}}
  >{\raggedright\arraybackslash}p{(\linewidth - 6\tabcolsep) * \real{0.4000}}
  >{\raggedright\arraybackslash}p{(\linewidth - 6\tabcolsep) * \real{0.2000}}@{}}
\toprule\noalign{}
\begin{minipage}[b]{\linewidth}\raggedright
维度
\end{minipage} & \begin{minipage}[b]{\linewidth}\raggedright
要素
\end{minipage} & \begin{minipage}[b]{\linewidth}\raggedright
问题提示
\end{minipage} & \begin{minipage}[b]{\linewidth}\raggedright
示例
\end{minipage} \\
\midrule\noalign{}
\endhead
\bottomrule\noalign{}
\endlastfoot
场景 & 输入源/运行环境 & 摄像头?批处理? & 室内 1080p30 低光 \\
目标 & 功能/业务价值 & 用户希望看到什么结果? & 实时检测 + 分析 \\
指标 & Latency/FPS/精度 & 哪些分位数重要? & \textless80ms / ≥25FPS /
mAP≥0.6 \\
约束 & 能耗/内存/带宽 & 上限是多少? & 功耗≤15W 内存≤3GB \\
风险 & 数据/硬件/算法 & 失败模式有哪些? & 低光/遮挡/抖动 \\
合规 & 隐私/许可 & 是否需要脱敏? & 仅上传事件元数据 \\
成本 & 硬件/云 & ROI 衡量? & 10 台板卡预算 \\
\end{longtable}

输出:\passthrough{\lstinline!requirement.yaml!}(版本化),后续所有评审基于此文档。

\section{指标分层与优先级}\label{ux6307ux6807ux5206ux5c42ux4e0eux4f18ux5148ux7ea7}

\begin{longtable}[]{@{}lllll@{}}
\toprule\noalign{}
层级 & 类别 & 指标 & 说明 & 失败后果 \\
\midrule\noalign{}
\endhead
\bottomrule\noalign{}
\endlastfoot
SLO A & 体验 & p95 延迟 & 端到端 & 体验差/丢帧 \\
SLO A & 性能 & FPS & 稳态吞吐 & 处理拥堵 \\
SLO B & 质量 & mAP/F1/Top1 & 任务正确性 & 无法满足业务 \\
SLO B & 稳定 & Crash/小时 & 可靠性 & 运维成本高 \\
SLO C & 资源 & 内存峰值/功耗 & 成本约束 & 设备异常/降频 \\
SLO C & 带宽 & 上行 kbps & 成本/合规 & 费用/拥塞 \\
\end{longtable}

优先级:先保障 A(体验+功能可用),再稳定 B(质量/稳定),最后优化
C(资源效率)。

\section{Baseline
策略与控制变量法}\label{baseline-ux7b56ux7565ux4e0eux63a7ux5236ux53d8ux91cfux6cd5}

Baseline 目标:建立 ``最小改动可运行'' 标尺。原则:

\begin{enumerate}
\def\labelenumi{\arabic{enumi}.}
\tightlist
\item
  不提前做微优化;
\item
  记录所有关键参数:模型版本、输入尺寸、预处理策略、硬件温度范围;
\item
  一次仅改变单个变量(batch、精度、线程数)。
  基线存档:\passthrough{\lstinline!baseline/<date>-<commit>/metrics.json!};对比脚本生成差异报告。
\end{enumerate}

\section{评测集设计原则}\label{ux8bc4ux6d4bux96c6ux8bbeux8ba1ux539fux5219}

\begin{longtable}[]{@{}ll@{}}
\toprule\noalign{}
原则 & 内容 \\
\midrule\noalign{}
\endhead
\bottomrule\noalign{}
\endlastfoot
代表性 & 涵盖主流场景/光照/角度 \\
覆盖边界 & 极端尺寸、模糊、遮挡 \\
可再现 & 文件命名规范 + 固定清单 \\
可扩展 & 新增样本不破坏旧索引 \\
标注一致 & 标注工具/规范/审校流程 \\
\end{longtable}

目录示例:

\begin{lstlisting}
dataset_eval/
  images/
    day/*.jpg
    night/*.jpg
    occlusion/*.jpg
  annotations/
    instances_train.json
    instances_val.json
  meta/
    README.md
    version.txt
\end{lstlisting}

提供 Hash 列表,防止样本被替换而影响回归可信度。

\section{迭代计划与看板}\label{ux8fedux4ee3ux8ba1ux5212ux4e0eux770bux677f}

四阶段:

\begin{longtable}[]{@{}llll@{}}
\toprule\noalign{}
Sprint & 目标 & 核心产出 & 风险控制 \\
\midrule\noalign{}
\endhead
\bottomrule\noalign{}
\endlastfoot
0 & 环境/基线 & baseline metrics & 依赖清单齐全 \\
1 & 精度与功能稳固 & 精度报告 & 数据问题快速反馈 \\
2 & 性能与稳定 & 性能对比表/监控上线 & Watchdog 验证 \\
3 & 工程交付包装 & Release Notes/脚本 & 灰度计划制定 \\
\end{longtable}

看板列:Backlog → Doing → Review → Bench → Done;性能/精度任务需进入
Bench 列执行对比脚本通过后才可 Done。

\section{资产沉淀文档体系}\label{ux8d44ux4ea7ux6c89ux6dc0ux6587ux6863ux4f53ux7cfb}

\begin{longtable}[]{@{}lll@{}}
\toprule\noalign{}
文档 & 内容 & 更新频率 \\
\midrule\noalign{}
\endhead
\bottomrule\noalign{}
\endlastfoot
README & 快速启动 & 版本变化时 \\
ARCHITECTURE & 架构图/模块说明 & 结构调整 \\
MODEL\_CARD & 模型来源/许可/精度/限制 & 模型更新 \\
EVAL\_REPORT & 数据与评测方法/指标 & 每次发布 \\
PERF\_REPORT & 基线/优化对比 & 优化后 \\
CHANGELOG & 可见版本差异 & 每次版本 \\
RISK\_LOG & 已知风险列表 & 动态 \\
\end{longtable}

MODEL\_CARD 需包含:数据来源、训练超参摘要、输入契约、已知局限、许可(如
Apache-2.0)、安全与偏见说明(若涉及识别敏感属性声明避免用途)。

\section{交付目录与不可变产物}\label{ux4ea4ux4ed8ux76eeux5f55ux4e0eux4e0dux53efux53d8ux4ea7ux7269}

\begin{lstlisting}
release/
  v1.0/
    manifest.json     # 产物 hash / 版本矩阵
    models/
      detect.om
      classify.om
      signature.json
    scripts/
      run.ps1
      run.sh
      watchdog.sh
    configs/
      default.yaml
    docs/
      model_card_detect.md
      model_card_classify.md
      QUICKSTART.md
    reports/
      perf.json
      accuracy.json
\end{lstlisting}

manifest.json
字段:\passthrough{\lstinline!\{version, commit, build\_time, model\_hashes, dependencies\}!}。

\section{上线前综合
Checklist}\label{ux4e0aux7ebfux524dux7efcux5408-checklist}

\begin{longtable}[]{@{}lll@{}}
\toprule\noalign{}
类别 & 检查项 & 通过标准 \\
\midrule\noalign{}
\endhead
\bottomrule\noalign{}
\endlastfoot
功能 & 核心用例 100\% & 自动化用例通过 \\
性能 & p95 \textless{} 目标 +5\% & 连续 30min 稳定 \\
精度 & mAP/Top1 回归差 \textless{} 阈值 & 与基线对比 \\
资源 & 内存峰值 \textless{} 75\% & 1h 稳态无泄漏 \\
稳定 & Crash=0, 重启=0 & 守护日志清洁 \\
安全 & 日志无敏感泄露 & 关键字段脱敏 \\
配置 & 签名校验一致 & Hash 匹配 \\
回滚 & 验证上一版本可用 & 切换 \textless{} 30s \\
\end{longtable}

\section{验收、回归与漂移监测}\label{ux9a8cux6536ux56deux5f52ux4e0eux6f02ux79fbux76d1ux6d4b}

交付后 7 天加密监控:记录时延、精度漂移(采样对比模型输出变化)。
漂移检测:相同输入集合(Shadow Set)每天抽样跑一次 → 统计 logits KL
散度/TopK 变化率,高于阈值(如 KL \textgreater{}
0.05)触发报警(潜在数据分布变化或模型文件损坏)。
回归集版本化:\passthrough{\lstinline!eval\_set\_vX!};若需替换样本 →
新增版本,不覆盖旧数据。

\section{风险管理与决策日志}\label{ux98ceux9669ux7ba1ux7406ux4e0eux51b3ux7b56ux65e5ux5fd7}

风险登记表:\passthrough{\lstinline!risk\_log.md!}
每条包含:ID、描述、影响、概率、缓解、当前状态。决策日志(Decision
Record,
ADR):记录架构/模型/精度策略选择及备选方案放弃理由,以便新成员快速建立上下文。

\section{章节小结}\label{ux7ae0ux8282ux5c0fux7ed3}

方法论的核心不是流程文档堆砌,而是``指标驱动 + 资产沉淀 +
可回滚''三支柱。通过契约化需求、标准化
Baseline、规范化评测与回归体系,使团队协作更高效、风险暴露更透明、交付结果更可信。

\section{实践任务}\label{ux5b9eux8df5ux4efbux52a1}

\begin{enumerate}
\def\labelenumi{\arabic{enumi}.}
\tightlist
\item
  输出 \passthrough{\lstinline!requirement.yaml!}(含指标与约束)。
\item
  构建 30 张代表图像的 mini 评测集并附 Hash 列表。
\item
  生成 baseline \passthrough{\lstinline!metrics.json!} 与后一次优化对比
  diff 报告。
\item
  制作一个 MODEL\_CARD 模板并填写一个模型示例。
\item
  编写上线 Checklist 并模拟一项未通过情形与处置方案。
\end{enumerate}


% 第八章
\chapter{合实战案例集}
\section{章节总览}\label{ux7ae0ux8282ux603bux89c8}

本章通过九个真实应用场景串联前面章节的知识:模型选择、转换、部署、性能与稳定性验证、迭代优化。所有案例采用统一模板,支持快速复制与对比评估。强调``结构化指标
+ 自动化脚本 + 可视化反馈''。

\section{案例统一模板(标准化规范)}\label{ux6848ux4f8bux7edfux4e00ux6a21ux677fux6807ux51c6ux5316ux89c4ux8303}

\begin{longtable}[]{@{}lll@{}}
\toprule\noalign{}
区块 & 内容要点 & 产出文件 \\
\midrule\noalign{}
\endhead
\bottomrule\noalign{}
\endlastfoot
场景描述 & 背景/输入/目标 & README.md\#scene \\
指标目标 & 延迟/FPS/精度/资源 & requirement.yaml \\
模型选择 & 候选对比 + 取舍 & model\_card*.md \\
数据准备 & 采集/标注/增强 & data\_prep.md \\
转换部署 & 导出→ATC 参数 & atc.sh / export.py \\
运行脚本 & 启动/参数/日志路径 & run.sh / run.ps1 \\
性能结果 & metrics.json (基线/优化) & metrics/*.json \\
质量验证 & 精度/漂移检查 & accuracy.json \\
风险改进 & 已知问题/迭代计划 & roadmap.md \\
\end{longtable}

\section{案例目录结构规范}\label{ux6848ux4f8bux76eeux5f55ux7ed3ux6784ux89c4ux8303}

\begin{lstlisting}
experiments/caseX/
  README.md
  requirement.yaml
  models/         # onnx / om / signatures
  scripts/
    export.py
    atc.sh
    run_infer.py
    benchmark.py
  data/           # 样本(或下载指令)
  metrics/
    baseline.json
    optimized.json
  logs/
  eval/
    accuracy.json
    drift.json
  assets/         # 截图/示意图
\end{lstlisting}

\section{例概览与重点}\label{ux4f8bux6982ux89c8ux4e0eux91cdux70b9}

\begin{longtable}[]{@{}lllll@{}}
\toprule\noalign{}
序 & 名称 & 关键技术点 & 指标核心 & 风险要素 \\
\midrule\noalign{}
\endhead
\bottomrule\noalign{}
\endlastfoot
1 & 人脸打卡机 & 人脸检测+比对+活体 & 识别成功率/伪拒率 & 光照/遮挡 \\
2 & 实时跟踪 & 检测+多目标关联 & 跟踪稳定度(IDF1) & 遮挡/抖动 \\
3 & 智能电子琴 & 音频节拍识别+分类 & 识别延迟/准确率 & 噪声/延迟 \\
4 & 掌纹识别 & ROI 提取+特征匹配 & 误识率/拒识率 & 采集姿态 \\
5 & 数据采集仪 & 传感融合+缓存上传 & 数据丢失率 & 网络波动 \\
6 & 智能小车 & 目标检测+路径策略 & 决策延迟 & 传感器同步 \\
7 & 智能相册 & 分类+聚类+去重 & 聚类纯度 & 相似干扰 \\
8 & 手势识别 & 时序建模(TSM) & 手势准确率/FPS & 动作模糊 \\
9 & 聊天机器人 & NLP 推理+缓存 & 响应时延/意图准确 & 语料漂移 \\
\end{longtable}

下列示例详细展开前三个具代表性的模式。

\section{案例
1:人脸打卡机}\label{ux6848ux4f8b-1ux4ebaux8138ux6253ux5361ux673a}

\subsection{场景}\label{ux573aux666f}

摄像头实时输入,人脸检测→关键点对齐→特征提取→特征库比对→授权决策→事件上报。

\subsection{指标}\label{ux6307ux6807}

\begin{longtable}[]{@{}lll@{}}
\toprule\noalign{}
指标 & 目标 & 说明 \\
\midrule\noalign{}
\endhead
\bottomrule\noalign{}
\endlastfoot
平均识别时延 & \textless{} 120ms & 从帧采集到结果 \\
最大 P95 & \textless{} 150ms & 抖动控制 \\
误识率(FAR) & \textless{} 0.001 & 安全性 \\
拒识率(FRR) & \textless{} 0.02 & 体验 \\
\end{longtable}

\subsection{模型链路}\label{ux6a21ux578bux94feux8def}

\begin{enumerate}
\def\labelenumi{\arabic{enumi}.}
\tightlist
\item
  人脸检测 (RetinaFace);
\item
  5 点关键点仿射对齐;
\item
  ArcFace 特征 512D;
\item
  向量归一化 + 余弦相似度;
\item
  阈值自适应(基于滑动窗口均值校正)。
\end{enumerate}

\subsection{性能优化}\label{ux6027ux80fdux4f18ux5316}

\begin{itemize}
\tightlist
\item
  批量特征比对:向量库转矩阵,使用 SIMD/BLAS;
\item
  缓存:最近识别通过用户特征缓存,减少重复比对;
\item
  光照增强:低光阈值触发 Gamma/直方图均衡。
\end{itemize}

\subsection{metrics 示例}\label{metrics-ux793aux4f8b}

\begin{lstlisting}
{
  "avg_latency_ms": 98.4,
  "p95_latency_ms": 121.3,
  "fps": 10.1,
  "face_detect_ms": 42.1,
  "feature_ms": 18.7,
  "match_ms": 5.2,
  "false_accept_rate": 0.0008,
  "false_reject_rate": 0.017
}
\end{lstlisting}

\section{案例 2:实时跟踪(检测 +
关联)}\label{ux6848ux4f8b-2ux5b9eux65f6ux8ddfux8e2aux68c0ux6d4b-ux5173ux8054}

\subsection{流程}\label{ux6d41ux7a0b}

帧采集 → 目标检测 → 外观特征提取 → 卡尔曼预测 → 匈牙利匹配 → 轨迹输出。

\subsection{难点}\label{ux96beux70b9}

遮挡/丢失:轨迹生命周期管理(状态:Tentative → Confirmed → Lost →
Removed)。

\subsection{优化}\label{ux4f18ux5316}

\begin{enumerate}
\def\labelenumi{\arabic{enumi}.}
\tightlist
\item
  检测降频:每 N 帧做一次全检测,中间帧仅跟踪预测;
\item
  多线程:检测与跟踪解耦;
\item
  ReID 模型轻量化(裁剪通道)。
\end{enumerate}

\subsection{评估指标}\label{ux8bc4ux4f30ux6307ux6807}

IDF1、MOTA、FP/FN、IDSW(身份切换)。

\section{案例
3:智能电子琴(音频)}\label{ux6848ux4f8b-3ux667aux80fdux7535ux5b50ux7434ux97f3ux9891}

\subsection{流程}\label{ux6d41ux7a0b-1}

音频采集 16kHz → 窗口分帧 FFT → 频谱/梅尔特征 → 分类模型(音符/节奏)→
校准节拍输出。

\subsection{优化点}\label{ux4f18ux5316ux70b9}

FFT
批处理使用向量库;低延迟滑动窗口;模型输出置信度平滑(指数滑动平均)。

\subsection{指标}\label{ux6307ux6807-1}

节拍延迟 \textless{} 80ms;识别准确率 \textgreater{} 95\%。

\section{结果记录与差异报告}\label{ux7ed3ux679cux8bb0ux5f55ux4e0eux5deeux5f02ux62a5ux544a}

基线与优化版本差异自动生成:

\begin{longtable}[]{@{}lllll@{}}
\toprule\noalign{}
指标 & baseline & optimized & 差异 & 状态 \\
\midrule\noalign{}
\endhead
\bottomrule\noalign{}
\endlastfoot
avg\_latency\_ms & 112.5 & 98.4 & -12.5\% & ✅ \\
p95\_latency\_ms & 140.3 & 121.3 & -13.5\% & ✅ \\
false\_accept\_rate & 0.0012 & 0.0008 & 改善 & ✅ \\
\end{longtable}

\section{自动化与复现保障}\label{ux81eaux52a8ux5316ux4e0eux590dux73b0ux4fddux969c}

\begin{longtable}[]{@{}ll@{}}
\toprule\noalign{}
机制 & 说明 \\
\midrule\noalign{}
\endhead
\bottomrule\noalign{}
\endlastfoot
Hash 校验 & onnx/om/脚本确保未篡改 \\
repeatable seed & 设定随机种子统一实验 \\
benchmark.py & 统一输出 metrics.json \\
drift 检测 & 周期性对比指标偏差 \\
一键脚本 & run.sh + run.ps1 支持跨平台 \\
\end{longtable}

\section{指标可视化建议}\label{ux6307ux6807ux53efux89c6ux5316ux5efaux8bae}

\begin{itemize}
\tightlist
\item
  时间序列:Latency / FPS / 温度。
\item
  箱线图:不同优化阶段的时延分布。
\item
  堆叠条:阶段占比(检测/特征/比对)。
\item
  散点:光照水平 vs 识别准确度。
\end{itemize}

\section{通用问题经验库}\label{ux901aux7528ux95eeux9898ux7ecfux9a8cux5e93}

\begin{longtable}[]{@{}llll@{}}
\toprule\noalign{}
问题 & 案例 & 根因 & 处理 \\
\midrule\noalign{}
\endhead
\bottomrule\noalign{}
\endlastfoot
相机丢帧 & 1/2 & 帧率不稳 & 缓冲+限速 \\
模型加载慢 & 全部 & 冷启动未预热 & 预加载预热10次 \\
OCR 错字 & 新增 & 图像模糊 & 降噪/锐化 \\
跟踪漂移 & 2 & 过度遮挡 & reinit + 短期外观缓存 \\
\end{longtable}

\section{扩展方向}\label{ux6269ux5c55ux65b9ux5411}

\begin{itemize}
\tightlist
\item
  多模态融合(视觉+语音指令)。
\item
  硬件加速协同(NPU + DSP 解码)。
\item
  大模型边缘裁剪(蒸馏 + 量化 + 分层推理)。
\end{itemize}

\section{贡献工作流}\label{ux8d21ux732eux5de5ux4f5cux6d41}

\begin{enumerate}
\def\labelenumi{\arabic{enumi}.}
\tightlist
\item
  Fork → 分支:\passthrough{\lstinline!case/<name>!};
\item
  新建目录遵循模板;
\item
  提交包含:README、metrics、脚本、model\_card;
\item
  CI 自动校验格式与 hash;
\item
  PR 模板填写:动机/数据/指标/风险。
\end{enumerate}

\section{章节小结}\label{ux7ae0ux8282ux5c0fux7ed3}

案例是知识的验证与反哺:通过统一模板与自动化度量,形成可延展的案例库,帮助新模型与新任务快速落地并保障质量。

\section{实践任务}\label{ux5b9eux8df5ux4efbux52a1}

\begin{enumerate}
\def\labelenumi{\arabic{enumi}.}
\tightlist
\item
  搭建 case1 目录,生成 baseline metrics。
\item
  实现 face detection + feature 比对流程,并输出 FAR/FRR。
\item
  将一次优化(裁剪/量化)前后差异写入 diff 表。
\item
  编写 benchmark.py:支持
  \passthrough{\lstinline!--repeat N --output metrics.json!}。
\item
  增加 drift 检测脚本(比较两次 metrics 差异,阈值报警)。
\end{enumerate}


% 第九章
\chapter{附录与工具箱}
\section{章节总览}\label{ux7ae0ux8282ux603bux89c8}

本附录聚焦``查得快、用得稳'':常见报错速查、转换参数模板、性能/质量
Checklist、术语字典、推荐资源与社区贡献规范。可作为日常开发随手翻阅的工具章节。

\section{常见报错速查}\label{ux5e38ux89c1ux62a5ux9519ux901fux67e5}

\begin{longtable}[]{@{}
  >{\raggedright\arraybackslash}p{(\linewidth - 8\tabcolsep) * \real{0.1111}}
  >{\raggedright\arraybackslash}p{(\linewidth - 8\tabcolsep) * \real{0.2222}}
  >{\raggedright\arraybackslash}p{(\linewidth - 8\tabcolsep) * \real{0.2222}}
  >{\raggedright\arraybackslash}p{(\linewidth - 8\tabcolsep) * \real{0.2222}}
  >{\raggedright\arraybackslash}p{(\linewidth - 8\tabcolsep) * \real{0.2222}}@{}}
\toprule\noalign{}
\begin{minipage}[b]{\linewidth}\raggedright
分类
\end{minipage} & \begin{minipage}[b]{\linewidth}\raggedright
报错/现象
\end{minipage} & \begin{minipage}[b]{\linewidth}\raggedright
可能原因
\end{minipage} & \begin{minipage}[b]{\linewidth}\raggedright
排查步骤
\end{minipage} & \begin{minipage}[b]{\linewidth}\raggedright
解决建议
\end{minipage} \\
\midrule\noalign{}
\endhead
\bottomrule\noalign{}
\endlastfoot
ATC & \passthrough{\lstinline!E19001: Op Not Supported!} &
新算子/版本落后 & 确认 CANN 版本 + onnxsim 简化 &
升级/替换结构/自定义算子 \\
ATC & Shape 推断失败 & 动态维度不明确 & 检查
\passthrough{\lstinline!--input\_shape!}/动态参数 &
固定关键维度或提供范围 \\
ACL & \passthrough{\lstinline!aclmdlLoadFromFile failed!} &
权限/路径/模型损坏 & 校验文件 hash/权限 & 修正权限/重新生成 OM \\
Runtime & OOM / alloc 失败 & Batch 或分辨率过大 & 统计输入分布 & 降
batch/分桶/复用内存 \\
运行 & 推理输出 NAN & 数值溢出/量化尺度错误 & Dump 中间 Tensor &
调整量化/保留 FP32 层 \\
性能 & Timeline 大量 gap & Host 阻塞/小算子 & Profiling 分析 &
合并算子/异步预取 \\
性能 & H2D 高占比 \textgreater25\% & 多次小拷贝 & 合并缓冲 & AIPP
下沉/批量化 \\
精度 & Top1 下降 \textgreater1\% & 预处理不匹配 & 对比 ONNX 输出 & 统一
Normalize \& Layout \\
精度 & mAP 不稳定 & 阈值或 NMS 误差 & 调整阈值/比对中间框 & 校准 NMS
公式/尺度 \\
稳定 & 间歇 Crash & 悬空指针/并发访问 & 启用 ASAN/日志回溯 &
修订生命周期/加锁 \\
部署 & 模型加载慢 & 冷启动/IO 慢 & 预热/缓存 & 预加载 + 固态存储 \\
安全 & 日志泄露敏感路径 & 直接 print & grep 审计 & 结构化日志脱敏 \\
\end{longtable}

\section{模型转换参数模板合集}\label{ux6a21ux578bux8f6cux6362ux53c2ux6570ux6a21ux677fux5408ux96c6}

\subsection{分类模型 (ResNet)}\label{ux5206ux7c7bux6a21ux578b-resnet}

\begin{lstlisting}
atc --model=resnet50.onnx \
    --framework=5 \
    --output=resnet50_fp16 \
    --input_format=NCHW \
    --input_shape="input:1,3,224,224" \
    --soc_version=Ascend310B \
    --precision_mode=allow_fp32_to_fp16 \
    --log=info
\end{lstlisting}

\subsection{YOLO 动态分辨率}\label{yolo-ux52a8ux6001ux5206ux8fa8ux7387}

\begin{lstlisting}
atc --model=yolov5s.onnx \
    --framework=5 \
    --output=yolov5s_640_768 \
    --dynamic_image_size="640,640;768,768" \
    --input_format=NCHW \
    --soc_version=Ascend310B \
    --op_select_implmode=high_performance \
    --precision_mode=allow_fp32_to_fp16
\end{lstlisting}

\subsection{INT8 量化(示例)}\label{int8-ux91cfux5316ux793aux4f8b}

\begin{lstlisting}
atc --model=resnet50.onnx \
    --framework=5 \
    --output=resnet50_int8 \
    --input_format=NCHW \
    --input_shape="input:1,3,224,224" \
    --soc_version=Ascend310B \
    --precision_mode=allow_mix_precision \
    --insert_op_conf=aipp.cfg \
    --enable_small_channel=true
\end{lstlisting}

\section{性能与质量
Checklist(执行勾项)}\label{ux6027ux80fdux4e0eux8d28ux91cf-checklistux6267ux884cux52feux9879}

性能:

\begin{itemize}
\tightlist
\item[$\square$]
  Profiling 无明显 Idle gap \textgreater{} 10\%
\item[$\square$]
  H2D + D2H 占比 \textless{} 25\%
\item[$\square$]
  Postprocess 占比 \textless{} 20\%
\item[$\square$]
  Stream 利用率平衡(无单流饱和)
\item[$\square$]
  使用内存池减少频繁 alloc/free
\end{itemize}

精度:

\begin{itemize}
\tightlist
\item[$\square$]
  ONNX vs OM Top1 差异 \textless{} 0.2\%
\item[$\square$]
  L1 平均误差 \textless{} 1e-3(FP16)
\item[$\square$]
  NMS 输出框数量与基线差异 \textless{} 1 框/图(平均)
\item[$\square$]
  INT8 校准集覆盖多场景
\end{itemize}

稳定性:

\begin{itemize}
\tightlist
\item[$\square$]
  1h 稳态无 Crash / OOM
\item[$\square$]
  温度在安全区间 \textless{} 85°C
\item[$\square$]
  看门狗重启次数 = 0
\end{itemize}

安全:

\begin{itemize}
\tightlist
\item[$\square$]
  日志无明文密钥
\item[$\square$]
  模型文件 hash 校验通过
\end{itemize}

\section{术语表(扩展)}\label{ux672fux8bedux8868ux6269ux5c55}

\begin{longtable}[]{@{}ll@{}}
\toprule\noalign{}
术语 & 说明 \\
\midrule\noalign{}
\endhead
\bottomrule\noalign{}
\endlastfoot
OM & Ascend 离线模型二进制格式 \\
ACL & Ascend 计算语言 API 层 \\
ATC & 模型转换/编译工具 \\
AIPP & 自动图像预处理模块 \\
Stream & 异步任务调度通道 \\
Profiling & 性能采样分析工具体系 \\
Fallback & 算子未匹配优化实现退回通用实现 \\
Quant Calibration & 量化尺度统计过程 \\
Baseline & 初始标准对照性能/精度集 \\
Drift & 指标随时间未经预期的漂移 \\
\end{longtable}

\section{推荐资源与外部引用}\label{ux63a8ux8350ux8d44ux6e90ux4e0eux5916ux90e8ux5f15ux7528}

\begin{itemize}
\tightlist
\item
  Ascend 官方文档入口(安装/算子列表/最佳实践)
\item
  CANN Release Notes:版本兼容与已知问题。
\item
  ONNX Operator 列表与语义说明。
\item
  Open Model Zoo / ModelScope:获取预训练模型与许可信息。
\item
  学术资源:算子融合、低比特量化、蒸馏相关论文列表。
\end{itemize}

\section{贡献指南摘要}\label{ux8d21ux732eux6307ux5357ux6458ux8981}

流程:Fork → 新分支 → 修改/新增 → 本地 lint \& 生成脚本 →
PR(描述动机/影响面/验证方式)。 PR 要求: \textbar{} 要素 \textbar{}
说明 \textbar{} \textbar{} ---- \textbar{} ---- \textbar{} \textbar{}
标题 \textbar{} 简明说明改动作用 \textbar{} \textbar{} 描述 \textbar{}
背景 + 修改点 + 风险 \textbar{} \textbar{} 验证 \textbar{}
性能/精度/功能截图或数据 \textbar{} \textbar{} 回滚 \textbar{}
若失败如何恢复 \textbar{} \textbar{} 关联 Issue \textbar{} 追踪链接
\textbar{}

\section{FAQ}\label{faq}

\begin{longtable}[]{@{}
  >{\raggedright\arraybackslash}p{(\linewidth - 2\tabcolsep) * \real{0.5000}}
  >{\raggedright\arraybackslash}p{(\linewidth - 2\tabcolsep) * \real{0.5000}}@{}}
\toprule\noalign{}
\begin{minipage}[b]{\linewidth}\raggedright
问题
\end{minipage} & \begin{minipage}[b]{\linewidth}\raggedright
回答
\end{minipage} \\
\midrule\noalign{}
\endhead
\bottomrule\noalign{}
\endlastfoot
模型转换慢怎么办? & 使用 SSD,关闭调试日志,检查不必要动态 shape。 \\
精度下降如何定位? & 离线脚本层级 Dump 比对,逐层二分。 \\
如何减少内存占用? & 启用内存池 + 减少中间冗余张量 + 固定 batch。 \\
量化后收益不明显? & 检查是否
Compute-bound,或激活分布集中导致尺度相近。 \\
NMS 很慢? & 合并小框批量处理/降低候选阈值/考虑 Device 版 NMS。 \\
\end{longtable}

\section{License 与引用}\label{license-ux4e0eux5f15ux7528}

本书内容遵循 Apache 2.0 许可证。引用: \textgreater{}
《昇腾310B实战:从入门到精通边缘计算与人工智能》(GitHub:
zhouxzh/Ascend310)

\section{版本路线回顾}\label{ux7248ux672cux8defux7ebfux56deux987e}

\begin{longtable}[]{@{}lll@{}}
\toprule\noalign{}
版本 & 内容 & 目标 \\
\midrule\noalign{}
\endhead
\bottomrule\noalign{}
\endlastfoot
v0.1 & 结构框架 & 验证框架可行 \\
v0.3 & 核心部署链路 & 形成可用主线 \\
v0.6 & 案例与工程化 & 贴近实战 \\
v1.0 & 全面审校发行 & 正式发布 \\
\end{longtable}

\section{实践任务}\label{ux5b9eux8df5ux4efbux52a1}

\begin{enumerate}
\def\labelenumi{\arabic{enumi}.}
\tightlist
\item
  为你的项目添加 1 条本地常见错误记录(含根因与解决)。
\item
  复制分类 ATC 模板并改写为检测模型版本(含动态尺寸)。
\item
  在术语表补充 3 个任务相关术语(并验证唯一性)。
\item
  选取 FAQ 一条,写出更深入排查脚本思路。
\end{enumerate}


% 第十章
\chapter{导读与准备工作}
\section{章节总览}\label{ux7ae0ux8282ux603bux89c8}

本章提供``鸟瞰 + 上手 + 约定 +
协作''四个维度:帮助读者在开始代码与实验前,建立清晰地图、完成环境自检、理解术语规范,并加入协作迭代。阅读后应能:A)
明确个人学习路径;B) 快速完成最小可行部署;C) 识别后续章节间的依赖关系。

\section{全书主线结构}\label{ux5168ux4e66ux4e3bux7ebfux7ed3ux6784}

技术主线:硬件与环境 (1) → 软件栈与转换 (2) → 边缘系统视角 (3) →
典型部署实践 (4) → 性能与算子优化 (5) → 高可用工程体系 (6) →
方法论与交付 (7) → 综合案例 (8) → 工具与附录 (9)。 知识图谱建议:

\begin{lstlisting}
硬件/板卡 → CANN 组件 → 模型转换 → 推理编程 → 多模型流水线 → 性能调优 → 系统可靠性 → 交付方法论 → 案例复现
\end{lstlisting}

\section{读者路径矩阵}\label{ux8bfbux8005ux8defux5f84ux77e9ux9635}

\begin{longtable}[]{@{}
  >{\raggedright\arraybackslash}p{(\linewidth - 8\tabcolsep) * \real{0.1111}}
  >{\raggedright\arraybackslash}p{(\linewidth - 8\tabcolsep) * \real{0.2222}}
  >{\raggedright\arraybackslash}p{(\linewidth - 8\tabcolsep) * \real{0.1667}}
  >{\raggedright\arraybackslash}p{(\linewidth - 8\tabcolsep) * \real{0.2222}}
  >{\raggedright\arraybackslash}p{(\linewidth - 8\tabcolsep) * \real{0.2778}}@{}}
\toprule\noalign{}
\begin{minipage}[b]{\linewidth}\raggedright
角色
\end{minipage} & \begin{minipage}[b]{\linewidth}\raggedright
起步路径
\end{minipage} & \begin{minipage}[b]{\linewidth}\raggedright
可跳过
\end{minipage} & \begin{minipage}[b]{\linewidth}\raggedright
深挖章节
\end{minipage} & \begin{minipage}[b]{\linewidth}\raggedright
目标里程碑
\end{minipage} \\
\midrule\noalign{}
\endhead
\bottomrule\noalign{}
\endlastfoot
零基础 & 1 → 2 → 4 & 5 深度优化细节 & 8 案例 & 跑通首个端到端推理 \\
嵌入式 & 1 → 2 → 5 → 6 & 7 方法论部分 & 5/6 性能与可靠性 &
优化资源占比 \\
AI 应用 & 2 → 4 → 7 → 8 & 1 硬件细节 & 4/8 部署差异 & 多任务流水线 \\
技术负责人 & 0 → 3 → 6 → 7 & 具体算子实现 & 7 评测体系 & 制定团队标准 \\
\end{longtable}

\section{硬件准备与兼容性}\label{ux786cux4ef6ux51c6ux5907ux4e0eux517cux5bb9ux6027}

\begin{longtable}[]{@{}llll@{}}
\toprule\noalign{}
组件 & 推荐 & 说明 & 检查点 \\
\midrule\noalign{}
\endhead
\bottomrule\noalign{}
\endlastfoot
开发板 & OrangePi AIpro 310B & 标准平台 & npu-smi 识别型号 \\
存储 & TF 64G+ / SSD & 加速 I/O & iostat 延迟 \textless10ms \\
散热 & 风扇+鳍片 & 长时间稳定 & 温度 \textless{} 85°C \\
摄像头 & USB UVC / MIPI & 即插即用 & v4l2-ctl 列设备 \\
网络 & 千兆以太网 & 低抖动 & ping 丢包率 ≈0 \\
电源 & PD 65W & 稳定供电 & 无随机重启 \\
\end{longtable}

准备完成后记录
\passthrough{\lstinline!hardware\_inventory.md!}:型号、序列号、固件版本、功耗模式。

\section{软件与工具栈细化}\label{ux8f6fux4ef6ux4e0eux5de5ux5177ux6808ux7ec6ux5316}

\begin{longtable}[]{@{}lll@{}}
\toprule\noalign{}
层级 & 工具/组件 & 说明 \\
\midrule\noalign{}
\endhead
\bottomrule\noalign{}
\endlastfoot
OS & Ubuntu 22.04 / openEuler & 官方验证环境 \\
驱动/固件 & 对应 CANN 版本 & 版本矩阵对齐 \\
CANN & Toolkit + Runtime & 提供 atc/acl/profiling \\
Python & 3.10+ & 脚本与评测 \\
依赖 & numpy/onnx/onnxruntime/opencv & 模型与预处理 \\
调试 & npu-smi/Profiler/日志系统 & 性能与稳定性分析 \\
\end{longtable}

建议创建 \passthrough{\lstinline!requirements.txt!} 并使用 venv 或 Conda
隔离。

\section{仓库目录与命名约定}\label{ux4ed3ux5e93ux76eeux5f55ux4e0eux547dux540dux7ea6ux5b9a}

\begin{longtable}[]{@{}lll@{}}
\toprule\noalign{}
目录 & 内容 & 约定 \\
\midrule\noalign{}
\endhead
\bottomrule\noalign{}
\endlastfoot
src/book & 文本章节 & 章节号前缀固定 \\
experiments & 案例 & caseX 模式 \\
models & 原始/导出中间模型 & 按模型名/版本 \\
scripts & 通用脚本 & 跨平台 \passthrough{\lstinline!.sh/.ps1!} \\
tools & 辅助分析脚本 & 单一功能命令化 \\
docs & 生成 PDF / 图 & 不放大模型文件 \\
benchmarks & 性能记录 & 时间戳 + commit \\
\end{longtable}

命名:\passthrough{\lstinline!<model>\_<precision>\_<shape>.om!},例如
\passthrough{\lstinline!yolov5s\_fp16\_1x3x640x640.om!}。

\section{最小可行环境验证
(MVE)}\label{ux6700ux5c0fux53efux884cux73afux5883ux9a8cux8bc1-mve}

执行脚本 \passthrough{\lstinline!scripts/verify\_env.sh!}(建议添加):

\begin{enumerate}
\def\labelenumi{\arabic{enumi}.}
\tightlist
\item
  \passthrough{\lstinline!npu-smi info!}:输出芯片与状态;
\item
  \passthrough{\lstinline!atc --version!}:版本号记录;
\item
  运行随机张量推理(内置简单 OM 或最小网络)验证 ACL API;
\item
  Profiling 采集一次,生成 timeline 文件;
\item
  记录结果写入 \passthrough{\lstinline!env\_report.json!}。
\end{enumerate}

判定:如某步骤失败阻断后续章节学习。

\section{全局术语与约定}\label{ux5168ux5c40ux672fux8bedux4e0eux7ea6ux5b9a}

\begin{longtable}[]{@{}lll@{}}
\toprule\noalign{}
术语 & 约定 & 说明 \\
\midrule\noalign{}
\endhead
\bottomrule\noalign{}
\endlastfoot
FPS & frames/second & 统计处理输出帧数 \\
Latency & ms & 端到端完成时间 \\
Pxx & 分位数 & P95/P99 评估抖动 \\
Pipeline & 阶段组 & 多阶段并行结构 \\
Signature & 模型签名 & I/O 名称与形状/格式 json \\
Baseline & 初始基线 & 第一版性能/精度记录 \\
\end{longtable}

所有时间单位默认 ms;数据大小默认字节(显式写 MB/GiB
时需指出换算基数)。

\section{协作工作流与质量闸门}\label{ux534fux4f5cux5de5ux4f5cux6d41ux4e0eux8d28ux91cfux95f8ux95e8}

工作流:Issue(需求/缺陷)→ 分支
\passthrough{\lstinline!feat|fix/<topic>!} → 提交(含描述)→ PR →
自动测试(Lint+精度/性能轻测)→ Review → Merge。 质量闸门:

\begin{longtable}[]{@{}lll@{}}
\toprule\noalign{}
闸门 & 说明 & 未通过处理 \\
\midrule\noalign{}
\endhead
\bottomrule\noalign{}
\endlastfoot
Lint & 代码/文档格式 & 修复后再提交 \\
Spell & 关键术语拼写 & 更正 \\
Signature 验证 & 模型签名一致 & 拒绝合并 \\
基线回归 & 性能/精度差异超阈值 & 标注需说明 \\
\end{longtable}

PR 模板字段:Motivation / Changes / Test / Risk / Rollback Plan。

\section{学习与实践建议}\label{ux5b66ux4e60ux4e0eux5b9eux8df5ux5efaux8bae}

\begin{enumerate}
\def\labelenumi{\arabic{enumi}.}
\tightlist
\item
  完成前 3 章后立即挑选一个轻量模型跑通部署(建立正反馈)。
\item
  每章输出``总结卡片'':知识点 → 应用场景 → 潜在风险。
\item
  建议建立个人实验日志:参数、结果、疑问与下一步假设。
\item
  失败样本收集:创建 \passthrough{\lstinline!failure\_cases/!}
  目录存储误检/漏检图像用于持续改进。
\end{enumerate}

\section{常见初学误区与规避}\label{ux5e38ux89c1ux521dux5b66ux8befux533aux4e0eux89c4ux907f}

\begin{longtable}[]{@{}lll@{}}
\toprule\noalign{}
误区 & 结果 & 规避 \\
\midrule\noalign{}
\endhead
\bottomrule\noalign{}
\endlastfoot
直接优化无基线 & 无从评估收益 & 先建立 baseline \\
混用不同预处理 & 精度随机波动 & 抽象统一函数 \\
缺少签名文件 & 部署时出错 & 每次转换生成签名 \\
未记录环境版本 & 难以复现 & env\_report.json \\
长日志未切割 & 磁盘占满 & 配置滚动策略 \\
\end{longtable}

\section{章节小结}\label{ux7ae0ux8282ux5c0fux7ed3}

通过环境、目录、术语、协作流程的标准化,后续学习聚焦问题本身,而不是环境与沟通摩擦。建议读者在继续前先完成``最小可行环境验证''并记录结果,以便后续调试时快速排除环境因素。

\section{实践任务}\label{ux5b9eux8df5ux4efbux52a1}

\begin{enumerate}
\def\labelenumi{\arabic{enumi}.}
\tightlist
\item
  撰写 \passthrough{\lstinline!hardware\_inventory.md!} 与
  \passthrough{\lstinline!env\_report.json!}(可手动草拟)。
\item
  建立 \passthrough{\lstinline!requirements.txt!}
  并安装依赖,记录安装耗时。
\item
  创建一个最小随机张量 OM 推理脚本并输出结果摘要。
\item
  制定个人 4 周学习计划(章节→目标→产出)。
\end{enumerate}


% 案例0
\chapter{案例0}
\input{cases/case0.tex}

% 案例1
\chapter{案例1:智能人脸识别打卡机}
\section{\#
案例1:智能人脸识别打卡机}\label{ux6848ux4f8b1ux667aux80fdux4ebaux8138ux8bc6ux522bux6253ux5361ux673a}

\section{项目简介}\label{ux9879ux76eeux7b80ux4ecb}

本项目旨在利用昇腾310B的强大AI算力,构建一个功能完整、响应迅速的智能人脸识别打卡系统。系统通过USB摄像头实时捕捉视频流,检测画面中的人脸,并与预先注册的员工/学生人脸数据库进行比对,完成身份验证和自动记录考勤。

该项目不仅是一个功能性的应用,更是一个端到端的AI实践案例,涵盖了从硬件选型、软件环境搭建、数据准备、模型训练与优化,到最终在边缘设备上部署的全过程。

\section{内容大纲}\label{ux5185ux5bb9ux5927ux7eb2}

\subsection{硬件准备}\label{ux786cux4ef6ux51c6ux5907}

\begin{itemize}
\tightlist
\item
  \textbf{核心计算单元}: 昇腾310B开发者套件
\item
  \textbf{图像采集}: USB摄像头 (推荐罗技C920或同等规格)
\item
  \textbf{显示设备 (可选)}: HDMI显示器,用于实时预览或UI展示
\item
  \textbf{外设}: 键盘、鼠标
\item
  \textbf{电源}: 为昇腾310B开发板提供稳定供电
\item
  \textbf{连接线}: HDMI线, USB-C数据线等
\end{itemize}

\emph{附:硬件连接示意图} \textgreater{}
(此处可插入一张图片,清晰展示所有硬件的连接方式)

\subsection{软件环境}\label{ux8f6fux4ef6ux73afux5883}

\begin{itemize}
\tightlist
\item
  \textbf{操作系统}: Ubuntu 20.04 或 openEuler
\item
  \textbf{CANN版本}: 7.0或更高
\item
  \textbf{Python版本}: 3.8.x
\item
  \textbf{主要依赖库}:

  \begin{itemize}
  \tightlist
  \item
    \passthrough{\lstinline!opencv-python!}: 用于图像和视频处理
  \item
    \passthrough{\lstinline!numpy!}: 用于数值计算
  \item
    \passthrough{\lstinline!scikit-learn!}: 用于评估模型性能
  \item
    \passthrough{\lstinline!onnxruntime!}: 用于运行ONNX模型
  \item
    \passthrough{\lstinline!PyQt5!} (可选): 用于构建图形用户界面
  \end{itemize}
\end{itemize}

\emph{附:一键安装环境脚本 (\passthrough{\lstinline!install\_env.sh!})}

\begin{lstlisting}[language=bash]
# 示例脚本
sudo apt update
sudo apt install -y python3-pip python3-opencv
pip3 install numpy scikit-learn onnxruntime
# ... 其他依赖安装命令
\end{lstlisting}

\subsection{数据集准备}\label{ux6570ux636eux96c6ux51c6ux5907}

\begin{itemize}
\item
  \textbf{数据集来源}:

  \begin{enumerate}
  \def\labelenumi{\arabic{enumi}.}
  \tightlist
  \item
    \textbf{公开数据集}: 如LFW (Labeled Faces in the
    Wild)、CASIA-WebFace等。
  \item
    \textbf{自建数据集}:
    推荐!使用摄像头为每位用户(员工/学生)拍摄多张、多角度、不同光照和表情的人脸照片。
  \end{enumerate}
\item
  \textbf{数据组织结构}:

\begin{lstlisting}
datasets/
├── zhang_san/
│   ├── 001.jpg
│   ├── 002.jpg
│   └── ...
├── li_si/
│   ├── 001.jpg
│   ├── 002.jpg
│   └── ...
└── ...
\end{lstlisting}
\item
  \textbf{预处理脚本 (\passthrough{\lstinline!preprocess.py!})}:

  \begin{itemize}
  \tightlist
  \item
    人脸检测与对齐
  \item
    图像增强(旋转、裁剪、调整亮度等)
  \item
    划分训练集和验证集
  \end{itemize}
\end{itemize}

\subsection{模型训练}\label{ux6a21ux578bux8badux7ec3}

\begin{itemize}
\tightlist
\item
  \textbf{模型选择}:

  \begin{itemize}
  \tightlist
  \item
    \textbf{人脸检测}: MTCNN 或 RetinaFace
  \item
    \textbf{人脸识别}: ArcFace, CosFace, 或 MobileFaceNet
    (推荐,因其轻量高效)
  \end{itemize}
\item
  \textbf{训练流程}:

  \begin{enumerate}
  \def\labelenumi{\arabic{enumi}.}
  \tightlist
  \item
    使用预处理好的数据集进行模型训练。
  \item
    调整超参数(学习率、批大小等)以获得最佳性能。
  \item
    在验证集上评估模型准确率、召回率等指标。
  \end{enumerate}
\item
  \textbf{模型导出}:
  将训练好的PyTorch或TensorFlow模型转换为昇腾亲和的ONNX格式。
\end{itemize}

\subsection{模型部署}\label{ux6a21ux578bux90e8ux7f72}

\begin{itemize}
\item
  \textbf{模型转换}: 使用ATC (Ascend Tensor Compiler)
  工具将ONNX模型转换为昇腾310B支持的\passthrough{\lstinline!.om!}离线模型。

\begin{lstlisting}[language=bash]
atc --model=./face_recognition.onnx --framework=5 --output=./face_recognition --input_format=NCHW --input_shape="data:1,3,112,112" --soc_version=Ascend310B1
\end{lstlisting}
\item
  \textbf{部署代码 (\passthrough{\lstinline!main.py!})}:

  \begin{enumerate}
  \def\labelenumi{\arabic{enumi}.}
  \tightlist
  \item
    初始化CANN和ACL (Ascend Computing Language) 资源。
  \item
    加载\passthrough{\lstinline!.om!}离线模型。
  \item
    循环读取摄像头帧。
  \item
    对每一帧进行人脸检测和识别推理。
  \item
    将识别结果与数据库比对,输出姓名。
  \item
    在画面上绘制矩形框和姓名,并记录打卡时间。
  \end{enumerate}
\end{itemize}

\subsection{3D打印结构件}\label{dux6253ux5370ux7ed3ux6784ux4ef6}

为了方便地将摄像头固定在合适的位置,我们设计了专用的摄像头支架和设备保护外壳。
- \textbf{文件列表}: - \passthrough{\lstinline!camera\_holder.stl!}:
摄像头支架模型文件 - \passthrough{\lstinline!device\_case.stl!}:
设备外壳模型文件 - \textbf{打印建议}: - \textbf{材料}: PLA 或 PETG -
\textbf{层高}: 0.2mm - \textbf{填充率}: 20\%

\subsection{用户手册}\label{ux7528ux6237ux624bux518c}

\begin{enumerate}
\def\labelenumi{\arabic{enumi}.}
\tightlist
\item
  \textbf{硬件组装}:
  参照\passthrough{\lstinline!2.1!}节的连接图连接好所有硬件。
\item
  \textbf{环境配置}:
  运行\passthrough{\lstinline!install\_env.sh!}脚本安装所有软件依赖。
\item
  \textbf{人脸注册}:
  运行\passthrough{\lstinline!register\_face.py!}脚本,根据提示输入姓名,并拍摄多张人脸照片进行注册。
\item
  \textbf{启动系统}:
  运行\passthrough{\lstinline!main.py!}启动人脸识别打卡程序。
\item
  \textbf{查看记录}:
  打卡记录将保存在\passthrough{\lstinline!attendance.csv!}文件中。
\end{enumerate}

\section{源代码}\label{ux6e90ux4ee3ux7801}

\begin{quote}
(此处未来可替换为GitHub仓库链接或详细的文件树)
\end{quote}

\section{效果演示}\label{ux6548ux679cux6f14ux793a}

\begin{quote}
(此处可插入系统运行时的截图或GIF动图,例如摄像头实时识别人脸的画面)
\end{quote}


% 参考文献
% \backmatter
% \chapter{参考文献}

% \nocite{*} % 显示所有参考文献,即使未被引用
% \bibliographystyle{plain} % 参考文献样式
% \bibliography{references} % 参考文献数据库

% % 附录
% \appendix
% \chapter{附录标题}

% 这里是附录内容。

\end{document}