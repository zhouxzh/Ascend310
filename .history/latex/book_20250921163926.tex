% !TEX program = xelatex
% 使用 XeLaTeX 编译
\documentclass[fontsize=12pt, paper=a4, twoside, openright, DIV=calc]{scrbook}

% 基本设置
\usepackage[UTF8]{ctex} % 中文支持
\usepackage{fontspec} % 字体设置
\usepackage{xeCJK} % 中日韩文字支持

% 设置默认字体为无衬线体(思源黑体)
\setCJKmainfont{Noto Serif SC}
% \setCJKmainfont{Noto Serif CJK SC} % 设置罗马族字体为思源宋体
% \setCJKsansfont{Noto Sans CJK SC}  % 设置无衬线族字体为思源黑体
% \setmonofont{Noto Sans Mono CJK SC} % 设置等宽族字体(可选)
% 页面布局和外观设置
\usepackage[automark,headsepline]{scrlayer-scrpage} % 页眉页脚
\clearpairofpagestyles
\ohead{\headmark} % 页眉显示章节名
\ofoot[\pagemark]{\pagemark} % 页脚显示页码

% 数学相关包
\usepackage{amsmath, amssymb, amsthm}

% 图形和颜色
\usepackage{graphicx}
\usepackage{xcolor}
\definecolor{mainblue}{RGB}{0, 90, 156}

% 超链接设置
\usepackage{hyperref}
\hypersetup{
    colorlinks=true,
    linkcolor=mainblue,
    citecolor=mainblue,
    urlcolor=mainblue,
    pdftitle={昇腾310B实战——从入门到精通边缘计算},
    pdfauthor={周贤中}
}

% 定理环境设置
\newtheoremstyle{break}
  {\topsep}{\topsep}%
  {\itshape}{}%
  {\bfseries}{}%
  {\newline}{}%
\theoremstyle{break}
\newtheorem{theorem}{定理}[chapter]
\newtheorem{lemma}[theorem]{引理}
\newtheorem{proposition}[theorem]{命题}
\newtheorem{corollary}[theorem]{推论}
\newtheorem{definition}[theorem]{定义}
\newtheorem{example}[theorem]{例子}
\newtheorem{remark}[theorem]{注记}

% 文档信息
\title{昇腾310B实战——从入门到精通边缘计算}
\author{周贤中}
\date{\today}

% 文档内容开始
\begin{document}

% 封面页
\maketitle

% 前言页
\frontmatter

% % 摘要
% \begin{abstract}
% 这是一段中文摘要内容。本文档是一个使用XeLaTeX和KOMA-Script scrbook类制作的中文书籍模板,采用了开源中文字体。模板包含了书籍排版常用的各种元素和设置,适合撰写中文书籍、论文或报告。
% \end{abstract}

% 目录
\tableofcontents

% 主内容开始
\mainmatter

% 第一章
\chapter{引言}
昇腾310B实战——从入门到精通边缘计算

\section{第一节}

这是一段示例文本,使用思源宋体显示。XeLaTeX 结合开源中文字体可以产生高质量的中文排版效果。\textbf{这是粗体文字},\textit{这是斜体文字}。

\subsection{小节}

数学公式示例:爱因斯坦的质能方程 $E = mc^2$,以及行内公式 $\int_a^b f(x) dx$。

多行公式示例:
\begin{equation}
\frac{\partial u}{\partial t} = \alpha \nabla^2 u
\end{equation}

\section{第二节}

\subsection{列表环境}

无序列表示例:
\begin{itemize}
\item 第一项
\item 第二项
\item 第三项
\end{itemize}

有序列表示例:
\begin{enumerate}
\item 第一项
\item 第二项
\item 第三项
\end{enumerate}

\subsection{表格环境}

表格示例:
\begin{table}[htbp]
\centering
\caption{示例表格}
\begin{tabular}{|c|c|c|}
\hline
姓名 & 年龄 & 职业 \\
\hline
张三 & 25 & 工程师 \\
李四 & 30 & 教师 \\
王五 & 28 & 医生 \\
\hline
\end{tabular}
\end{table}

% 第二章
\chapter{主要内容}

\section{定理环境示例}

\begin{definition}[可微函数]
设函数 $f: U \subseteq \mathbb{R}^n \to \mathbb{R}$,如果存在线性映射 $A: \mathbb{R}^n \to \mathbb{R}$ 使得
\[
f(x + h) = f(x) + A(h) + o(\|h\|)
\]
则称 $f$ 在点 $x$ 处可微。
\end{definition}

\begin{theorem}[中值定理]
如果函数 $f(x)$ 在闭区间 $[a, b]$ 上连续,在开区间 $(a, b)$ 内可导,则存在一点 $c \in (a, b)$ 使得
\[
f'(c) = \frac{f(b) - f(a)}{b - a}
\]
\end{theorem}

\begin{proof}
这是定理的证明过程。可以使用罗尔定理来证明中值定理。
\end{proof}

\begin{example}
这是一个例子环境,用于展示示例内容。
\end{example}

% 参考文献
\backmatter
\chapter{参考文献}

\nocite{*} % 显示所有参考文献,即使未被引用
\bibliographystyle{plain} % 参考文献样式
\bibliography{references} % 参考文献数据库

% 附录
\appendix
\chapter{附录标题}

这里是附录内容。

\end{document}