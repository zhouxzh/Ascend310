书名中的 ``实战'',核心是
``在编程实践中学习''。本书作为昇腾310B芯片的入门指南,将跳出单纯的理论讲解,通过真实的AI推理部署案例,带读者直观理解昇腾310B的硬件架构特性、Atlas工具链使用逻辑与端侧AI项目开发流程
------
从模型适配、量化优化到推理服务部署,每一个知识点都配套可落地的代码示例,让读者在动手编码的过程中,真正掌握昇腾310B的实战应用能力,实现从
``了解芯片'' 到 ``能用芯片落地项目'' 的跨越。

\subsection*{本书定位与目标读者}\label{本书定位与目标读者}

本书面向以下三类读者: - \textbf{高校学生 /
科研新人}:希望通过一套系统化路径快速理解边缘AI硬件与部署流程。 -
\textbf{嵌入式 / IoT 工程师}:已有一定Linux / C /
Python基础,希望把AI模型真正跑在边缘端并做性能调优。 -
\textbf{AI应用开发者 /
创客}:已经能使用主流深度学习框架,希望将训练好的模型迁移到昇腾310B进行高效推理与产品化落地。

阅读预期: -
零基础读者可依照``快速起步路径''完成第1\textasciitilde3章+精选案例; -
进阶读者可继续深入算子优化、系统整合与复杂多模型协同部署; -
有项目诉求的团队可参考``方法论 +
附录模板''直接搭建属于自己的边缘AI应用。

\subsection*{学习路径速览(建议路线)}\label{学习路径速览建议路线}

\begin{enumerate}
\def\labelenumi{\arabic{enumi}.}
\tightlist
\item
  环境 + 工具链:硬件认知 → 开发环境装配 → CANN工具初试
\item
  基础模型部署:图像分类→ 目标检测 → 语义分割 / OCR / NLP
\item
  性能优化:模型结构裁剪 → 精度-性能权衡(FP16 / INT8)→ 并行与pipeline
\item
  低级能力:自定义算子 → Profiling → ACL / GE 原理 → 内存与数据通路调优
\item
  系统构建:多进程/多模型协同 → 任务调度 → 监控与日志体系
\item
  实战案例:从需求分析 → 方案设计 → 模型适配 → 部署脚本 → 交付验收
\end{enumerate}

\subsection*{全书结构(初版规划)}\label{全书结构初版规划}

\begin{longtable}[]{@{}
  >{\raggedright\arraybackslash}p{(\linewidth - 6\tabcolsep) * \real{0.2500}}
  >{\raggedright\arraybackslash}p{(\linewidth - 6\tabcolsep) * \real{0.2500}}
  >{\raggedright\arraybackslash}p{(\linewidth - 6\tabcolsep) * \real{0.2500}}
  >{\raggedright\arraybackslash}p{(\linewidth - 6\tabcolsep) * \real{0.2500}}@{}}
\toprule\noalign{}
\begin{minipage}[b]{\linewidth}\raggedright
模块
\end{minipage} & \begin{minipage}[b]{\linewidth}\raggedright
章节标题
\end{minipage} & \begin{minipage}[b]{\linewidth}\raggedright
内容聚焦
\end{minipage} & \begin{minipage}[b]{\linewidth}\raggedright
读者收益
\end{minipage} \\
\midrule\noalign{}
\endhead
\bottomrule\noalign{}
\endlastfoot
Part 0 & 导读与准备 & 芯片特性、开发形态、学习地图 & 建立整体心智模型 \\
Part 1 & 昇腾310B硬件与环境 & 硬件结构、固件、驱动、系统配置、容器化 &
能独立搭建可复现环境 \\
Part 2 & CANN 软件栈核心 & CANN组件、ATC模型转换、OM模型结构、ACL编程 &
掌握模型从框架到OM全过程 \\
Part 3 & 边缘计算基础 & 边缘计算价值、典型架构、数据流、协同模式 &
会做架构选型与资源拆分 \\
Part 4 & 模型部署实战 & 分类/检测/NLP/多模态部署、性能测试、批处理与流式
& 会把主流任务完整迁移上板 \\
Part 5 & 性能与算子优化 &
Profiler使用、数据对齐、算子融合、自定义算子开发 &
会定位瓶颈并提升帧率/延迟 \\
Part 6 & 系统工程方法 & 多模型编排、任务调度、异常恢复、日志与监控 &
会构建工程级可维护部署系统 \\
Part 7 & 项目实战方法论 & 需求拆解、Baseline迭代、评测体系、交付模板 &
会组织团队快速交付边缘AI项目 \\
Part 8 & 典型综合案例 & 9个端侧AI实战案例(与 experiments 配套) &
迁移复用案例形成生产力 \\
Part 9 & 附录与工具 & FAQ、性能Checklist、脚手架模板、术语表 &
快速检索与复用加速迭代 \\
\end{longtable}

\subsection*{各模块核心要点概述}\label{各模块核心要点概述}

\subsection{Part 1
昇腾310B硬件与环境}\label{part-1-ux6607ux817e310bux786cux4ef6ux4e0eux73afux5883}

\begin{itemize}
\tightlist
\item
  SoC架构(昇腾AI Core、内存层次、带宽特性)
\item
  开发板接口与外设(摄像头/存储/网络)
\item
  固件刷新 \& 系统初始化
\item
  Docker / 容器化开发与远程调试
\end{itemize}

\subsection{Part 2 CANN
软件栈与模型转换}\label{part-2-cann-ux8f6fux4ef6ux6808ux4e0eux6a21ux578bux8f6cux6362}

\begin{itemize}
\tightlist
\item
  CANN组件:Driver / Runtime / Compiler / Toolkit
\item
  ATC模型转换参数详解(shape、输入格式、最优算子选择)
\item
  OM模型结构与可视化
\item
  ACL编程流程(初始化 → 内存 → 推理 → 释放)
\item
  常见错误(推理精度损失 / 内存不足 / 算子不支持)定位
\end{itemize}

\subsection{Part 3
边缘计算原理}\label{part-3-ux8fb9ux7f18ux8ba1ux7b97ux539fux7406}

\begin{itemize}
\tightlist
\item
  边云协同模式:云训练 + 边缘推理
\item
  数据生命周期:采集→预处理→推理→缓存→上报
\item
  典型架构模式(单板/多板/异构协同)
\item
  边缘QoS:功耗、热设计、延迟、稳定性
\end{itemize}

\subsection{Part 4
模型部署与优化实践}\label{part-4-ux6a21ux578bux90e8ux7f72ux4e0eux4f18ux5316ux5b9eux8df5}

\begin{itemize}
\tightlist
\item
  图像分类(ResNet / MobileNet)
\item
  目标检测(YOLO / FasterRCNN)
\item
  OCR \& NLP(文本检测 + 轻量文本识别 / 中文BERT推理)
\item
  多模型串联(检测→裁剪→分类)Pipeline设计
\item
  精度 vs.~性能:Batch、FP16、算子融合、降采样策略
\end{itemize}

\subsection{Part 5
低级算子与性能调优}\label{part-5-ux4f4eux7ea7ux7b97ux5b50ux4e0eux6027ux80fdux8c03ux4f18}

\begin{itemize}
\tightlist
\item
  Profiling 工具使用(时间线 / 算子耗时 / 内存峰值)
\item
  数据对齐与内存复用策略
\item
  常用自定义算子开发模板(算子描述 → 编译 → 集成)
\item
  典型瓶颈案例:数据搬运 \textgreater{} 计算、Host/Device同步等待
\end{itemize}

\subsection{Part 6
系统集成与工程实践}\label{part-6-ux7cfbux7edfux96c6ux6210ux4e0eux5de5ux7a0bux5b9eux8df5}

\begin{itemize}
\tightlist
\item
  任务调度(多进程、多线程、异步队列)
\item
  资源隔离与监控(显存 / Host内存 / 温度 / 带宽)
\item
  高可用设计:看门狗、超时熔断、故障降级
\item
  交付形态:容器镜像 / 一键部署脚本 / OTA升级
\end{itemize}

\subsection{Part 7
项目实战方法论}\label{part-7-ux9879ux76eeux5b9eux6218ux65b9ux6cd5ux8bba}

\begin{itemize}
\tightlist
\item
  需求澄清 \& 场景指标设定(Latency / FPS / Accuracy / Power)
\item
  Baseline快速验证:裁剪 vs.~重构 vs.~迁移
\item
  评测体系:功能、性能、稳定性、可维护性
\item
  SRE视角的上线准备 Checklist
\end{itemize}

\subsection{\texorpdfstring{Part 8 综合实战案例(与
\texttt{src/experiment}
配套)}{Part 8 综合实战案例(与 src/experiment 配套)}}\label{part-8-ux7efcux5408ux5b9eux6218ux6848ux4f8bux4e0e-srcexperiment-ux914dux5957}

包含 9
大可复现案例:人脸打卡机、实时跟踪、智能电子琴、掌纹识别、数据采集仪、智能小车、智能相册、手势识别、聊天机器人。每个案例均提供:
- 需求说明 \& 功能结构图 - 模型与数据选择依据 - 转换 \& 部署脚本 -
性能测试报告(延迟 / 吞吐 / 资源占用) - 可选 3D 打印结构件与装配说明

\subsection{Part 9
附录与工具箱}\label{part-9-ux9644ux5f55ux4e0eux5de5ux5177ux7bb1}

\begin{itemize}
\tightlist
\item
  常见报错速查表(ATC / ACL / Runtime)
\item
  模型转换与部署参数模板
\item
  性能调优Checklist(内存 / 数据流 / 并行 / 算子)
\item
  术语表 / 推荐资料 / 贡献指南
\end{itemize}

\subsection*{实践驱动与开源协作}\label{实践驱动与开源协作}

本书所有示例代码、脚本、案例与附录工具均开源托管于本仓库。欢迎通过 Issue
/ PR 反馈问题、提交改进、补充案例或翻译。我们鼓励: - 增补新模型 /
新任务的部署范式 - 分享自定义算子优化经验 - 提交性能测试报告(含硬件信息
+ 指标) - 翻译与文档校对

\subsection*{如何使用本书}\label{如何使用本书}

\begin{longtable}[]{@{}
  >{\raggedright\arraybackslash}p{(\linewidth - 6\tabcolsep) * \real{0.2500}}
  >{\raggedright\arraybackslash}p{(\linewidth - 6\tabcolsep) * \real{0.2500}}
  >{\raggedright\arraybackslash}p{(\linewidth - 6\tabcolsep) * \real{0.2500}}
  >{\raggedright\arraybackslash}p{(\linewidth - 6\tabcolsep) * \real{0.2500}}@{}}
\toprule\noalign{}
\begin{minipage}[b]{\linewidth}\raggedright
读者类型
\end{minipage} & \begin{minipage}[b]{\linewidth}\raggedright
推荐阅读路径
\end{minipage} & \begin{minipage}[b]{\linewidth}\raggedright
目标
\end{minipage} & \begin{minipage}[b]{\linewidth}\raggedright
补充建议
\end{minipage} \\
\midrule\noalign{}
\endhead
\bottomrule\noalign{}
\endlastfoot
零基础学生 & Part1 → Part2 → Part4(入门任务) & 能跑通首个模型 &
结合案例做改动实验 \\
嵌入式工程师 & Part1 → Part2 → Part5 → Part6 & 掌握部署与优化 &
关注资源与稳定性章节 \\
AI应用开发者 & Part2 → Part4 → Part7 → Part8 & 快速场景落地 &
记录调参与性能差异 \\
技术负责人 & Part0 → Part3 → Part6 → Part7 & 构建团队方法论 &
制定内部模板体系 \\
\end{longtable}

\subsection*{更新与版本计划}\label{更新与版本计划}

\begin{itemize}
\tightlist
\item
  v0.1(当前):结构规划 + 前3章草稿 + 2个示例案例
\item
  v0.3:补齐核心部署链路 + 性能调优初稿
\item
  v0.6:全案例上线 + 工程化章节完善
\item
  v1.0:补齐附录 + 全面审校 + PDF / LaTeX 发行
\end{itemize}

\subsection*{许可证与引用}\label{许可证与引用}

本书内容采用 Apache 2.0 许可证。引用本书内容请注明: \textgreater{}
《昇腾310B实战:从入门到精通边缘计算与人工智能》(GitHub:
zhouxzh/Ascend310)

\begin{center}\rule{0.5\linewidth}{0.5pt}\end{center}

欢迎加入共建,一起把``边缘AI实战''这件事做成!
